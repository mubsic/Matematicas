%capitulo 1

\chapter{Axiomas de Cuerpo}
\begin{center}
\begin{tcolorbox}
\begin{axioma}[Propiedad conmutativa]
$$x+y = y+x, \; \; xy=yx$$
\end{axioma}

\begin{axioma}[Propiedad Asociativa]
$$x+(y+z)=(x+y)+z, \; \; x(yz) = (xy)z $$
\end{axioma}

\begin{axioma}[propiedad distributiva]
$$x(y+z)=xy+xz$$
\end{axioma}

\begin{axioma}[Existencia de elementos neutros]
Existen dos números reales distintos, que se indican por 0 y 1 tales que para cada número real $x$ se tiene:
$$0+x = x+0=x \; \; y \; \; 1\cdot x = x\cdot 1 = x$$
\end{axioma}

\begin{axioma}[Existencia de negativos]
Para cada número real $x$ existe un número real y tal que:
$$x+y=y+x=0$$
\end{axioma}

\begin{axioma}[Existencia del recíproco]
Para cada número real $x\neq 0$ existe un número real y tal que:
$$xy = yx = 1$$
\end{axioma}
\end{tcolorbox}
\end{center}


\section[Teoremas]{Teoremas\footnote{Tom Apostol Vol. 1 pag. 22-23}} 
%teorema 1
\begin{teo}[Ley de simplificación para la suma]
Si $a+b=a+c$ entonces $b=c$ (En particular esto prueba que el número 0 del axioma 4 es único)\\\\
Demostración.- \;
Dado a+b=a+c. En virtud de la existencia de negativos, se puede elegir y de manera que $y+a=0$, con lo cual $y+(a+b)=y+(a+c)$ y aplicando la propiedad asociativa tenemos $(y+a)+b=(y+a)+c$ entonces, $0+b=0+c$. En virtud de la existencia de elementos neutros, se tiene $b=c$.\\
por otro lado este teorema demuestra que existe un solo número real que tiene la propiedad del 0 en el axioma 4. En efecto, si $0$ y $0^{'}$ tuvieran ambos esta propiedad, entonces $0+0^{'}=0$ y $0+0=0$; por lo tanto, $0+0^{'}=0+0$ y por la ley de simplificación para la suma $0=0^{'}$\\\\
\end{teo}

%teorema2
\begin{teo}[Posibilidad de la sustracción]
Dado $a$ y $b$ existe uno y sólo un $x$ tal que $a+x=b$. Este $x$ se designa por $b-a$. En particular $0-a$ se escribe simplemente $-a$ y se denomina el negativo de $a$\\\\
Demostración.- \;
Dados $a$ y $b$ por el axioma 5 se tiene $y$ de manera que $a + y = 0$ ó $y=-a$, por hipótesis y teorema tenemos que $x=b-a$ sustituyendo $y$ tenemos $x=b+y$ y propiedad conmutativa $x=y+b$, entonces $a+x=a+(y+b)=(a+y)+b=0+b=b$ esto por sustitución, propiedad asociativa y propiedad de neutro, Por lo tanto hay por lo menos un $x$ tal que $a+x=b$. Pero en virtud del teorema 1.1, hay a lo sumo una. Luego hay una y sólo una $x$ en estas condiciones.\\\\ 
\end{teo}

%teorema 3
\begin{teo}
$b-a=b+(-a)$\\\\
Demostración.- \; Sea $x=b-a$ y sea $y=b+(-a)$. Se probará que $x=y$. por definición de $b-a$, $x+a=b$ y $y+a=\left[ b+(-a)\right]+a=b+\left[ (-a)+a \right]=b+0=b$, por lo tanto, $x+a=y+a$ y en virtud de teorema 1.1 $x=y$\\\\
\end{teo}

%teorema 4
\begin{teo}
$-(-a)=a$\\\\
Demostración.- \;
Se tiene $a+(-a)=0$ por definición de $-a$ incluido en el teorema 1.1. Pero esta igualdad dice que $a$ es el opuesto de $-a$, es decir, que si $a+(-a)=0$ entonces $a=0-(-a)=a=-(-a)$\\\\
\end{teo}

%teorema 5
\begin{teo}
$a(b-c)=ab-ac$\\\\
Demostración.- \;
Sea $a(b-c)$ por teorema 1.3 tenemos que $a\left[ b+(-c)\right]$  y por la propiedad distributiva $\left[ ab + a(-c) \right]$, y en virtud de los teorema 1.12 y 1.3 nos queda $ ab - ac $ \\\\
\end{teo}

%teorema 6
\begin{teo}
$0\cdot a = a\cdot 0 =0$\\\\
Demostración.- \;
Sea $0\cdot a$ por la propiedad conmutativa $a\cdot 0$, $a\cdot 0 + 0$ y $a\cdot 0 + \left[a+(-a) \right]$ y en virtud la propiedad asociativa y distributiva $a(0 + 1)+(-a)$ después $1(a)+(-a)$, luego por elemento neutro y existencia de negativos tenemos $0$, Así queda demostrado que cualquier número multiplicado por cero es cero.\\\\ 
\end{teo}

%teorema 7
\begin{teo}[Ley de simplificación para la multiplicación] Si $ab=ac$ y $a \neq 0$, entonces $b=c$. (En particular esto demuestra que el número 1 del axioma 4 es único)\\\\
Demostración.- \;
Sea $b$, $a\neq 0$, y por el existencia del recíproco tenemos $a\cdot a^{'}=1 $  luego,  $b=b\cdot 1=b\left[a(a^{'})\right]=(ab)(a^{'})=(ac)(a^{'})=c(a\cdot a^{'})=c\cdot 1=c$ por lo tanto queda demostrado la ley de simplificación.\\\\
\end{teo}

%teorema 8
\begin{teo}[Posibilidad de la división] Dados $a$ y $b$ con $a\neq 0$, existe uno y sólo un $x$ tal que $ax=b$. La $x$ se designa por $b/a$ ó $\displaystyle\frac{b}{a}$ y se denomina cociente de $b$ y $a$. En particular $1/a$ se escribe también $a^{-1}$ y se designa recíproco de $a$\\\\
Demostración .- \;
Sea $a$ y $b$ por axióma 6 se tiene un $y$ de manera que $a\cdot y = 1$ ó $y = a^{-1}$. Por hipótesis y teorema se tiene $x=b\cdot a^{-1}$, sustituyendo tenemos $x=y\cdot b$ entonces $ax=a(y\cdot b)=(a\cdot y)b=1\cdot b = b$  por lo tanto hay por lo menos un $x$ tal que $ax=b$ pero en virtud del teorema 1.7 hay por lo mucho uno, luego hay una y sólo una $x$ en estas condiciones.\\\\ 
\end{teo}

%teorema 9
\begin{teo}
Si $a\neq 0$, entonces $b/a=b\cdot a^{-1}$\\\\
Demostración.- \;
Sea $x =b/a$ y sea $y=b\cdot a^{-1}$ se probará que $x=y$, por definición de $b/a$, $ax=b$ y $ya=(b\cdot a^{-1})a=b(a^{-1}a)=b\cdot 1 = 1$, entonces $ya=xa$ y por la ley de simplificación para la multiplicación $y=x$ \\\\
\end{teo}

%teorema 10
\begin{teo}
Si $a\neq 0$, entonces $(a^{-1})^{-1}=a$\\\\
Demostración.- \;
Si $a\neq 0$ entonces  $(a^{-1})^{-1} = 1\cdot (a^{-1})^{-1} = \displaystyle\frac{1}{a^{-1}}=a$ esto por axioma de neutro, definición de $a^{-1}$ y teorema 1.9, así concluimos que $(a^{-1})^{-1}=a $ \\\\
\end{teo}

%teorema 1|
\begin{teo}
Si $ab = 0$, entonces ó $a=0$ ó $b=0$\\\\
Demostración.- \;
Veamos dos casos, cuando $x\neq 0$ y cuando $x=0$\\
Si $x\neq 0$ y  $ab = 0$ entonces  $b=b\cdot 1 = b (a\cdot a^{-1}) = (ab)a^{-1}=0a^{-1}=0$, ahora si $a=0$ y virtud del teorema 1.6 nos queda demostrado que la multiplicación de dos números cualesquiera es igual a cero si $a=0$ ó $b=0$  \\\\
\end{teo}

%teorema 12
\begin{teo}
$(-a)b=-(ab) \; y \; (-a)(-b) = ab$\\\\
Demostración.- \;
Empecemos demostrando la primera proposición, Por la ley de simplificación para la suma podemos escribir como $(-a)b+ab=0$ entonces por la propiedad distributiva $b\left[ (-a)+a \right]$ por lo tanto $b\cdot 0$, luego por el teorema 1.6 queda demostrado la primera proposición.\\
Para demostrar la segunda proposición acudimos a la primera proposición, $(-a)(-b)=-\left[ a(-b)\right]$ y luego, \\ $-\left[ a(-b)+b+(-b)\right]=-\left[ (-b)(a+1)+b\right]=-\left[ (-b)(a+1)-1(-b)\right]=-\left[( -b)(a+1-1)\\\right]=-\left[ (-b)a\right]=-\left[ -(ab)\right]$ y en virtud  del el teorema 1.4 $(-a)(-b)=ab$ así queda demostrado la proposición.  \\\\
\end{teo}

%teorema 13
\begin{teo}
$\left(a/b \right) + \left(c/d \right) = \left( ad+bc  \right) / \left( bd \right) $ si $b\neq 0$ y $d\neq 0$.\\\\
Demostración.- \;
Si $\left(a/b \right) + \left(c/d \right)$ entonces por definición de $a/b$, $a\cdot b^{-1}+c\cdot d^{-1}=(a\cdot b^{-1})\cdot 1+(c\cdot d^{-1})\cdot 1=(a\cdot b^{-1})\cdot 1+(c\cdot d^{-1})\cdot 1=(a\cdot b^{-1})\cdot dd^{-1}+(c\cdot d^{-1})\cdot bb^{-1}$ por las propiedades asociativa  conmutativa y distributiva, $(b^{-1}d^{-1})(ad)+(b^{-1}d^{-1})(cb)=(b^{-1}d^{-1})(ad+cb)$, por lo tanto $(ad+bc)/bd$ esto por definición.\\\\
\end{teo}

%teorema 14
\begin{teo}
$(a/b)(c/d)=(ac)/(bd)$ si $b\neq 0$ y $d\neq 0$\\\\
Demostración.- \;
Por definición, $(ab^{-1})(cd^{-1})$, propiedades conmutativa y asociativa $(ac)(b^{-1}d^{-1})$, y por definición queda demostrado la proposición.\\\\
\end{teo}

%corolario 1
\begin{col.}
Si $c\neq 0$ y $d\neq 0$ entonces $(cd^{-1})=c^{-1}d$\\\\
Demostración.- \;
Por definición de $a^{-1}$ tenemos que $(cd^{-1})^{-1}=\displaystyle\frac{1}{cd^{-1}}$, por el teorema de posibilidad de la división $1=(c^{-1}d)(cd^{-1})$ y en virtud de los axiomas de conmutatividad y asociatividad $1=(c^{1}c)(dd^{-1})$, luego $1=1$. quedando demostrado el corolario.\\\\
\end{col.}

%teorema 15 
\begin{teo}
$(a/b)/(c/d)=(ad)/(bc)$ si $b\neq 0$,  $c\neq 0$ y $d\neq 0$\\\\
Demostración.- \;
Sea $(a/b)/(c/d)$ entonces por definición $(ab^{-1})(cd^{-1})^{-1}$, en virtud del corolario 1 se tiene que $(ab^{-1})(c^{-1}d)$, y luego por axioma conmutativa y asociativa $(ad)(c^{-1}b^{-1})$, así por definición concluimos que $(ad)/(cd)$\\\\
\end{teo}

\section{Ejercicios y Demostraciones}
\subsection[Ejercicios]{Ejercicios\footnote{Calculus Vol 1, Tom M. Apostol Pag. 24}}
%ejercicio 1
\begin{teo}
$-0=0$\\\\
Demostración.- \;
Sabemos que por el axioma 5 $a+(-a)=0$, $- \left[ a+(-a) \right] = 0$ y $-a+-(-a)=0$ en virtud de teorema 1.12 y propiedad conmutativa $a+(-a)=0$, por lo tanto $0=0$.\\\\ 
\end{teo}

%ejercicio 2
\begin{teo}
$1^{-1}=1$\\\\
Demostración.- \;
Por la existencia de elementos nuestros tenemos $1^{-1}\cdot 1$ y por axioma de existencia de reciproco $1=1$\\\\
\end{teo}

%ejercicio 3
\begin{teo}
El cero no tiene reciproco\\\\
Demostración.- \;
Supongamos que el cero tiene reciproco es decir $0\cdot 0^{-1}=1$ pero por el teorema 1.6 se tiene que $0\cdot 0^{-1}=0$ y $0=1$ esto no es verdad, por lo tanto el cero no tiene reciproco.\\\\
\end{teo}

%ejercicio 4
\begin{teo}
$-(a+b)=-a-b$\\\\
Demostración.- \;
Por existencia de reciproco $-\left[1(a+b)\right]$ y teorema 1.12 $(-1)(a+b)$ luego por la propiedad distributiva $\left[ (-1)b \right] + \left[ (-1)b \right]$, una vez mas por el teorema 1.12 $-(1a)+ \left[-(1b)\right]$, en virtud del axioma 4 $-a+(-b)$, y teorema 1.3, $-a-b$ \\\\ 
\end{teo}

%ejercicio 5
\begin{teo}
$-(-a-b)=a+b$\\\\
Demostración\\\\
Si $-(-a-b)$ entonces por axioma $-\left[1(-a-b)\right]$, luego $(-1)(-a-b)=(-1)(-a)-\left[(-1)b\right])= (1\cdot a)-\left[-(1\cdot b)\right]$ y por axioma $a - \left[ - (b)\right]$, así por teorema $a+b$.\\\\
\end{teo}

%ejericio 6
\begin{teo}
$(a-b)+(b-c)=a-c$\\\\
Demostración.- \;
Por definición tenemos $\left[ a+(-b) \right]+\left[ b+(-c) \right]$, y axiomas de asociatividad y conmutatividad $\left[ a+(-c) \right]+\left[ b+(-b) \right]$, luego por existencia de negativos  $\left[ a+(-c) \right] + 0$,, así $a+(-c)$ y $a-c$. \\\\
\end{teo}

%ejercico 7
\begin{teo}
$-(a/b)=(-a/b)=a/(-b)$ si $b\neq 0$\\\\
Demostración.- \;
Primero demostremos que $-(a/b)=(-a/b)$, Sea $b\neq 0$, en virtud de definición de la división y teorema 1.12 no queda que $-(a/b)=(-a)\cdot b^{-1}=-a/b$.\\
Ahora demostramos que $-(a/b)=(-a/b)=a/(-b)$, sea $b\neq0$, luego $-(b^{-1}\cdot a)=\left[-(b^-1)\right]\cdot a = a/-b$. \\\\
\end{teo}

%ejercicio 8
\begin{teo}
$(a/b)-(c/d)=(ad-bc)/(bd)$ si $b\neq 0$ y $d\neq 0$\\\\
Demostración.- \;
Sea $b\neq 0$ y $d\neq 0$ y por definición $ab^{-1}-cd^{-1}$, luego por axiomas $(ab^{-1})(d\cdot d^{-1})-(cd^{-1})(b\cdot b^{-1})$, y en virtud del teorema 1.5 y propiedad asociativa $b^{-1}\cdot d^{-1}(ad-bc)$ y $(ad-bc)/bd$\\\\
\end{teo}




\subsection[Demostraciones]{Demostraciones\footnote{Cálculo infinitesimal, Michael Spivak pag 17 al 26}}
%teorema 1.24
\begin{teo}
Si $ax=a$ para algún número $a\neq 0$, entonces $x=1$\\\\
Demostración.- \;
Sea $a\neq 0$ entonces por la ley de simplificación $x=1$\\\\
\end{teo}

%teorema 1.25
\begin{teo}
$(x^2-y^2)=(x-y)(x+y)$\\\\
Demostración.- \;
Partamos de $(x-y)(x+y)$ donde por la propiedad distributiva tenemos $(x-y)x+(x-y)y$, luego $x^2-xy+xy-y^2$, por lo tanto por los axioma  de inverso y  neutro $x^2-y^2$\\\\ 
\end{teo}

%Corolario 2
\begin{col.}
Si $a=b$ entonces $ac=bc$\\\\
Demostración.- \;
vemos por la propiedad reflexiva de los números que $ac=ac$ y por hipótesis no queda que $ac=bc$\\\\
\end{col.}

%teorema 1.26
\begin{teo}
Si $x^2=y^2$, entonces $x=y$ o $x=-y$\\\\
Demostración.- \;
Dada la hipótesis aplicamos el corolario anterior $x^2+\left[ - (y^2) \right]=y+\left[ - (y^2) \right]$ y por axioma de neutro y definición $x^2-y^2=0$, por ejercicio 1.10 $(x-y)(x+y)$ donde en virtud del teorema $ab=0$ entonces $a=0$ o $b=0$ no queda $x-y=0$ ó $x+y=0$, por lo tanto $x=y$ ó $x=-y$ \\\\ 
\end{teo}

%teorema 1.27
\begin{teo}
$(x^3-y^3)=(x-y)(x^2+xy+y^2)$\\\\
Demostración.- \;
Dado $(x-y)(x^2+xy+y^2)$ y por la propiedad distributiva $(x-y)x^2+(x-y)xy+(x-y)y^2 = x^3 -x^2y +x^2y-xy^2+xy^2-y^3$ por lo tanto en virtud de los axiomas de inverso y neutro $x^3-y^3$.\\\\
\end{teo}

%teorema 1.28
\begin{teo}
$x^n-y^n=(x+y)(x^{n-1}+x^{n-2} y + ... + xy^{n-2}+y^{n-1})$\\\\
Demostración.- \; 
\begin{center}
\begin{tabular}{r c l}
&&$(x-y)(x^{n-1}+x^{n-2}y+...+xy^{n-2}+y^{n-1})$\\
&=&$x(x^{n-1}+x^{n-2}y+...+xy^{n-2}+y^{n-1})-y(x^{n-1}+x^{n-2}y+...+xy^{n-2}+y^{n-1}$\\
&=&$x^n+x^{n-1}y+...+x^2y^{n-2}+xy^{n-1}-(x^{n-1}y+x^{n-2}y^2+...+xy^{n-1}+y^n)$\\
&=&$x^n-y^n$\\\\	
\end{tabular}
\end{center}
\end{teo}

%teorema 1.29
\begin{teo}
$x^3+y^3=(x+y)(x^2-xy+y^2)$\\\\
Demostración.- \;
La demostración es similar al teorema 1.12.\\\\
\end{teo}

%teorema 1.30
\begin{teo}
$\displaystyle\frac{a}{b}=\frac{ac}{bc}$ si $b$, $c \neq 0$\\\\
Demostración.- \;
Por definición tenemos que $\displaystyle\frac{a}{b}=ab^{-1}$, como $b, \; c \neq 0$ entonces $(ab)(c\cdot c^{-1})$, por  las propiedades asociativa y conmutativa, $(ac)(b^{-1}c^{-1})$, y en virtud del corolario 1  $(ac)(bc)^{-1}$ y $\displaystyle\frac{ac}{bc}$ \\\\ 
\end{teo}

%teorema 1.31
\begin{teo}
Si $b,\; c \neq 0$, entonces $\displaystyle\frac{a}{b}=\frac{c}{d}$ si sólo si $ad=bc$, Determinar también cuando es $\displaystyle\frac{a}{b}=\frac{b}{a}$\\\\
Demostración.- \;
Sea $b,\; c \neq 0$ si sólo si $ab^{-1}=cd^{-1}$ por corolario 2 $(ab^{-1})b=cd^{-1}b$ y por propiedades asociativa y conmutativa $a(b\cdot b^{-1})=(bc)d^{-1}$, $a=(bc)d^{-1}$ luego $ad=bc(d\cdot d^{-1})$, por lo tanto $ad=bc$. Análogamente $\displaystyle\frac{a}{b}=\frac{b}{a}\Leftrightarrow a^2=b^2$\\\\
 
\end{teo} 