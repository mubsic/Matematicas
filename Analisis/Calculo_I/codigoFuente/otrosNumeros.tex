\chapter{Distintas clases de números}
\section{El principio de la inducción matemática}
%teorema 
\begin{teo}[Principio de inducción matemática]
Sea $S$ un conjunto de enteros positivos que tienen las dos propiedades siguientes:
\begin{enumerate}[\bfseries a)]
\item El número 1 pertenece al conjunto $S$.
\item Si un entero $k$ pertenece al conjunto $S$, también $k+1$ pertenece a $S$.
\end{enumerate}
Entonces todo entero positivo pertenece al conjunto $S$.\\\\
Demostración.- \; Las propiedades $a)$ y $b)$ nos dicen que $S$ es un conjunto inductivo. Por consiguiente $S$ tiene cualquier entero positivo.\\\\ 
\end{teo}

\section{Teoremas y Ejercicios}
\subsection[Ejercicios]{Ejercicios\footnote{Tom Apostol Vol 1, pag 44, pag 49}}
%ejercicio 5.1
\begin{ej}
Demostrar por inducción las fórmulas siguientes:
\begin{enumerate}[\bfseries a)]
\item $1+2+3+...+n=n(b+1)/2$\\\\
Demostración.- \; Sea $n=k$ entonces $1+2+3+...+k=k(k+1)/2$.\\
Para $k=1$ se tiene $1=1(1+1)/2$. \\ 
Por ultimo si  $k=k+1$ nos queda probar que  $1+2+3+...+k+(k+1)=\dfrac{(k+1)(k+2)}{2}$, luego $\dfrac{k(k+1)}{2}+(k+1)=\dfrac{k(k+1)+2(k+1)}{2}$. Así $\dfrac{k^2+3k+2}{2}$ \; y \; $\dfrac{(x+1)(x+2)}{2}$\\\\

\item $1+3+5+...+(2n-1)=n^2$\\\\
Demostración.- \; Sea $n=k$ entonces $1+3+5+...+(2k-1)=k^2$.\\
Para $k=1$ se tiene $[2(1)-1]=1^2$, así $1=1$\\
Luego, si $k=k+1$ entonces $1+3+5+...+[2(k+1)-1]=(k+1)^2$. Por lo tanto $k^2+2k+1=(k+1)^2$.\\\\

\item $1^3+2^3+3^3+...+n^3=(1+2+3+...+n)^2$\\\\
Demostración.- \; Sea $n=k$ entonces, $$1^3+2^3+3^3+...+k^3=(1+2+3+...+k)^2$$
Para $k=1$ $$1=1,$$
Luego $k=k+1$, $$1^3+2^3+3^3+...+k^3+(k+1)^3=(1+2+3+...+k)^2,$$
Así,
\begin{center}
\begin{tabular}{r c l}
$\left(\dfrac{k(k+1)}{2} \right)^2+(k+1)^3=$&=&$\left( \dfrac{k(k+1)}{2}+(k+1)\right) ^2$\\\\
$\dfrac{k^2(k+1)^2}{4}+(k+1)^3$&=&$\dfrac{k^2(k+1)^2}{4}+k(k+1)^2+(k+1)^2$\\\\
$\dfrac{k^2(k+1)^2+4(k+1)^3}{4}$&=&$\dfrac{k^2(k+1)^2+4k(k+1)^2+4(k+1)^2}{4}$\\\\
$\dfrac{(k+1)^2 (k^2+4k+4)}{4}$&=&$\dfrac{(k+1)^2 (k^2 + 4k +4)}{4}$\\\\
\end{tabular}   
\end{center}

\item $1^3+2^3+...+(n-1)^3<n^4/4<1^3+2^3+...+n^3$\\\\
Demostración.- \; Sea $n=k$ entonces $$1^3+2^3+...+(k-1)^3<k^4/4 \; \; \; (1)$$ 
Para $k=1$, \; $0<1/4$ se observa que se cumple.\\
Después, para $k=k+1$, $$1^3+2^3+...+k^3<(k+1)^4/4,$$ sumando $k^3$ a $(1)$, 
$$1^3+2^3+...+k^3<k^4/4+k^3$$ y para deducir como consecuencia de $k+1$, basta demostrar, $$k^4/4+k^3<(k+1)^4/4$$, Pero esto es consecuencia inmediata de la igualdad $$(k+1)^4/4=(k^4+4k^3+6k^2+4k+1)/4=k^4/4+k^3+(3k^2)/2+k+1$$ Por tanto se demostró que $k+1$ es consecuencia de $k$\\\\
\end{enumerate}
\end{ej}

\begin{ej}
Obsérvese que: 
\begin{center}
\begin{tabular}{r c l}
$1$&$=$&$1$\\
$1-4$&$=$&$-(1+2)$\\
$1-4+9$&$=$&$1+2+3$\\
$1-4+9-16$&$=$&$-(1+2+3+4)$\\
\end{tabular}
\end{center}
Indúzcase la ley general y demuéstrese por inducción\\\\
Demostración.- \: Verificando tenemos que la ley general es $1-4+9-16+...+(-1)^{n+1}\cdot n^2=(-1)^{n+1}(1+2+3+...+n)$. \\
Ahora pasemos a demostrarlo. Sea $n=k$ entonces, $$1-4+9-16+...+(-1)^{k+1}\cdot k^2=(-1)^{k+1}(1+2+3+...+k)$$ Si $k=1$, se sigue, $(-1)^2 \cdot 1^2 = (-1)^2\cdot 1$, vemos que satisface para $k=1.$
Luego $k=k+1$,  $$1-4+9-16+...+(-1)^{k+2}\cdot (k+1)^2=(-1)^{k+2}\left[\dfrac{(k+1)(k+2)}{2}\right]$$.\\
Sumando $(-1)^{k+2}\cdot (k+1)^2$  a la segunda igualdad dada, se tiene,
$$1-4+9-16+...+ (-1)^{k+2}\cdot (k+1)^2 = (-1)^{k+1}\left(\dfrac{k(k+1)}{2}\right) + (-1)^{k+2}\cdot (k+1)^2$$ Por lo tanto, basta demostrar que $(-1)^{k+1}\left(\dfrac{k(k+1)}{2}\right) + (-1)^{k+2}\cdot (k+1)^2=(-1)^{k+2}\left[\dfrac{(k+1)(k+2)}{2}\right]$\\
\begin{center}
\begin{tabular}{r c l}
$(-1)^{k+1}\left(\dfrac{k(k+1)}{2}\right) + (-1)^{k+2}\cdot (k+1)^2$&$=$&$(-1)^{k+2}\left\lbrace \dfrac{[(-1)(k+1)k]+2(k^2+2k+1)}{2} \right\rbrace$\\\\
&$=$&$(-1)^{k+2} \left( \dfrac{-k^2 -k +2k^2 +4k +2}{2} \right)$\\\\
&$=$&$(-1)^{k+2} \left( \dfrac{k^2+3k+2}{2} \right)$\\\\
&$=$&$(-1)^{k+2} \left[ \dfrac{(x+1)(x+2)}{2} \right]$\\\\
\end{tabular}
\end{center}
\end{ej}

%ejercicio 5.3
\begin{ej}
Obsérvese que 
\begin{center}
\begin{tabular}{r c l}
$1+\frac{1}{2}$&=&$2-\frac{1}{2}$\\\\
$1+\frac{1}{2}+\frac{1}{4}$&=&$2-\frac{1}{4}$\\\\
$1+\frac{1}{2}+\frac{1}{4}+\frac{1}{8}$&=&$2-\frac{1}{8}$\\\\
\end{tabular}
\end{center}
Demostración.- \; Se verifica que $1+\frac{1}{2}+\frac{1}{4}+...+\dfrac{1}{2^n}=2-\dfrac{1}{2^n}$.\\\\
Para $n=k$ $$1+\frac{1}{2}+\frac{1}{4}+...+\dfrac{1}{2^k}=2-\dfrac{1}{2^k}$$\\
$k=1$ $$1+\dfrac{1}{2^1}=2-\dfrac{1}{2^1}$$\\
Luego $k=k+1$ $$1+\frac{1}{2}+\frac{1}{4}+...+\dfrac{1}{2^{k+1}}=2-\dfrac{1}{2^{k+1}}$$\\
Así solo falta demostrar que, $$2-\dfrac{1}{2^k}+\dfrac{1}{2^{k+1}}=2-\dfrac{1}{2^{k+1}}$$
\begin{center}
\begin{tabular}{r c l}
$2-\dfrac{1}{2^k}+\dfrac{1}{2^{k+1}}$&=&$2+\dfrac{-2+1}{2^{k+1}}$\\\\
&=&$2-\dfrac{1}{2^{k+1}}$\\\\
\end{tabular}
\end{center}
\end{ej}

%ejercicio 5.4
\begin{ej}
Obsérvese que
\begin{center}
\begin{tabular}{r c l}
$1-\frac{1}{2}$&=&$\frac{1}{2}$\\\\
$(1-\frac{1}{2})(1-\frac{1}{3})$&=&$\frac{1}{3}$\\\\
$(1-\frac{1}{2})(1-\frac{1}{3})(1-\frac{1}{4})$&=&$\frac{1}{4}$\\\\
\end{tabular}
\end{center}
Indúzcase la ley general y demuéstrese por inducción.\\\\
Demostración.- \; Se induce que $\left( 1-\dfrac{1}{2} \right)\left( 1-\dfrac{1}{3} \right)...\left(1- \dfrac{1}{n} \right)=\dfrac{1}{n}$ para todo $n>1$.\\
Sea $n=k$, entonces $\left( 1-\dfrac{1}{2} \right)\left( 1-\dfrac{1}{3} \right)...\left(1- \dfrac{1}{k} \right)=\dfrac{1}{k}$. Después para $k=2$, \; $1-\dfrac{1}{2}=\dfrac{1}{2}$. Si $k=k+1$ tenemos $\left( 1-\dfrac{1}{2} \right) \left( 1-\dfrac{1}{3} \right)...\left(1- \dfrac{1}{k+1} \right)=\dfrac{1}{k+1}$. Luego es fácil comprobar que 
$\dfrac{1}{k}\left(1 - \dfrac{1}{k+1} \right)=\dfrac{1}{k+1}$. \\\\
\end{ej}


%ejercicio 5.5.
\begin{ej}
Hallar la ley general que simplifica al producto $$\left( 1-\dfrac{1}{4} \right)\left( 1-\dfrac{1}{9} \right)\left( 1- \dfrac{1}{16} \right)..\left( 1- \dfrac{1}{n^2} \right)$$ y demuéstrese por inducción.\\\\
Demostración.- \; Inducimos que $\left( 1-\dfrac{1}{4} \right)\left( 1-\dfrac{1}{9} \right)\left( 1- \dfrac{1}{16} \right)..\left( 1- \dfrac{1}{n^2} \right)=\dfrac{n+1}{2n}$,para todo $n>1$. Después $n=k=1$, $$1-\dfrac{1}{2^2}=\dfrac{2+1}{2\dot 2} \Rightarrow \dfrac{3}{4}=\dfrac{3}{4}$$ 
Luego $k+1$, $$\left( 1-\dfrac{1}{4} \right)\left( 1-\dfrac{1}{9} \right)\left( 1- \dfrac{1}{16} \right)..\left( 1- \dfrac{1}{(k+1)^2} \right)=\dfrac{(k+1)+1}{2(k+1)}$$
Así,
\begin{center}
\begin{tabular}{r c l}
$\left( \dfrac{k+1}{2k}\right) \left( 1- \dfrac{1}{(k+1)^2} \right)$&=&$\dfrac{(k+1)+1}{2(k+1)}$\\\\
$\left( \dfrac{k+1}{2k} - \dfrac{1}{2k(k+1)} \right)$&=&$\dfrac{k+2}{2k+2}$\\\\
$\dfrac{(k+1)^2-1}{2k(k+1)}$&=&$\dfrac{k+2}{2k+2}$\\\\
$\dfrac{k+2}{2k+2}$&=&$\dfrac{k+2}{2k+2}$\\\\
\end{tabular}
\end{center}
\end{ej}

%ejercicio I 4.7 problema 1 pag 49
\begin{ej}
Hallar los valores numéricos de las sumas siguientes:
\begin{enumerate}[\bfseries a)]
\item $\displaystyle\sum_{k=1}^{4} k = 1 + 2 + 3 + 4 = 10$\\\\

\item $\displaystyle\sum_{n=2}^{5} 2^{n-2} = 2^{0} + 2^{1} + 2^{2} + 2^{3} = 15$\\\\

\item $\displaystyle\sum_{r=0}^{3} 2^{2r+1} = 2^{1} + 2^{3} +  2^{5} + 2^{7} = 2 + 8 + 32 + 128 = 170$ \\\\

\item $\displaystyle\sum_{n=1}^{4} n^n = 1^1 + 2^2 + 3^3 + 4^4 = 1 + 4 + 27 + 256= 288$ \\\\

\item $\displaystyle\sum_{i=0}^{5} (2i + 1) = 1 + 3 + 5 + 7 + 9 + 11 = 36$ \\\\

\item $\displaystyle\sum_{k=1}^{5} \dfrac{1}{k(k+1)} = \dfrac{1}{2} + \dfrac{1}{6} + \dfrac{1}{12} + \dfrac{1}{20} + \dfrac{1}{30} = 0,83333.....$ \\\\
\end{enumerate}
\end{ej}






\subsection[Teoremas]{Teoremas\footnote{Calculus, Tom Apostol, pag. 45-46, 49-50}}

%teorema 5.2
\begin{teo}[principio de buena ordenación]
Todo conjunto no vacío de enteros positivos contiene uno que es el menor \\\\
Demostración.- \; Sea $T$ una colección no vacía de enteros positivos. Queremos demostrar que $t_0$ tiene un número que es el menor, esto es, que hay en T un entero positivo t.; tal que $t_0\leq t$ para todo $t$ de $T$.\\
Supongamos que no fuera así. Demostraremos que esto nos conduce a una contradicción. El entero $1$ no puede pertenecer a $T$ (de otro modo él sería el menor número de $T$). Designemos con $S$ la colección de todos los enteros positivos $n$ tales que $n<t$ para todo $t$ de $T$. Por tanto $1$ pertenece a $S$ porque $1 < t$ para todo $t$ de $T$. Seguidamente, sea $k$ un entero positivo de $S$. Entonces $k < t$ para todo $t$ de $T$. Demostraremos que $k + 1$ también es de $S$. Si no fuera así, entonces para un cierto $t$, de $T$ tendríamos $t_1 \leq k+1$. Puesto que $T$ no posee número mínimo, hay un entero $t_2$ en $T$ tal que $t_2 < t_1$ Y por tanto $t_2 < k + 1$. Pero esto significa que $t_2 \leq k$, en contradicción con el hecho de que $k < t$ para todo $t$ de $T$. Por tanto $k + 1$ pertenece a $S$. Según el principio de inducción, $S$ contiene todos los enteros positivos. Puesto que $T$ es no vacío, existe un entero positivo $t$ en $T$. Pero este $t$ debe ser también de $S$ (ya que $S$ contiene todos los enteros positivos). De la definición de $S$ resulta que $t < t$, lo cual es absurdo. Por consiguiente, la hipótesis de que $T$ no posee un número mínimo nos lleva a una contradicción. Resulta pues que $T$ debe tener un número mínimo, y a su vez esto prueba que el principio de buena ordenación es una consecuencia del de inducción.\\\\
\end{teo}

%ejercicio I 4.4 PROBLEMA 6 pag 45
\begin{teo}
Sea $A(n)$ la proporción: $1+2+...+n=\dfrac{1}{8} (2n+1)^2$.
\begin{enumerate}[\bfseries a)]
\item Probar que si $A(k)$, $A(k+1)$ también es cierta.\\\\
Demostración.- \; Para $A(k+1)$, 
\begin{center}
\begin{tabular}{r c l}
$\dfrac{1}{8}(2k+1)^2+(k+1)$&=&$\dfrac{1}{8}\left[2(k+1)+1\right]^2$\\\\
$\dfrac{4k^2+12k+9}{8}$&=&$\dfrac{4k^2+12k+9}{8}$\\\\
\end{tabular}
\end{center}
\item Critíquese la proposición $"$de la inducción se sigue que $A(n)$ es cierta para todo $n$ $"$.\\\\
Se ve que no se cumple para ningún entero $A(n)$ pero si para $A(n+1)$.\\\\ 
\item Transfórmese $A(n)$ cambiando la igualdad por una desigualdad que es cierta para todo entero positivo $n$\\\\
Primero comprobemos para $A(1)$, \; $1<\dfrac{9}{8}$.\\
Luego para $A(k),$ $$1+2+...+(k+1)<\dfrac{1}{8}(2k+1)^2$$
Después para $A(k+1)$ $$1+2+...+(k+1)<\dfrac{1}{8}(2k+1)^2$$
Remplazando $(k+1)$ a $A(k)$ $$1+2+...+(k+1)<\dfrac{1}{8}(2k+1)^2+(k+1)$$
por último solo nos queda demostrar $$\dfrac{1}{8}(2k+1)^2<\dfrac{1}{8}(2k+1)^2+(k+1)$$
Así $\dfrac{4k^2+12k+9}{8}<\dfrac{4k^2+12k+9}{8} +(k+1)$, vemos que la inecuación se cumple para cualquier número natural.\\\\ 
\end{enumerate} 
\end{teo}

%ejercicio I 4.4 problema 7 pag 45
\begin{teo}
Sea $n_1$ el menor entero positivo $n$ para el que la desigualdad $(1+x)^n>1+nx+nx^2$ es cierta para todo $x>0$. Calcular $n_1$, y demostrar que la desigualdad es cierta para todos los enteros $n\geq n_1$\\\\
Demostración.- \; vemos que la proposición es validad para $n_1=3$, $$(1+x)^3>1+3x+3x^2,$$ y no así para $n=1$ \; y \;$n=2$ entonces $A(n)=A(k)\geq 3$, $(1+x)^k>1+kx+kx^2.$ Después para un $A(k+1)$, $(1+x)^{k+1}>1+(k+1)x+(k+1)x^2,$ así \; $(1+kx+kx^2)(1+k)>1+(k+1)x+(k+1)x^2,$ luego se cumple la desigualdad $x(kx^2)+x^2+kx+x+1+x^2>kx^2+x^2+kx+x+1$.\\\\
\end{teo}

%ejercicio I 4.4 problema 8 pag 45
\begin{teo}
Dados números reales positivos $a_1,a_2,a_3,...,$ tales que $a_n\leq ca_{a-1}$ para todo $n\geq 2$. Donde $c$ es un número positivo fijo, aplíquese el método de inducción para demostrar que $a_n \leq a_1 c^{n-1}$ para cada $n \geq 1$\\\\
Demostración.- \; Primero, para el caso $n=1$, tenemos $a_1c^0=a_1$, por lo tanto la desigualdad es validad. Ahora supongamos que la desigualdad es válida para algún número entero $k$: $a_k\leq a_1c^{k-1}$, luego multiplicamos por $c$, $ca_k\leq a_1c^k$, pero dado que se asume por hipótesis $a_{k+1} \leq ca_k$, entonces $a_{k+1}\leq a_1c^k$, por lo tanto, la declaración es válida para todo $n$.\\\\
\end{teo}

%ejercicio I 4.4 problema 9 pag 45
\begin{ej}
Demuéstrese por inducción la proposición siguiente: Dado un segmento de longitud unidad, el segmento de longitud $\sqrt{n}$ se puede construir con regla y compás para cada entero positivo $n$.\\\\
Demostración.- \; Dada una línea de longitud 1, podemos construir una línea de longitud $\sqrt{2}$ tomando la hipotenusa del triángulo rectángulo con patas de longitud 1.\\
Ahora, supongamos que tenemos una línea de longitud 1 y una línea de longitud $\sqrt{k}$ para algún número entero $k$. Luego podemos formar un triángulo rectángulo con patas de longitud 1 y longitud $\sqrt{k}$. La hipotenusa de este triángulo es $\sqrt{k+1}$. Por lo tanto, si podemos construir una línea de longitud $\sqrt{k}$, entonces podemos construir una línea de longitud $\sqrt{k+1}$. Como podemos construir una línea de longitud $\sqrt{2}$ en el caso base, podemos construir una línea de longitud $\sqrt{n}$ para todos los enteros $n$.\\\\
\end{ej}

%ejercicio I 4.4. problema 10 pag 45
\begin{teo}
Sea $b$ un entero positivo. Demostrar por inducción la proposición siguiente: Para cada entero $n\geq 0$ existen enteros no negativos $q$ \; y \; $r$ tales que:
$$n=qb+r, \; \; \; 0\leq r < b$$ \\
Demostración.- \; Sea $b$ ser un entero positivo fijo. Si $n=0$ , luego $q=r=0$, la afirmación es verdadera (ya que $0=0b+0$).\\
Ahora suponga que la afirmación es cierta para algunos $k \in \mathbb{N}$. Por hipótesis de inducción sabemos que existen enteros no negativos $q$ \; y \; $r$ tales que $$k=qb+r, \; \; \; 0\leq r < b,$$ Por lo tanto, sumando 1 a ambos lados tenemos, $$k+1=qb+(r+1).$$ Pues $0\leq r < b$ entonces sabemos que $0\leq r \leq (b-1)$. Si $0\leq r < b-1,$ entonces $0\leq r+1<b,$ y la declaración aún se mantiene con la misma elección $q$ \; y \; $r+1$ en lugar de $r$.\\
Por otro lado, si $r=b-1,$ entonces $r+1=b$ y tenemos, $$k+1=qb+b=(q+1)b+0$$
Por lo tanto, la declaración se mantiene de nuevo, pero con $q+1$ en lugar de $q$ \; y con $r=0$ (que es válido ya que si $r=0$ tenemos $a\leq r < b$). Por ende, si el algoritmo de división es válido para $k$, entonces también es válido para $k+1.$ Entonces, es válido para todos $n \in \mathbb{N}$.\\\\
\end{teo}

%ejercicio I 4.4 problema 11 pag 45
\begin{teo}
Sea $n$ \; y \; $d$ enteros. Se dice que $d$ es un divisor de $n$ si $n=cd$ para algún entero $c$. Un entero $n$ se denomina primo si $n>1$ y los únicos divisores de $n$ son $1$ \; y \; $n$. Demostrar por inducción que cada entero $n>1$ es o primo o producto de primos.\\\\
Demostración.- \; LA prueba se hará por inducción. Si $n=2$ ó $n=3$ entonces $n$ es primo, entonces la proposición es verdadera.\\
Ahora supongamos que la afirmación es verdadera para todos los enteros desde $2$ hasta $k$. Se debe demostrar que esto implica $k+1$ es primo o un producto de primos. Si $k+1$ es primo, entonces no hay nada que demostrar. Por otro lado, si $k+1$ no es primo, entonces sabemos que hay enteros $c$ \; y \; $d$ tal que $1<c,$ $d<k+1$ en otras palabras decimos que $n$ es divisible por números disntintos de $1$ y de sí mismo.\\
Por hipótesis de inducción, sabemos que $2\leq c$, $d\leq k$ entonces $c$ \; y \; $d$ son primos o son producto de primos.\\
Por lo tanto, si la declaración es verdadera para todos los enteros mayores que $1$ hasta $k$ entonces, también es verdadera para $k+1$, así es cierto para todo $n\in \mathbb{Z}^+$\\\\
\end{teo}

%ejercicio I 4.4  problema 12 pag 45 
\begin{teo}
Explíquese el error en la siguiente demostración por inducción.\\
Proposición.- Dado un conjunto de n niñas rubias, si por 10 menos una de las niñas tiene ojos azules, entonces las n niñas tienen ojos azules.\\
Demostración.-\; La proposición es evidentemente cierta si $n = 1$. El paso de $k$ a $k + 1$ se puede ilustrar pasando de $n = 3$ a $n = 4$. Supóngase para ello que la proposición es cierta para $n=3$ Y sean $G_1, G_2, G_3, G_4$ cuatro niñas rubias tales que una de ellas, por lo menos, tenga ojos azules, por ejemplo, la $G_1$, Tomando $G_1,G_2, G_3,$ conjuntamente y haciendo uso de la proposición cierta para $n =3$, resulta que también $G_2$ y $G_3$ tienen ojos azules. Repitiendo el proceso con $G_1, G_2$ y $G_4,$ se encuentra igualmente que $G_4$ tiene ojos azules. Es decir, las cuatro tienen ojos azules. Un razonamiento análogo permite el paso de $k$ a $k + 1$ en general.\\
\textbf{Corolario.} Todas las niñas rubias tienen ojos azules.\\
Demostración.- \; Puesto que efectivamente existe una niña rubia con ojos azules, se puede aplicar el resultado precedente al conjunto formado por todas las niñas rubias.\\\\
Esta prueba supone que la afirmación es cierta n = 3, es decir, supone que si hay tres chicas rubias, una de las cuales tiene ojos azules, entonces todas tienen ojos azules. Claramente, esta es una suposición falsa.\\\\
\end{teo}

%ejercicio 1 4.7 problema 2 pag 49
\begin{teo}Establecer las siguientes propiedades del símbolo sumatorio. 
\begin{enumerate}[\bfseries a)]
\item $\displaystyle\sum_{k=1}^{n}(a_k + b_k) = \sum_{k=1} a_k + \sum_{k=1}^{n} b_k$ (propiedad aditiva)
\begin{center}
\begin{tabular}{r c l l}
$\displaystyle\sum_{k=1}^{n}(a_k + b_k)$&$=$&$(a_1 + b_1) + (a_2 + b_2) + ... + (a_n + b_n)$&\\\\
&$=$&$(a_1 + a_2 + ... + a_n) + (b_1 + b_2 + ... + b_n)$&Asociatividad y conmutatividad\\\\
&$=$&$\displaystyle\sum_{k=1}^{n} a_k + \sum_{k=1}^{n} b_k$&\\\\
\end{tabular}
\end{center}

\item $\displaystyle\sum_{k=1}^{n} (ca_k) = c \sum_{k=1}^{n} a_k$ (Prpiedad homogénea)
\begin{center}
\begin{tabular}{r c l l}
$\displaystyle\sum_{k=1}^{n} (ca_k)$&$=$&$ca_1 + ca_2 + ... + ca_n$&\\\\
&$=$&$c(a_1 + a_2 +...+a_n)$&distributividad\\\\
&$=$&$ c \displaystyle\sum_{k=1}^{n} a$&\\\\
\end{tabular}
\end{center}

\item $\displaystyle\sum_{k=1}^{n} (a_k - a_{k-1}) = a_n - a_0$ (Propiedad telescópica)
\begin{center}
\begin{tabular}{r c l l}
$\displaystyle\sum_{k=1}^{n} (a_k - a_{k-1})$&$=$&$a_n + \displaystyle\sum_{k=1}^{n-1} a_k - \sum_{k=0}^{n-1} a_k$&Reindexando la segunda suma\\\\
&$=$&$a_n + \displaystyle\sum_{k=1}^{n-1} a_k - a_0 - \sum_{k=1}^{n-1} a_k$&\\\\
&$=$&$a_n - a_0$&\\\\
\end{tabular}
\end{center}
\end{enumerate}
\end{teo}

% ejercicio I 4.7 problema 3 pag 49
\begin{teo}
$\displaystyle\sum_{k=1}^{n} 1 = n$ (El sentido de esta suma es $\displaystyle\sum_{k=1}^{n} a_k$, cuando $a_k=1$)\\\\
Demostración.- \; Probemos por inducción, Si $n=1$, entonces la proposición es verdadera ya que $\displaystyle\sum_{k=1}^{1} 1 = 1.$ Ahora supongamos que el enunciado es verdadero para $n=m \in \mathbb{Z}^+.$ Luego, $$\displaystyle\sum_{k=1}^m 1 = m \Rightarrow \left( \sum_{k=1}^m 1 = m \right) + 1 = m + 1$$\\\\
\end{teo}

% ejercicio I 4.7 problema 4 pag 49
\begin{teo}
$\displaystyle\sum_{k=1}^{n} (2k - 1) = n^2$ $[Indicación, \; 2k-1 = k^2 - (k-1)^2]$\\\\
Demostración.- \; Sea $2k-1 = k^2 - (k^2 - 2k + 1) = k^2 - (k-1)^2$ entonces  por la propiedad telescópica se tiene $\displaystyle\sum_{k=1}^n = \sum_{k=1}^{n} (k^2 - (k-1)^2) = n^2$\\\\
\end{teo}

% ejercicio I 4.7 problema 5 pag 49
\begin{teo}
$\displaystyle\sum_{k=1}^{n} k = \dfrac{n^2}{2} + \dfrac{n}{2} $ \; \; [indicación. \; Úsese el ejercicio 3 y el 4.\\\\
Demostración.- \;Por aditividad y homogeneidad se tiene $\displaystyle\sum_{k=1}^{n}(2k-1) = 2\displaystyle\sum_{k=1}^{n} k - \sum_{k=1}^{n} 1 = n^2$ entonces $\displaystyle\sum_{k=1}^{n} k - n =n^2$ ya que $\displaystyle\sum_{k=1}^{n} 1 = n$, luego $\displaystyle\sum_{k=1}^{n} k = \dfrac{n^2}{2} + \dfrac{n}{2}$\\\\
\end{teo}

% ejercicio I 4.7 problema 6 pag 49
\begin{teo}
$\\displaystyle\sum_{k=1}^{n} k^2 = \dfrac{n^4}{4} + \dfrac{n^3}{2} + \dfrac{n^2}{4}$ $[Indicación. \; k^3 - (k-1)^3 = 3k^2 - 2k + 1]$\\\\
Demostración.- \; 
\end{teo}























\subsection[Teoremas]{Teoremas \footnote{Calculus, Tom Apostol, pag. 54-58}}

% 4.10 problema 1 pag 54
\begin{ej}
Calcúlese los valores de los siguientes coeficientes binomiales:
\begin{enumerate}[\bfseries a)]
\item ${5 \choose 3}$
$${5 \choose 3} = \dfrac{5!}{3! (5-3)!} = \dfrac{5!}{3! 2!} = \dfrac{20}{2} = 10$$ \\

\item ${7 \choose 0}$
$${7 \choose 0} = \dfrac{7!}{0!(7-0)!} = \dfrac{7!}{7!} = 1$$ \\

\item ${7 \choose 1}$
$${7 \choose 1} = \dfrac{7!}{1!(7-1)!} = \dfrac{7!}{6!} = 7$$ \\

\item ${7 \choose 2}$
$${7 \choose 2} = \dfrac{7!}{2!(7-2)!} = \dfrac{7!}{2!5!} = \dfrac{42}{2} = 21$$ \\

\item ${17 \choose 14}$
$${17 \choose 14} = \dfrac{17!}{14!(17-14)!} = \dfrac{17!}{14!3!} = \dfrac{17 \cdot 16 \cdot 15}{3!} = \\dfrac{4080}{6} = 680$$\\

\item ${0\choose 0}$
$${0\choose 0}=\dfrac{0!}{0!(0-0)!}=\dfrac{0!}{0!0!} = 1$$\\\\
\end{enumerate}
\end{ej}
	
% 4.10 problema 2 pag 54
\begin{teo}Demostrar:
\begin{enumerate}[\bfseries a)]
\item Demostrar que: ${k \choose n} = {n \choose n-k}$\\\\
Demostración .- \; 

\item Sabiendo que ${n \choose 10} = {n \choose 7}$ calcular $n$\\\\
Demostración.- \; 

\item Sabiendo que ${14 \choose k} = {14 \choose k-4}$ calcular $k$\\\\
Demostración.- \; 

\item ¿Existe un $k$ tal que ${12 \choose k} = {12 \choose k-3 }$\\\\
Demostración.- \;
\end{enumerate}

\end{teo}



\subsection[Ejercicios]{Ejercicios\footnote{Cálculo infinitesimal, Michael Spivak,Pag. 35 al 45}}
%ejercicio 
\begin{ej}
Demostrar por inducción la siguiente fórmula: $1^2+...+n^2=\dfrac{n(n+1)(2n+1)}{6}$\\\\
Demostración.- \; Sea $n=k$: $$1^2+...+k^2=\dfrac{k(k+1)(2k+1)}{6},$$ Para $k=1$, $$1^2=\dfrac{1(1+2)(2+1)}{6}$$ por lo tanto se cumple para $k=1$, Luego para $k=k+1$, $$1^2+...+(k+1)^2=\dfrac{(k+1)(k+2)(2k+3)}{6},$$ así cabe demostrar que:
\begin{center}
\begin{tabular}{r c l}
$\dfrac{k(k+1)(2k+1)}{6}+(k+1)^2$&=&$\dfrac{(k+1)(k+2)(2k+3)}{6}$\\\\
$\dfrac{2k^3+k^2+2k^2+k+6k^2+12k+6}{6}$&=&$\dfrac{2k^3+3k^2+6k^2+9k+4k+6}{6}$\\\\
$\dfrac{2k^3+9k^2+13k+6}{6}$&=&$\dfrac{2k^3+9k^2+13k+6}{6}$\\\\
\end{tabular}
\end{center}
\end{ej}

%ejercicio 
\begin{ej}
Encontrar una fórmula para 
\begin{enumerate}[\bfseries i)]
\item  $ \displaystyle\sum_{i=1}^{n} (2i-1) = 1 +3 +5 + ... + (2n-1)$\\
\begin{center}
\begin{tabular}{r c c c l}
1&=&1&=&$1^2$\\
1+3&=&4&=&$2^2$\\
1+3+5&=&9&=&$3^2$\\
1+3+5+7&=&16&=&$4^2$\\
1+3+5+7+9&=&25&=&$5^2$\\
\end{tabular}
\end{center}
Por lo tanto $ \displaystyle\sum_{i=1}^{n} (2i-1) = 1 +3 +5 + ... + (2n-1) = n^2$\\\\
\item $\displaystyle\sum_{i=1}^{n} (2i-1)^2 = 1^2 + 3^2 + 5^2 + ... + (2n-1)^2$
\begin{center}
\begin{tabular}{r c l l}
$1^2 + 3^2 + 5^2 + ... + (2n-1)^2$&$=$&$\left[ 1^2 +2^2 +...+(2n)^2 \right] - \left[ 2^2 + 4^2 +6^2 +...+ (2n)^2\right]$& verifique sustrayendo\\\\
&$=$&$\left[ 1^2 + 2^2 + ...+ (2n)^2 \right] - 4\left[ 1^2 + 2^2 +3^2 + ... + n^2 \right]$&\\\\
&$=$&$\dfrac{2n(2n+1)(4n+1)}{6} -\dfrac{4n(n+1)(2n+1)}{6}$&por ejercicio 5.13\\\\
&$=$&$\dfrac{2n(2n+1)\left[ 4n+1 -2 (n+1) \right]}{6}$&\\\\
&$=$&$\dfrac{n(2n+1)(2n-1)}{6}$&\\\\
\end{tabular}
\end{center}
\end{enumerate}
\end{ej}

% problema 6 pag 37
\begin{ej}
La fórmula para $1^2 + 2^2 + ... + n^2$ se puede obtener como sigue: Empezamos con la fórmula $$(k+1)^3 - k^3 = 3k^2 + 3k +1$$
particularmente esta fórmula para $k=1,...,n$ \; y sumando, obtenemos \begin{center}
\begin{tabular}{r c l}
$2^3 - 1^3$&=&$3\cdot +1$\\
$3^3 + 2^3$&=&$2^2 + 3\cdot 2 +1 $\\
$.$&=&\\
$.$&=&\\
$.$&=&\\
$(n+1)^3 - n^3$&=&$n^2 + 3 \cdot n + 1$\\
\hline
$(n+1)^3 - 1$&=&$3 \left[ 1^2 + ... + n^2 \right] + 3 \left[ 1 + .... + n \right] + n$\\
\end{tabular}
\end{center}
De este modo podemos obtener $\displaystyle\sum_{k=1}^n k^2$ una vez conocido $\displaystyle\sum_{k=1}^n k $ (lo cual puede obtenerse mediante un procedimiento análogo). Aplíquese este método para obtener.
\begin{enumerate}[\bfseries i)]
\item $1^3 + ... n^3$\\\\
Sea $(k+1)^4 - k^4 = 4k^3 + 6k^2 +4k + 1, \; \; para \; k=1,...,n $ por hipótesis tenemos $(n+1)^4 - 1 = 4 \displaystyle\sum_{k=1}^n k^2 + 6 \sum_{k=1}^n k^2 + 4 \sum_{k=1}^n k + n,$ de modo que $$\displaystyle\sum_{k=1}^n k^3 = \dfrac{ (n+1)^4 -1 - 6 \dfrac{n(n+1)(2n+1)}{6} - 4 \dfrac{n(n+1)}{2} - n}{4} = \dfrac{n^4}{4} + \dfrac{n^3}{2} + \dfrac{n^2}{4}$$\\\\

\item $1^4 + ... + n^4$\\\\
Similar al anterior ejercicio partimos de $(k+1)^5 - k^5 = 5k^4 + 10k^3 + 10k^2 + 5k + 1 \; \; \; k=1,...,n$ para obtener $(k+1)^5 - k^5 = 5 \left( \displaystyle\sum_{k=1}^n k^4 \right) + 10 \left( \sum_{k=1}^n k^3 \right) + 10 \left( \sum_{k=1}^n k^2 \right) + 5 \left( \sum_{k=1}^n k \right) + n,$ así $$\displaystyle\sum_{k=1}^n k^4 = \dfrac{(n+1)^5 - 1 - 10\left( \dfrac{n^4}{4} + \dfrac{n^3}{2} + \dfrac{n^2}{4} - 10 \dfrac{n(n+1)(n+2)}{6} - 5 \dfrac{n(n+1)(n+2)}{2} -n \right)}{5} = $$ $$\dfrac{n^5}{5} + \dfrac{n^4}{2} + \dfrac{n^3}{3} - \dfrac{n}{30}$$\\\\

\item $\dfrac{1}{1 \cdot 2} + \dfrac{1}{2 \cdot 3} + ... + \dfrac{1}{n(n+1)}$\\\\ 
A partir de $$\dfrac{1}{k} - \dfrac{1}{k+1} = \dfrac{1}{k(k+1)}, \; \; \; k=1,...,n$$ obtenemos $$1-\dfrac{1}{n+1} = \displaystyle\sum_{k=1}^n \dfrac{1}{k(k+1)}$$ {\color{green} completarlo xxxxxxxxxxxxxxxxxx}\\\\

\item $\dfrac{3}{1^2 \cdot 2^2} + \dfrac{5}{2^2 \cdot 3^2} + ... + \dfrac{2n +1}{n^2 (n+1)^2}$\\\\
De $$\dfrac{1}{k^2} - \dfrac{1}{(k+1)^2} = \dfrac{2k+1}{k^2(k+1)^2}, \; \; \; k=1,...,n$$ obtenemos $$1-\dfrac{1}{(n+1)^2} = \displaystyle\sum_{k=1}^n \dfrac{2k+1}{k^2(k+1)^2}$$ {\color{green} completarlo xxxxxxxxxxxxxxxxxx} \\\\
\end{enumerate}
\end{ej}

\subsection[Teoremas]{Teoremas\footnote{Cálculo infinitesimal, Michael Spivak,Pag. 35 al 45}}

%problema 3 pag 35
\begin{teo}
Si $0 \leq k \leq n,$ se define el coeficiente binomial $ {n \choose k} $ por $${n \choose k} = \dfrac{n!}{k!(n-k)!}=\dfrac{n(n-1)...(n - k + 1)}{k!}, \; si \; k \neq 0, \; n$$ $${n \choose 0} = {n \choose n} = 1.$$ Esto se convierte en un caso particular de la priera fórula si se define $0! = 1.$
\begin{enumerate}[\bfseries a)]
\item Demostrar que $${n +1 \choose k} = {n \choose k - 1} + {n \choose k}$$ Esta relación de lugar a la siguiente configuración, conocida por triángulo de Pascal: Todo número que no esté sobre uno de los lados es la suma de los dos números que tiene encima: El coeficiente binomial ${n \choose k}$ es el número k-ésimo de la fila $(n+1)$.
\begin{center}
\begin{tabular}{ccccccccccc}
  &    &    &    &    &  1 &    &    &    &    &   \\
  &    &    &    &  1 &    &  1 &    &    &    &   \\
  &    &    &  1 &    &  2 &    &  1 &    &    &   \\
  &    &  1 &    &  3 &    &  3 &    &  1 &    &   \\
  &  1 &    &  4 &    &  6 &    &  4 &    &  1 &   \\
1 &    &  5 &    & 10 &    & 10 &    &  5 &    & 1 \\\\
\end{tabular}
\end{center}
Demostración.- \; \\
\begin{center}
\begin{tabular}{r c l}
$ {n \choose k-1}  +  {n \choose k} $&$=$&$\dfrac{n!}{(k-1)(n-k+1)!}+ \dfrac{n!}{k!(n-k)!}$\\\\
&$=$&$\dfrac{kn!}{k!(n+1-k)!} + \dfrac{(n+1-k)n!}{k!(n+1-k)!}$\\\\
&$=$&$\dfrac{(n+1)n!}{k!(n+1-k)!}$\\\\
&$=$&$ {n+1 \choose k} $\\\\
\end{tabular}
\end{center}
\item Obsérvese que todos los números del triángulo de Pascal son números naturales. Utilícese la parte $(a)$ para demostrar por inducción que $ {n \choose k}$ es siempre un número natural.\\\\
Demostración.- \; Se ve claramente que ${1 \choose 1}$ es un número natural. Supóngase que ${n \choose p}$ es un número natural para todo $p \leq n$. Al ser: $${ n+1 \choose p } = {n \choose p-1} + {n \choose p} \; para \; p \leq n,$$ se sigue que ${n+1 \choose p}$ es un número natural para todo $p \leq n,$ mientras que ${n+1 \choose n+1}$ es también un número natural. Así pues, ${n+1 \choose p}$ es un número natural para todo $p \leq n+1$\\\\

\item Dése otra demostración de que ${n \choose k}$ es un número natural, demostrando que ${n \choose k}$ es el número de conjuntos de exactamente $k$ enteros elegidos cada uno entre $1,...,n$.\\\\
Demostración.- \; Existen $n(n -1) \cdot ... \cdot (n - k + 1)$ k-tuplas de enteros distintos elegidos entre $1, ..., n$, ya que el primero puede ser elegido de n maneras, el segundo de $n - 1$ maneras, etc. Ahora bien, cada conjunto formado exactamente por $k$ enteros distintos, da lugar a $k!$ k-tuplas, de
modo que el número de conjuntos será $n(n — 1) \cdot ... \cdot (n - k + l)/k! = {n\choose k}$\\\\ 

\item Demostrar el \textbf{TEOREMA DEL BINOMIO}: Si $a$ \; y \; $b$ son números cualesquiera, entonces
$(a+b)^n = a^n + {n \choose 1} a^{n-1} b + {n \choose 2} a^{n-2} b^2 + ... + {n \choose n-1} a b^{n-1} + b^n = \displaystyle \sum_{j=0}^n {n \choose j} a^{n-j} b^j.$\\\\
El teorema del binomio resulta claro para $n=1.$ Supónganse que $$(a+b)^n = \displaystyle \sum_{j=0}^n {n \choose j} a^{n-j} b^j.$$
Entonces 
\begin{center}
\begin{tabular}{r c l l}
$(a+b)^{n+1}$&=&$(a+b)(a+b)^n$&\\\\
&=&$(a+b) \displaystyle \sum_{j=0}^n {n \choose j} a^{n-j} b^j$&\\\\
&=&$\displaystyle \sum_{j=0}^n {n \choose j} a^{n+1-j} b^j + \sum_{j=0}^{n} {n \choose j} a^{n-j} b^{j+1}$&\\\\
&=&$\displaystyle \sum_{j=0}^n {n \choose j} a^{n+1-j} b^j + \sum_{j=0}^{n+1} {n \choose j-1} a^{n+1-j} b^{j}$&sustituimos $j$ por $j-1$ en la $2^{da}$ suma\\\\
&=&$\displaystyle \sum_{j=0}^{n+1} {n+1 \choose j} a^{n+1-j} b^j$& por la parte $b)$\\\\
\end{tabular}
\end{center}
Según la parte $a)$, con lo que el teorema del binomio es válido para $n+1.$\\\\

\item Demostrar que 
\begin{enumerate}[\bfseries i)]
\item $\displaystyle\sum_{j=0}^{n} {n \choose j} = {n \choose 0} + ... + {n \choose n} = 2^n$\\\\
Demostración.- \; Por el teorema del binomio $2^n=(1+1)^n = \displaystyle\sum_{j=0}^n {n \choose j}(1^j)(1^{n-j})=\sum_{j=0}^n {n \choose j}$\\\\

\item $\displaystyle\sum_{j=0}^n (-1)^j {n \choose j} = {n \choose 0}- {n \choose 1}+...\pm {n \choose n} =0$\\\\
Demostración.- \;  De igual manera por el teorema del binomio $0=(1+(-1))^n = \displaystyle\sum_{j=0}^n(-1)^j {n \choose j}$\\\\ 

\item $\displaystyle\sum_{l \; impar} {n \choose l} = {n \choose 1} + {n \choose 3}+ ... = 2^{n-1}$\\\\
Demostración.- \;  {\color{green}Terminar demostración xxxxxxxxxxxxxxxxxxxxxxxxxxxxxxxxxxxxxxx}

\item $\displaystyle\sum_{l \; par} {n \choose l} = {n \choose 0} + {n \choose 2} + ...  = 2^{n-1}$\\\\
Demostración.- \; {\color{green}Terminar demostración xxxxxxxxxxxxxxxxxxxxxxxxxxxxxxxxxxxxxxx}
\end{enumerate}
\end{enumerate}
\end{teo}


%problema 3.4 pag 37
\begin{teo}Demostrar:
\begin{enumerate}[\bfseries a)]
\item Demostrar que $$\displaystyle\sum_{k=0}^{l} {n \choose k} {m \choose l-k} = {n+m \choose l}$$\\\\
Demostración.- \; Aclaremos la proposición primeramente con un ejemplo.\\
Tenemos $n$ hombres y $m$ mujeres, y queremos formar un conjunto de combinaciones $l$ de personas de la forma $n+m$. Claramente hay ${n+m \choose l}$ caminos para formar dichas combinaciones.\\
Contemos el número de combinaciones de otra manera. Tenemos $0$ hombres y $l$ mujeres. Tal combinación se pueden formar en ${n \choose 0 } {m \choose l}$ caminos, o también:
\begin{itemize}
\item Podemos tener $1$ hombre y $l-1$ mujeres. Tal combinación se puede formar como ${n \choose l}{m \choose l-1}$ maneras.
\item Ó podemos tener $2$ hombre y $l-2$ mujeres. Tal combinación se puede formar como ${n \choose 2}{m \choose l-2}$ maneras.
\end{itemize}
Así las combinaciones totales serian $$\displaystyle\sum_{k=0}^l {n \choose k}{m \choose l -k}$$
Pero ya vimos que el número de combinaciones es ${n+m \choose l}$ siempre y cuando se cumpla la condición ${a \choose b} = 0$ si $b>a$\\
Ahora si pasemos a demostrar. Sea por producto de Cauchy de dos series infinitas:
$$\left( \displaystyle\sum_{k=0}^{\infty} a_k x^k \right) \left( \displaystyle\sum_{k=0}^{\infty} b_k x^k \right) = \sum_{k=0}^{\infty} \left( \displaystyle\sum_{j=0}^{k} a_j x^j b_{k-j} x^{k-j}\right) = \sum_{k=0}^{\infty} \left( \displaystyle\sum_{j=0}^{k} a_j b_{k-j} \right) x^k$$ entonces $(1+x)^m = \displaystyle\sum_{k=0}^{\infty} {m \choose k}x^k$ y $(1+x)^n \displaystyle\sum_{k=0}^{\infty} {m \choose k}x^k$ entonces queda:
$$(1+x)^{m+n} = \sum_{k=0}^{\infty} {m+n \choose k} x^k$$
Se extiende los índices en las sumas a $\infty$  ya que $k>n$, ${n \choose k} = 0 $. Luego 
$$(1+x)^m (1+x)^n = \displaystyle\sum_{k=0}^{\infty} \left( \sum_{j=0}^k {m \choose j} {n \choose k-j} \right) x^k$$ Así 
$${m+n \choose k} = \displaystyle\sum_{j=0}^k {j \choose m}{n \choose k-j}$$\\\\


\item demostrar que $$\displaystyle\sum_{k=0}^{n} {n \choose k}^2 = {2n \choose n}$$\\\\
Demostración.- \; Sea $m$, $l=n$ en la parte $a)$ y notar que ${n \choose k} = {n \choose n-k}.$ \\\\ {\color{green}completar demostración xxxxxxxxxxxxxxxxxxxxxxxxxxxxxxxx}
\end{enumerate}
\end{teo}

%problema 5 a) pag 37
\begin{teo}Demostrar:
\begin{enumerate}[\bfseries a)]
\item Demostrar por inducción sobre $n$ que $$1 + r +r^2 + ... + r^n = \dfrac{1 - r^{n+1}}{1-r}$$ si $r\neq 1$ (Si es $r=1$, el cálculo de la suma no presenta problema alguno).\\\\
Demostración.- \; Sea $n=1$ entonces $$1+r = \dfrac{1- r^2}{1-r}$$ el cual vemos que se cumple.\\
Luego
\begin{center}
\begin{tabular}{r c l}
$1+r+r^2 + ... + r^n + r^{n+1}$&$=$&$\dfrac{1- r^{r+1}}{1-r} + r^{r+1}$\\\\
&$=$&$\dfrac{1 - r^{n+1} + r^{n+1} (1-r)}{1-r}$\\\\
&$=$&$\dfrac{1 - r^{n+1}}{1-r}$\\\\
\end{tabular}
\end{center} 

\item Deducir este resultado poniendo $S=1+r+...+r^n$, multiplicando esta ecuación por $r$ y despejando $S$ entre las dos ecuaciones.\\\\
Tenemos $r\cdot S = r + .... + r^n + r^{n+1}$ luego $S - rS = S(1-r) = 1-r^{n+1}$ por lo tanto $S = \dfrac{1-r^{n+1}}{1-r}$\\\\
\end{enumerate}
\end{teo}

%capítulo 2, problema 7, pag 38
\begin{teo}
Utilizar el método del problema 6 para demostrar que $\displaystyle\sum_{k=1}^n k^p$ puede escribirse siempre en la forma $$\dfrac{n^{p+1}}{p+1} + An^p + Bn^{p-1} + Cn^{p-2} + ...$$
Las diez primeras de estas expresiones son
\begin{center}
\begin{tabular}{r c l}
$\displaystyle\sum_{k=1}^n k$&=&$\dfrac{1}{2}n^2 + \dfrac{1}{2}n$\\\\
$\displaystyle\sum_{k=1}^n k^2$&=&$\dfrac{1}{3} n^3 + \dfrac{1}{2}n^2 + \dfrac{1}{6}n$\\\\
$\displaystyle\sum_{k=1}^n k^3$&=&$\dfrac{1}{4}n^4 + \dfrac{1}{2} n^3 + \dfrac{1}{4} n^2$\\\\
$\displaystyle\sum_{k=1}^n k^4$&=&$\dfrac{1}{5} n^6 + \dfrac{1}{2} n^4 + \dfrac{1}{3} n^3 - \dfrac{1}{30}n$\\\\
$\displaystyle\sum_{k=1}^n k^5$&=&$\dfrac{1}{6}n^6 + \dfrac{1}{2} n^5 + \dfrac{5}{12}n^4 - \dfrac{1}{12}n^2$\\\\
$\displaystyle\sum_{k=1}^n k^6$&=&$\dfrac{1}{7}n^7 + \dfrac{1}{2}n^6 + \dfrac{1}{2}n^5 - \dfrac{1}{6}n^3 + \dfrac{1}{42}n$\\\\
$\displaystyle\sum_{k=1}^n k^7$&=&$\dfrac{1}{8}n^8 + \dfrac{1}{2}n^7 + \dfrac{7}{12}n^6 - \dfrac{7}{24}n^4 + \dfrac{1}{12}n^2$\\\\
$\displaystyle\sum_{k=1}^n k^8$&=&$\dfrac{1}{9}n^9 + \dfrac{1}{2}n^8 + \dfrac{2}{3}n^7 - \dfrac{7}{15}n^5 + \dfrac{2}{9}n^3 - \dfrac{1}{30}n$\\\\
$\displaystyle\sum_{k=1}^n k^9$&=&$\dfrac{1}{10}n^{10} + \dfrac{1}{2} n^9 + \dfrac{3}{4}n^8 - \dfrac{7}{10}n^6 + \dfrac{1}{2}n^4 - \dfrac{3}{20} n^2$\\\\
$\displaystyle\sum_{k=1}^n k^10$&=&$\dfrac{1}{11}n^{11} + \dfrac{1}{2}n^{10} + \dfrac{5}{6}n^9 - 1n^7 + 1n5 - \dfrac{1}{2}n^3 + \dfrac{5}{66}n$\\\\
\end{tabular}
\end{center}
Obsérvese que los coeficientes de la segunda columna son siempre $\dfrac{1}{2}$ y que después de la tercera columna las potencias de $n$ de coeficiente no nulo van decreciendo de dos en dos hasta llegar a $n^2$ o a $n$. Los coeficientes de todas las columnas, salvo las dos primeras, parecen bastante fortuitos, pero en realidad obedecen a cierta regla; encontrarla puede considerarse como una prueba de superspicacia. Para descifrar todo el asunto, véase el problema 26-17)\\\\
Demostración.- \; La prueba se hará por inducción en $p$. La afirmación es verdadera para $p=1$, ya que $$\displaystyle\sum_{k=1}^n k = \dfrac{n(n+1)}{2} = \dfrac{n^2}{2} + n.$$
Suponemos que la afirmación es verdadera para todos los números naturales $\leq p$. Por el teorema binomial, tenemos la ecuación, $$(k+1)^{p+1} - k^{p+1} = (p+1)k^p + \mbox{terminos que implican las potencias inferiores a k}$$
Agragando para $k=1,...,n$ obtenemos,
$$\dfrac{(n+1)^{p+1}}{p+1} = \displaystyle\sum_{k=1}^n k^p + \mbox{terminos que involucran} \; \sum_{k=1}^n k^r \; para \; r<p$$
por suposición, tenemos que escribir cada $\displaystyle\sum_{k=1}^n k^r$ como una expresión que involucra las potencias de $n^s$ con $s\leq p$. Se sigue que $$\displaystyle\sum_{k=1}^n k^p = \dfrac{(n+1)^{p+1}}{p+1} + \; \mbox{terminos que involucran las potencias de n menos } \; p+1$$
\end{teo} 

%capítulo 2, problema 8.
\begin{teo}
Demostrar que todo número natural es o par o impar.\\\\
La demostración esta dada ya que es igual al capítulo 1, ejercicio 3.12, problema 10 Calculus Vol 1 Tom Apostol\\\\
\end{teo}

%capitulo 2, problema 9.
\begin{teo}
Demostrar que si un conjunto $A$ de números naturales contiene $n_0$ y contiene $k+1$ siempre que contenga $k$, entonces $A$ contiene todos los números naturales $\geq n_0$.\\\\
Demostración.- \; Sea $1 \in A$ y que $n, \; n+1 \in A$. Si $A$ no contiene a todos lo números naturales, entonces el conjunto $B$ de números naturales que no están en $A$ es distinto de $\emptyset$. POr lo tanto $B$ tiene un elemento mínimo $n_o$. Ahora bien, $n_o \neq 1$, ya que $1 \in A$, de modo que podemos poner $n_o = (n_o -1) + 1$ donde $n_o - 1$ es un número natural. Pero $n_o-1$ no está en $B$ y por lo tanto $n_o -1$ está en $A$. Por hipótesis, $n_o$, tiene que estar en $A$, con lo que $n_o$ no está en $B$, contrario a lo supuesto.\\\\
\end{teo}

%capítulo 2, problema 10
\begin{teo}
Demostrar el principio de inducción completa a partir del principio de buena ordenación.\\\\
Que $1$ está en $B$ claro. Si $k$ está en $B$, entonces $1,...,k$ están todos en $A$, de modo que $k+1$ está en $A$ y así $1,...,k+1$ están en $A$, con lo que $k+1$ está en $B$. Por inducción, $B=N$, así que también $A=N$.\\\\
\end{teo}

%capítulo 2, problema 11
\begin{teo}
Demostrar el principio de inducción completa a partir del principio de inducción ordinario.\\\\
La demostración se encuentra en la página 38 y 39 del libro calculus vol 1 de Tom Apostol.\\\\
\end{teo}

%capítulo 2, problema 12
\begin{teo}Demostrar
\begin{enumerate}[\bfseries a)]
\item Si $a$ es racional y \; $b$ es irracional ¿es $a+b$ necesariamente irracional? ¿Y si $a$ \; y \; $b$ es irracional? \\\\
El mismo ejercicio se encuentra en el capítulo I ejercicio 3.12 problema 7 de calculus vol 1 de Tom Apostol que ya fue demostrada en este libro.\\\\

\item Si $a$ es racional y \; $b$ es irracional, ¿es $ab$ necesariamente irracional?\\\\
El mismo ejercicio se encuentra en el capítulo I ejercicio 3.12 problema 7 de calculus vol 1 de Tom Apostol que ya fue demostrada en este libro.\\\\

\item ¿Existe algún número $a$ tal que $a^2$ es irracional pero $a^4$ racional?\\\\
Si existe por ejemplo $\sqrt[4]{2}$\\\\

\item ¿Existen dos números iracionales tales que sean racionales tanto su suma como su producto?\\\\
Si existen por ejemplo $\sqrt{2}$ y $- \sqrt{2}$\\\\
\end{enumerate}
\end{teo}

%capítulo 1, problema 13
\begin{teo}Demostrar
\begin{enumerate}[\bfseries a)]
\item Demostrar que $\sqrt{3}$, $\sqrt{5}$ y $\sqrt{6}$ son irracionales. Indicación: Para tratar $\sqrt{3},$ por ejemplo, aplíquese el hecho de que todo entero es de la forma $3n$ ó $3n+1$ ó $3n+2$ ¿Por qué no es aplicable esta demostración para $\sqrt{4}$?\\\\
Demostración.- \; Puesto que:
\begin{center}
\begin{tabular}{r c l c l}
$(3n+1)^2$&=&$9n^2 + 6n + 1$&=&$3(3n^2+2n) + 1$\\
$(3n+2)^2$&=&$9n^2+12n + 4$&=&$3(3n^2 + 4n + 1) + 1$\\
\end{tabular}
\end{center}
queda demostrado que un número no es múltiplo de $3$ si es de la forma $3n+1$ ó $3n+2$.\\
se sigue que $k^2$ es divisible por $3$, entones $k$ debe ser también divisible por $3$. Supóngase ahora que $\sqrt{3}$ fuese racional, y sea $\sqrt{3} = p/q$, donde $p$ \; y \; $q$ no tienen factores comunes. Entonces $p^2=3q^2,$ de modo que $p^2$ es divisible por $3$, así que también lo debe ser $p$. de este modo, $p=3p^{'}$ para algún número natural $p^{'}$, y en consecuencia $(3p^{'})^2 = 3q^2$ ó $(3p^{'})^2 = q^2.$ Así pues, $q$ es tambien divisible por $3$, lo cual es una contradicción.\\
Las mismas demostraciones valen para $\sqrt{5}$ y $\sqrt{6}$, ya que las ecuaciones,
\begin{center}
\begin{tabular}{rclcl}
$(5n+1)^2$&=&$25n^2 + 10n + 1$&=&$5(5n^2 + 2n)+1$\\
$(5n+2)^2$&=&$25n^2 + 20n + 4$&=&$5(5n^2 + 4n)+4$\\
$(5n+3)^2$&=&$25n^2 + 30n + 9$&=&$5(5n^2 + 6n + 1)+4$\\
$(5n+4)^2$&=&$25n^2 + 40n + 16$&=&$5(5n^2+8n+3)+1$\\
\end{tabular}
\end{center}
la ecuación correspondiente para los números de la forma $6n+m$ demuestran que si $k^2$ es divisible por $5$ ó $6$, entones también lo debe ser $k$. La demostración falla para $\sqrt{4}$, porque $(4n+2)^2$ es divisible por $4$.\\\\

\item Demostrar que $\sqrt[3]{2}$ y $\sqrt[3]{3}$ son irracionales.\\\\
Demostración.- \; Puesto que,
$$(2n+1)^3 = 8n^3 + 12n^2 + 6n + 1 = 2(4n^3 + 6n^2 + 3n) + 1,$$
se sigue que si $k^3$ es par, entonces $k$ es par. Si $\sqrt[3]{2} = p/q,$ donde $p$ \; y \; $q$ no tienen factores comunes, entonces $p^3 = 2q^3,$ de modo que $p^3$ es divisible por $2,$ por lo que también lo debe ser $p.$ Así pues, $p=2p^{'}$ para algún número natural $p^{'}$ y en consecuencia $(2p^{'})^3 = 2q^3,$ ó $4(p^{'})^3 = q^3.$ Por lo tanto, $q$ es también par, lo cual es una contradicción.\\
La demostración para $\sqrt[3]{3}$ es análogo, utilizando las ecuaciones.
$$(3n+1)^3 = 27n^3 + 27n^3 + 27n^2 + 9n +1 = 3(9n^3 + 9n^2 + 3n) + 1,$$
$$(3n+2)^3 = 27n^3 + 54n^2 + 36n + 8 = 3(9n^2 + 18n^2 + 12n + 2) + 2.$$\\\\
\end{enumerate}
\end{teo}

% capítulo 2 problema 14
\begin{teo} Demostrar:
\begin{enumerate}[\bfseries a)]
\item $\sqrt{2} + \sqrt{3}$ es irracional.\\\\
Demostración.- \; Sea $\sqrt{2} + \sqrt{3}$ racional, entonces $\left( \sqrt{2} + \sqrt{3} \right)^2$ sería racional, luego $$5 + 2 \sqrt{6}$$ y en consecuencia $\sqrt{6}$ sería racional lo cual es falso.\\\\

\item $\sqrt{6} - \sqrt{2} - \sqrt{3}$ es irracional.\\\\
Demostración.- \;  Sea $\sqrt{6} - \sqrt{2} - \sqrt{3}$ racional, entonces 
\begin{center}
\begin{tabular}{rcl}
$\left[ \sqrt{6} + \left( \sqrt{2} + \sqrt{3} \right) \right]^2$&=&$6 + \left( \sqrt{2} + \sqrt{3} \right)^2 - 2 \sqrt{6} \left( \sqrt{2} + \sqrt{3} \right)$\\
&=&$11 + 2\sqrt{6} \left[ 2 - \left( \sqrt{2} + \sqrt{3} \right) \right]$\\
\end{tabular}
\end{center}
Así, $\sqrt{6} \left[ 2 - \left( \sqrt{2} + \sqrt{3} \right) \right]$ sería racional, con lo que de igual manera sería,
\begin{center}
\begin{tabular}{r c l}
$\lbrace \sqrt{6} \left[ 1 - \left( \sqrt{2} + \sqrt{3} \right) \right] \rbrace ^2$&=&$6 \left[ 1 - \left( \sqrt{2} + \sqrt{3} \right) \right]^2$\\
&=&$11 + 2\sqrt{6} \left[1 - \left( \sqrt{2} + \sqrt{3} \right) \right]$\\
\end{tabular}
\end{center}
De este modo $\sqrt{6} - (\sqrt{2} + \sqrt{3})$ y $\sqrt{6} - 2 \left( \sqrt{2} + \sqrt{3} \right)$ serían racionales, lo que implicaría que $\sqrt{2} + \sqrt{3}$ fuese racional, en contradicción de la parte $a)$.\\\\
\end{enumerate}
\end{teo}

%capitulo 2 problema 15 pag 40
\begin{teo} Demostrar
\begin{enumerate}[\bfseries a)]
\item Demostrar que si $x=p+ \sqrt{q}$, donde $p$ \; y \; $q$ son racionales, y \; $m$ es un número natural, entonces $x^m = a + b \sqrt{q}$ siendo $a$ \; y \; $b$ números racionales.\\\\
Demostración.- \; Sea $m=1$ entonces $(p + \sqrt{q})^1 = a + b\sqrt{q}$. Supongamos que se cumple para $m$, entonces $$(p+\sqrt{q})^{m+1} = (a + b\sqrt{q})(p + \sqrt{q}) = (ap+bq)+(a+pb)\sqrt{q}$$ donde $ap+bq$ y $a+bp$ son racionales.\\\\

\item Demostrar también que $(p - \sqrt{q})^m = a - b\sqrt{q}$\\\\
Demostración.- \; Similar a la parte $a)$, se cumple para $m=1$. Si es verdad para $m$, entonces $$(p-\sqrt{q})^{m+1} = (a - b\sqrt{q})(p - \sqrt{q}) = (ap+bq)-(a+pb)\sqrt{q}$$.\\\\
\end{enumerate}
\end{teo}

%capitulo 2 problema 16 pag 40
\begin{teo}Demostrar
\begin{enumerate}[\bfseries a)]
\item Demostrar que si $m$ \; y \; $n$ son números naturales y $m^2/n^2 < 2$, entonces $\left( m+2n \right)^2 / \left( m + 2 \right)^2 > 2;$ demostrar, además que $$\dfrac{\left( m + 2n \right)^2}{\left( m + n \right)^2} - 2 < 2 - \dfrac{m^2}{n^2}$$\\\\
Demostración.- \; Si $m^2/n^2 < 2$ entonces $m^2 < 2 n^2$, sumando $m^2$, $4mn$ y $2n^2$ tenemos $2m^2 + 4mn + 2n^2 < 4n^2 + m^2 + 4mn$, luego $2(m+n)^2 < (m+2n)^2$, así nos queda $(m + 2n)^2 / (m+n)^2 > 2$\\
Para la segunda parte podemos partir de $m^2 - 2n^2<0$, entonces:

\begin{center}
\begin{tabular}{r c l l}
$m^2 - 2n^2$&$<$&$0$&\\\\
$m^3 - 2mn^2$&$<$&$0$&multiplicando por $m$\\\\
$mn^2 + m^3 + mn^2 - 4mn^2$&$<$&$0$&escribiendo $mn^2$ de otra manera\\\\
$mn^2 + 2n^3 + m^3 + m^2 n + mn^2 -2m^2 n - 4mn^2 -2n^3$&$<$&$0$&sumando $2n^3$ y $2m^2 n$\\\\
$n^2 (m+2n) + \left[ (m^2 + 2mn + n^2)(m-2n) \right]$&$<$&$0$&\\\\
$ n^2(m+2n)^2  +  \left[ (m+n)^2 (m+2n)(m-2n) \right]$&$<$&$0$&multiplicando por $m+2n$\\\\
$n^2(m+2n)^2  +  \left[ (m+n)^2 (m^2 - 4n^2) \right]$&$<$&$0$&\\\\
$\dfrac{n^2(m+2n)^2 - 4n^2(m+n)^2 + m^2(m+n)^2}{n^2(m+n)^2}$&$<$&$0$&dividimos por $n^2(m+n)^2$\\\\
$\dfrac{(m+2n)^2 - 2(m+2)^2 - 2n^2(m+2)^2}{n^2(m+n)^2}$&$<$&$- \dfrac{m^2}{n^2}$&\\\\
$\dfrac{(m+2n)^2}{(m+n)^2} - 2$&$<$&$2 - \dfrac{m^2}{n^2}$&\\\\
\end{tabular}
\end{center}

\item Demostrar los mismos resultados con todos los signos de desigualdad invertidos. \\\\
Demostración.- \; Quedará de la siguiente forma, Si $m^2/n^2>2$, entonces $\left( m+2n \right)^2 / \left( m + 2 \right)^2 < 2$, luego demostrar que $$\dfrac{\left( m + 2n \right)^2}{\left( m + n \right)^2} - 2 > 2 - \dfrac{m^2}{n^2}$$
Similar a la parte $a)$ tendremos $m^2 > 2n^2$, luego $2m^2 + 4mn + 2n^2 > 4n^2 + m^2 + 4mn$, así $ (m+2n)^2 > 2(m + n)^2 $\\
Después se puede demostrar la segunda parte con facilidad siguiendo el ejemplo $a)$ pero invirtiendo la desigualdad ya que $n$ \; y \; $m$ son números natural.\\\\

\item Demostrar que si $m/n < \sqrt{2}$, entonces existe otro número racional $m^2 / n^2$ con $m/n < m^{'} / n^{'} < \sqrt{2}$\\\\
Demostración.- \; Sea $m^{'}/n^{'}$ un número racional, por el ejercicio I 3.12 problema 6 de Calculus de Tom Apostol Vol I, queda demostrada la proposición.\\\\
\end{enumerate}
\end{teo}

%chapter 2 problema 14 
\begin{teo}
Parece normal que $\sqrt{n}$ tenga que ser irracional siempre que el número natural $n$ no sea el cuadrado de otro número natural. Aunque puede usarse en realidad el método del problema 13 del capitulo 2 de Michael Spivak para tratar cualquier caso particular, no está claro, sin más, que este método tenga que dar necesariamente resultados, y para una demostración del caso general se necesita más información. Un número natural $p$ se dice que es un número primo si es imposible escribir $p=ab$; por conveniencia se considera que $1$ no es un número primo. Los primeros números primos son $2,3,5,7,11,13,17,19.$ Si $n>1$ no es primo, entonces $n=ab,$ con $a$ \; y \; $b$ ambos $<n;$ si uno de los dos $a$ \; o \; $b$ no es primo, puede ser factorizado de manera parecido; continuando de esta manera se demuestra que se puede escribir $n$ como producto de números primos. Por ejemplo, $28=2\cdot 2\cdot 7.$\\
\begin{enumerate}[\bfseries a)]
\item Conviértase este argumento en una demostración riguroso por inducción completa. (En realidad, cualquier matemático razonable aceptaría este argumento informal, pero ello se debería en parte a que para él estaría claro cómo formularla rigurosamente.)\\
Un teorema fundamental acerca de enteros, que no demostraremos aquí, afirma que esta factorización es única, salvo en lo que respeta al orden de los factores. Así, por ejemplo, $28$ no puede escribirse nunca como producto de números primos uno de los cuales sea $3$, ni puede ser escrito de manera que $2$ aparezca una sola vez (ahora debería verse clara la razón de no admitir a $1$ como número primo.)\\\\
demostración.- \; Supóngase que para todo número $<n$ puede ser escrito  como un producto de primos. Si $n>1$ no es primo, entonces $n=ab$, para $a,b<n$. Pero $a$ \; y \; $b$ son ambos producto de primos, así que $n=ab$ lo es también.\\\\

\item Utilizando este hecho, demostrar que $\sqrt{n}$ es irracional a no ser que $n=m^2$ para algún número natural $m.$\\\\
Demostración.- \; Sea $\sqrt{n} = \dfrac{p}{q}$, entonces $nb^2 = a^2,$  luego si descomponemos en producto de factores primos, $nb^2$ y $a^2$ deberían coincidir. Ahora según lo explicado anteriormente, cada número primo debe aparecer un número par de veces en $a^2$ y $b^2$, y por lo tanto deberá ocurrir lo mismo con $n.$ Esto implica que $n$ es un cuadrado perfecto.\\\\

\item Demostrar que $\sqrt[k]{n}$ es irracional a no ser que $n=m^k$\\\\
Demostración.- \; 

\item Al tratar de números primos no se puede omitir la hermosa demostración de Euclides de que existe un número infinito de ellos. Demuestre que no puede haber sólo un número finito de números primos $p_1, p_2, p_3,...,p_n$ considerando $p_1\cdot p_2 \cdot ... \cdot p_k + 1$\\\\
Demostración.- \;  
\end{enumerate}
\end{teo}