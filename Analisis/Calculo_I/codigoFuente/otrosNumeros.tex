e\chapter{Distintas clases de números}
\section{El principio de la inducción matemática}
%teorema 
\begin{teo}[Principio de inducción matemática]
Sea $S$ un conjunto de enteros positivos que tienen las dos propiedades siguientes:
\begin{enumerate}[\bfseries a)]
\item El número 1 pertenece al conjunto $S$.
\item Si un entero $k$ pertenece al conjunto $S$, también $k+1$ pertenece a $S$.
\end{enumerate}
Entonces todo entero positivo pertenece al conjunto $S$.\\\\
Demostración.- \; Las propiedades $a)$ y $b)$ nos dicen que $S$ es un conjunto inductivo. Por consiguiente $S$ tiene cualquier entero positivo.\\\\ 
\end{teo}

\section{Teoremas y Ejercicios}
\subsection[Ejercicios]{Ejercicios\footnote{Tom Apostol Vol 1, pag 44}}
%ejercicio 5.1
\begin{ej}
Demostrar por inducción las fórmulas siguientes:
\begin{enumerate}[\bfseries a)]
\item $1+2+3+...+n=n(b+1)/2$\\\\
Demostración.- \; Sea $n=k$ entonces $1+2+3+...+k=k(k+1)/2$.\\
Para $k=1$ se tiene $1=1(1+1)/2$. \\ 
Por ultimo si  $k=k+1$ nos queda probar que  $1+2+3+...+k+(k+1)=\dfrac{(k+1)(k+2)}{2}$, luego $\dfrac{k(k+1)}{2}+(k+1)=\dfrac{k(k+1)+2(k+1)}{2}$. Así $\dfrac{k^2+3k+2}{2}$ \; y \; $\dfrac{(x+1)(x+2)}{2}$\\\\

\item $1+3+5+...+(2n-1)=n^2$\\\\
Demostración.- \; Sea $n=k$ entonces $1+3+5+...+(2k-1)=k^2$.\\
Para $k=1$ se tiene $[2(1)-1]=1^2$, así $1=1$\\
Luego, si $k=k+1$ entonces $1+3+5+...+[2(k+1)-1]=(k+1)^2$. Por lo tanto $k^2+2k+1=(k+1)^2$.\\\\

\item $1^3+2^3+3^3+...+n^3=(1+2+3+...+n)^2$\\\\
Demostración.- \; Sea $n=k$ entonces, $$1^3+2^3+3^3+...+k^3=(1+2+3+...+k)^2$$
Para $k=1$ $$1=1,$$
Luego $k=k+1$, $$1^3+2^3+3^3+...+k^3+(k+1)^3=(1+2+3+...+k)^2,$$
Así,
\begin{center}
\begin{tabular}{r c l}
$\left(\dfrac{k(k+1)}{2} \right)^2+(k+1)^3=$&=&$\left( \dfrac{k(k+1)}{2}+(k+1)\right) ^2$\\\\
$\dfrac{k^2(k+1)^2}{4}+(k+1)^3$&=&$\dfrac{k^2(k+1)^2}{4}+k(k+1)^2+(k+1)^2$\\\\
$\dfrac{k^2(k+1)^2+4(k+1)^3}{4}$&=&$\dfrac{k^2(k+1)^2+4k(k+1)^2+4(k+1)^2}{4}$\\\\
$\dfrac{(k+1)^2 (k^2+4k+4)}{4}$&=&$\dfrac{(k+1)^2 (k^2 + 4k +4)}{4}$\\\\
\end{tabular}   
\end{center}

\item $1^3+2^3+...+(n-1)^3<n^4/4<1^3+2^3+...+n^3$\\\\
Demostración.- \; Sea $n=k$ entonces $$1^3+2^3+...+(k-1)^3<k^4/4 \; \; \; (1)$$ 
Para $k=1$, \; $0<1/4$ se observa que se cumple.\\
Después, para $k=k+1$, $$1^3+2^3+...+k^3<(k+1)^4/4,$$ sumando $k^3$ a $(1)$, 
$$1^3+2^3+...+k^3<k^4/4+k^3$$ y para deducir como consecuencia de $k+1$, basta demostrar, $$k^4/4+k^3<(k+1)^4/4$$, Pero esto es consecuencia inmediata de la igualdad $$(k+1)^4/4=(k^4+4k^3+6k^2+4k+1)/4=k^4/4+k^3+(3k^2)/2+k+1$$ Por tanto se demostró que $k+1$ es consecuencia de $k$\\\\
\end{enumerate}
\end{ej}

\begin{ej}
Obsérvese que: 
\begin{center}
\begin{tabular}{r c l}
$1$&$=$&$1$\\
$1-4$&$=$&$-(1+2)$\\
$1-4+9$&$=$&$1+2+3$\\
$1-4+9-16$&$=$&$-(1+2+3+4)$\\
\end{tabular}
\end{center}
Indúzcase la ley general y demuéstrese por inducción\\\\
Demostración.- \: Verificando tenemos que la ley general es $1-4+9-16+...+(-1)^{n+1}\cdot n^2=(-1)^{n+1}(1+2+3+...+n)$. \\
Ahora pasemos a demostrarlo. Sea $n=k$ entonces, $$1-4+9-16+...+(-1)^{k+1}\cdot k^2=(-1)^{k+1}(1+2+3+...+k)$$ Si $k=1$, se sigue, $(-1)^2 \cdot 1^2 = (-1)^2\cdot 1$, vemos que satisface para $k=1.$
Luego $k=k+1$,  $$1-4+9-16+...+(-1)^{k+2}\cdot (k+1)^2=(-1)^{k+2}\left[\dfrac{(k+1)(k+2)}{2}\right]$$.\\
Sumando $(-1)^{k+2}\cdot (k+1)^2$  a la segunda igualdad dada, se tiene,
$$1-4+9-16+...+ (-1)^{k+2}\cdot (k+1)^2 = (-1)^{k+1}\left(\dfrac{k(k+1)}{2}\right) + (-1)^{k+2}\cdot (k+1)^2$$ Por lo tanto, basta demostrar que $(-1)^{k+1}\left(\dfrac{k(k+1)}{2}\right) + (-1)^{k+2}\cdot (k+1)^2=(-1)^{k+2}\left[\dfrac{(k+1)(k+2)}{2}\right]$\\
\begin{center}
\begin{tabular}{r c l}
$(-1)^{k+1}\left(\dfrac{k(k+1)}{2}\right) + (-1)^{k+2}\cdot (k+1)^2$&$=$&$(-1)^{k+2}\left\lbrace \dfrac{[(-1)(k+1)k]+2(k^2+2k+1)}{2} \right\rbrace$\\\\
&$=$&$(-1)^{k+2} \left( \dfrac{-k^2 -k +2k^2 +4k +2}{2} \right)$\\\\
&$=$&$(-1)^{k+2} \left( \dfrac{k^2+3k+2}{2} \right)$\\\\
&$=$&$(-1)^{k+2} \left[ \dfrac{(x+1)(x+2)}{2} \right]$\\\\
\end{tabular}
\end{center}
\end{ej}

%ejercicio 5.3
\begin{ej}
Obsérvese que 
\begin{center}
\begin{tabular}{r c l}
$1+\frac{1}{2}$&=&$2-\frac{1}{2}$\\\\
$1+\frac{1}{2}+\frac{1}{4}$&=&$2-\frac{1}{4}$\\\\
$1+\frac{1}{2}+\frac{1}{4}+\frac{1}{8}$&=&$2-\frac{1}{8}$\\\\
\end{tabular}
\end{center}
Demostración.- \; Se verifica que $1+\frac{1}{2}+\frac{1}{4}+...+\dfrac{1}{2^n}=2-\dfrac{1}{2^n}$.\\\\
Para $n=k$ $$1+\frac{1}{2}+\frac{1}{4}+...+\dfrac{1}{2^k}=2-\dfrac{1}{2^k}$$\\
$k=1$ $$1+\dfrac{1}{2^1}=2-\dfrac{1}{2^1}$$\\
Luego $k=k+1$ $$1+\frac{1}{2}+\frac{1}{4}+...+\dfrac{1}{2^{k+1}}=2-\dfrac{1}{2^{k+1}}$$\\
Así solo falta demostrar que, $$2-\dfrac{1}{2^k}+\dfrac{1}{2^{k+1}}=2-\dfrac{1}{2^{k+1}}$$
\begin{center}
\begin{tabular}{r c l}
$2-\dfrac{1}{2^k}+\dfrac{1}{2^{k+1}}$&=&$2+\dfrac{-2+1}{2^{k+1}}$\\\\
&=&$2-\dfrac{1}{2^{k+1}}$\\\\
\end{tabular}
\end{center}
\end{ej}

%ejercicio 5.4
\begin{ej}
Obsérvese que
\begin{center}
\begin{tabular}{r c l}
$1-\frac{1}{2}$&=&$\frac{1}{2}$\\\\
$(1-\frac{1}{2})(1-\frac{1}{3})$&=&$\frac{1}{3}$\\\\
$(1-\frac{1}{2})(1-\frac{1}{3})(1-\frac{1}{4})$&=&$\frac{1}{4}$\\\\
\end{tabular}
\end{center}
Indúzcase la ley general y demuéstrese por inducción.\\\\
Demostración.- \; Se induce que $\left( 1-\dfrac{1}{2} \right)\left( 1-\dfrac{1}{3} \right)...\left(1- \dfrac{1}{n} \right)=\dfrac{1}{n}$ para todo $n>1$.\\
Sea $n=k$, entonces $\left( 1-\dfrac{1}{2} \right)\left( 1-\dfrac{1}{3} \right)...\left(1- \dfrac{1}{k} \right)=\dfrac{1}{k}$. Después para $k=2$, \; $1-\dfrac{1}{2}=\dfrac{1}{2}$. Si $k=k+1$ tenemos $\left( 1-\dfrac{1}{2} \right) \left( 1-\dfrac{1}{3} \right)...\left(1- \dfrac{1}{k+1} \right)=\dfrac{1}{k+1}$. Luego es fácil comprobar que 
$\dfrac{1}{k}\left(1 - \dfrac{1}{k+1} \right)=\dfrac{1}{k+1}$. \\\\
\end{ej}


%ejercicio 5.5.
\begin{ej}
Hallar la ley general que simplifica al producto $$\left( 1-\dfrac{1}{4} \right)\left( 1-\dfrac{1}{9} \right)\left( 1- \dfrac{1}{16} \right)..\left( 1- \dfrac{1}{n^2} \right)$$ y demuéstrese por inducción.\\\\
Demostración.- \; Inducimos que $\left( 1-\dfrac{1}{4} \right)\left( 1-\dfrac{1}{9} \right)\left( 1- \dfrac{1}{16} \right)..\left( 1- \dfrac{1}{n^2} \right)=\dfrac{n+1}{2n}$,para todo $n>1$. Después $n=k=1$, $$1-\dfrac{1}{2^2}=\dfrac{2+1}{2\dot 2} \Rightarrow \dfrac{3}{4}=\dfrac{3}{4}$$ 
Luego $k+1$, $$\left( 1-\dfrac{1}{4} \right)\left( 1-\dfrac{1}{9} \right)\left( 1- \dfrac{1}{16} \right)..\left( 1- \dfrac{1}{(k+1)^2} \right)=\dfrac{(k+1)+1}{2(k+1)}$$
Así,
\begin{center}
\begin{tabular}{r c l}
$\left( \dfrac{k+1}{2k}\right) \left( 1- \dfrac{1}{(k+1)^2} \right)$&=&$\dfrac{(k+1)+1}{2(k+1)}$\\\\
$\left( \dfrac{k+1}{2k} - \dfrac{1}{2k(k+1)} \right)$&=&$\dfrac{k+2}{2k+2}$\\\\
$\dfrac{(k+1)^2-1}{2k(k+1)}$&=&$\dfrac{k+2}{2k+2}$\\\\
$\dfrac{k+2}{2k+2}$&=&$\dfrac{k+2}{2k+2}$\\\\
\end{tabular}
\end{center}
\end{ej}

%ejercicio 5.6
\begin{ej}
Sea $A(n)$ la proporción: $1+2+...+n=\dfrac{1}{8} (2n+1)^2$.
\begin{enumerate}[\bfseries a)]
\item Probar que si $A(k)$, $A(k+1)$ también es cierta.\\\\
Demostración.- \; Para $A(k+1)$, 
\begin{center}
\begin{tabular}{r c l}
$\dfrac{1}{8}(2k+1)^2+(k+1)$&=&$\dfrac{1}{8}\left[2(k+1)+1\right]^2$\\\\
$\dfrac{4k^2+12k+9}{8}$&=&$\dfrac{4k^2+12k+9}{8}$\\\\
\end{tabular}
\end{center}
\item Critíquese la proposición $"$de la inducción se sigue que $A(n)$ es cierta para todo $n$ $"$.\\\\
Se ve que no se cumple para ningún entero $A(n)$ pero si para $A(n+1)$.\\\\ 
\item Transfórmese $A(n)$ cambiando la igualdad por una desigualdad que es cierta para todo entero positivo $n$\\\\
Primero comprobemos para $A(1)$, \; $1<\dfrac{9}{8}$.\\
Luego para $A(k),$ $$1+2+...+(k+1)<\dfrac{1}{8}(2k+1)^2$$
Después para $A(k+1)$ $$1+2+...+(k+1)<\dfrac{1}{8}(2k+1)^2$$
Remplazando $(k+1)$ a $A(k)$ $$1+2+...+(k+1)<\dfrac{1}{8}(2k+1)^2+(k+1)$$
por último solo nos queda demostrar $$\dfrac{1}{8}(2k+1)^2<\dfrac{1}{8}(2k+1)^2+(k+1)$$
Así $\dfrac{4k^2+12k+9}{8}<\dfrac{4k^2+12k+9}{8} +(k+1)$, vemos que la inecuación se cumple para cualquier número natural.\\\\ 
\end{enumerate} 
\end{ej}

%ejercicio 5.7
\begin{ej}
Sea $n_1$ el menor entero positivo $n$ para el que la desigualdad $(1+x)^n>1+nx+nx^2$ es cierta para todo $x>0$. Calcular $n_1$, y demostrar que la desigualdad es cierta para todos los enteros $n\geq n_1$\\\\
Demostración.- \; vemos que la proposición es validad para $n_1=3$, $$(1+x)^3>1+3x+3x^2,$$ y no así para $n=1$ \; y \;$n=2$ entonces $A(n)=A(k)\geq 3$, $(1+x)^k>1+kx+kx^2.$ Después para un $A(k+1)$, $(1+x)^{k+1}>1+(k+1)x+(k+1)x^2,$ así \; $(1+kx+kx^2)(1+k)>1+(k+1)x+(k+1)x^2,$ luego se cumple la desigualdad $x(kx^2)+x^2+kx+x+1+x^2>kx^2+x^2+kx+x+1$.\\\\
\end{ej}

%ejercicio 5.8
\begin{ej}
Dados números reales positivos $a_1,a_2,a_3,...,$ tales que $a_n\leq ca_{a-1}$ para todo $n\geq 2$. Donde $c$ es un número positivo fijo, aplíquese el método de inducción para demostrar que $a_n \leq a_1 c^{n-1}$ para cada $n \geq 1$\\\\
Demostración.- \; Primero, para el caso $n=1$, tenemos $a_1c^0=a_1$, por lo tanto la desigualdad es validad. Ahora supongamos que la desigualdad es válida para algún número entero $k$: $a_k\leq a_1c^{k-1}$, luego multiplicamos por $c$, $ca_k\leq a_1c^k$, pero dado que se asume por hipótesis $a_{k+1} \leq ca_k$, entonces $a_{k+1}\leq a_1c^k$, por lo tanto, la declaración es válida para todo $n$.\\\\
\end{ej}

%ejercicio 5.9
\begin{ej}
Demuéstrese por inducción la proposición siguiente: Dado un segmento de longitud unidad, el segmento de longitud $\sqrt{n}$ se puede construir con regla y compás para cada entero positivo $n$.\\\\
Demostración.- \; Dada una línea de longitud 1, podemos construir una línea de longitud $\sqrt{2}$ tomando la hipotenusa del triángulo rectángulo con patas de longitud 1.\\
Ahora, supongamos que tenemos una línea de longitud 1 y una línea de longitud $\sqrt{k}$ para algún número entero $k$. Luego podemos formar un triángulo rectángulo con patas de longitud 1 y longitud $\sqrt{k}$. La hipotenusa de este triángulo es $\sqrt{k+1}$. Por lo tanto, si podemos construir una línea de longitud $\sqrt{k}$, entonces podemos construir una línea de longitud $\sqrt{k+1}$. Como podemos construir una línea de longitud $\sqrt{2}$ en el caso base, podemos construir una línea de longitud $\sqrt{n}$ para todos los enteros $n$.\\\\
\end{ej}

%ejercicio 5.10
\begin{ej}
Sea $b$ un entero positivo. Demostrar por inducción la proposición siguiente: Para cada entero $n\geq 0$ existen enteros no negativos $q$ \; y \; $r$ tales que:
$$n=qb+r, \; \; \; 0\leq r < b$$ \\
Demostración.- \; Sea $b$ ser un entero positivo fijo. Si $n=0$ , luego $q=r=0$, la afirmación es verdadera (ya que $0=0b+0$).\\
Ahora suponga que la afirmación es cierta para algunos $k \in \mathbb{N}$. Por hipótesis de inducción sabemos que existen enteros no negativos $q$ \; y \; $r$ tales que $$k=qb+r, \; \; \; 0\leq r < b,$$ Por lo tanto, sumando 1 a ambos lados tenemos, $$k+1=qb+(r+1).$$ Pues $0\leq r < b$ entonces sabemos que $0\leq r \leq (b-1)$. Si $0\leq r < b-1,$ entonces $0\leq r+1<b,$ y la declaración aún se mantiene con la misma elección $q$ \; y \; $r+1$ en lugar de $r$.\\
Por otro lado, si $r=b-1,$ entonces $r+1=b$ y tenemos, $$k+1=qb+b=(q+1)b+0$$
Por lo tanto, la declaración se mantiene de nuevo, pero con $q+1$ en lugar de $q$ \; y con $r=0$ (que es válido ya que si $r=0$ tenemos $a\leq r < b$). Por ende, si el algoritmo de división es válido para $k$, entonces también es válido para $k+1.$ Entonces, es válido para todos $n \in \mathbb{N}$.\\\\
\end{ej}

%ejercicio 5.11
\begin{ej}
Explíquese el error en la siguiente demostración por inducción.\\
Proposición.- Dado un conjunto de n niñas rubias, si por 10 menos una de las niñas tiene ojos azules, entonces las n niñas tienen ojos azules.\\
Demostración.-\; La proposición es evidentemente cierta si $n = 1$. El paso de $k$ a $k + 1$ se puede ilustrar pasando de $n = 3$ a $n = 4$. Supóngase para ello que la proposición es cierta para $n=3$ Y sean $G_1, G_2, G_3, G_4$ cuatro niñas rubias tales que una de ellas, por lo menos, tenga ojos azules, por ejemplo, la $G_1$, Tomando $G_1,G_2, G_3,$ conjuntamente y haciendo uso de la proposición cierta para $n =3$, resulta que también $G_2$ y $G_3$ tienen ojos azules. Repitiendo el proceso con $G_1, G_2$ y $G_4,$ se encuentra igualmente que $G_4$ tiene ojos azules. Es decir, las cuatro tienen ojos azules. Un razonamiento análogo permite el paso de $k$ a $k + 1$ en general.\\
\textbf{Corolario.} Todas las niñas rubias tienen ojos azules.\\
Demostración.- \; Puesto que efectivamente existe una niña rubia con ojos azules, se puede aplicar el resultado precedente al conjunto formado por todas las niñas rubias.\\\\
Esta prueba supone que la afirmación es cierta n = 3, es decir, supone que si hay tres chicas rubias, una de las cuales tiene ojos azules, entonces todas tienen ojos azules. Claramente, esta es una suposición falsa.\\\\
\end{ej}



\subsection[Teoremas]{Teoremas \footnote{Calculus, Tom Apostol, pag. 54-58}}

% 4.10 problema 1 pag 54
\begin{ej}
Calcúlese los valores de los siguientes coeficientes binomiales:
\begin{enumerate}[\bfseries a)]
\item ${5 \choose 3}$
$${5 \choose 3} = \dfrac{5!}{3! (5-3)!} = \dfrac{5!}{3! 2!} = \dfrac{20}{2} = 10$$ \\

\item ${7 \choose 0}$
$${7 \choose 0} = \dfrac{7!}{0!(7-0)!} = \dfrac{7!}{7!} = 1$$ \\

\item ${7 \choose 1}$
$${7 \choose 1} = \dfrac{7!}{1!(7-1)!} = \dfrac{7!}{6!} = 7$$ \\

\item ${7 \choose 2}$
$${7 \choose 2} = \dfrac{7!}{2!(7-2)!} = \dfrac{7!}{2!5!} = \dfrac{42}{2} = 21$$ \\

\item ${17 \choose 14}$
$${17 \choose 14} = \dfrac{17!}{14!(17-14)!} = \dfrac{17!}{14!3!} = \dfrac{17 \cdot 16 \cdot 15}{3!} = \\dfrac{4080}{6} = 680$$\\

\item ${0\choose 0}$
$${0\choose 0}=\dfrac{0!}{0!(0-0)!}=\dfrac{0!}{0!0!} = 1$$\\\\
\end{enumerate}
\end{ej}
	
% 4.10 problema 2 pag 54
\begin{teo}Demostrar:
\begin{enumerate}[\bfseries a)]
\item Demostrar que: ${k \choose n} = {n \choose n-k}$\\\\
Demostración .- \; 

\item Sabiendo que ${n \choose 10} = {n \choose 7}$ calcular $n$\\\\
Demostración.- \; 

\item Sabiendo que ${14 \choose k} = {14 \choose k-4}$ calcular $k$\\\\
Demostración.- \; 

\item ¿Existe un $k$ tal que ${12 \choose k} = {12 \choose k-3 }$\\\\
Demostración.- \;
\end{enumerate}

\end{teo}

\subsection[Teoremas]{Teoremas\footnote{Calculus, Tom Apostol, pag. 45-46}}

%teorema 5.2
\begin{teo}[principio de buena ordenación]
Todo conjunto no vacío de enteros positivos contiene uno que es el menor \\\\
Demostración.- \; Sea $T$ una colección no vacía de enteros positivos. Queremos demostrar que $t_0$ tiene un número que es el menor, esto es, que hay en T un entero positivo t.; tal que $t_0\leq t$ para todo $t$ de $T$.\\
Supongamos que no fuera así. Demostraremos que esto nos conduce a una contradicción. El entero $1$ no puede pertenecer a $T$ (de otro modo él sería el menor número de $T$). Designemos con $S$ la colección de todos los enteros positivos $n$ tales que $n<t$ para todo $t$ de $T$. Por tanto $1$ pertenece a $S$ porque $1 < t$ para todo $t$ de $T$. Seguidamente, sea $k$ un entero positivo de $S$. Entonces $k < t$ para todo $t$ de $T$. Demostraremos que $k + 1$ también es de $S$. Si no fuera así, entonces para un cierto $t$, de $T$ tendríamos $t_1 \leq k+1$. Puesto que $T$ no posee número mínimo, hay un entero $t_2$ en $T$ tal que $t_2 < t_1$ Y por tanto $t_2 < k + 1$. Pero esto significa que $t_2 \leq k$, en contradicción con el hecho de que $k < t$ para todo $t$ de $T$. Por tanto $k + 1$ pertenece a $S$. Según el principio de inducción, $S$ contiene todos los enteros positivos. Puesto que $T$ es no vacío, existe un entero positivo $t$ en $T$. Pero este $t$ debe ser también de $S$ (ya que $S$ contiene todos los enteros positivos). De la definición de $S$ resulta que $t < t$, lo cual es absurdo. Por consiguiente, la hipótesis de que $T$ no posee un número mínimo nos lleva a una contradicción. Resulta pues que $T$ debe tener un número mínimo, y a su vez esto prueba que el principio de buena ordenación es una consecuencia del de inducción.\\\\
\end{teo}

\subsection[Ejercicios]{Ejercicios\footnote{Cálculo infinitesimal, Michael Spivak,Pag. 35 al 45}}
%ejercicio 
\begin{ej}
Demostrar por inducción la siguiente fórmula: $1^2+...+n^2=\dfrac{n(n+1)(2n+1)}{6}$\\\\
Demostración.- \; Sea $n=k$: $$1^2+...+k^2=\dfrac{k(k+1)(2k+1)}{6},$$ Para $k=1$, $$1^2=\dfrac{1(1+2)(2+1)}{6}$$ por lo tanto se cumple para $k=1$, Luego para $k=k+1$, $$1^2+...+(k+1)^2=\dfrac{(k+1)(k+2)(2k+3)}{6},$$ así cabe demostrar que:
\begin{center}
\begin{tabular}{r c l}
$\dfrac{k(k+1)(2k+1)}{6}+(k+1)^2$&=&$\dfrac{(k+1)(k+2)(2k+3)}{6}$\\\\
$\dfrac{2k^3+k^2+2k^2+k+6k^2+12k+6}{6}$&=&$\dfrac{2k^3+3k^2+6k^2+9k+4k+6}{6}$\\\\
$\dfrac{2k^3+9k^2+13k+6}{6}$&=&$\dfrac{2k^3+9k^2+13k+6}{6}$\\\\
\end{tabular}
\end{center}
\end{ej}

%ejercicio 
\begin{ej}
Encontrar una fórmula para 
\begin{enumerate}[\bfseries i)]
\item  $ \displaystyle\sum_{i=1}^{n} (2i-1) = 1 +3 +5 + ... + (2n-1)$\\
\begin{center}
\begin{tabular}{r c c c l}
1&=&1&=&$1^2$\\
1+3&=&4&=&$2^2$\\
1+3+5&=&9&=&$3^2$\\
1+3+5+7&=&16&=&$4^2$\\
1+3+5+7+9&=&25&=&$5^2$\\
\end{tabular}
\end{center}
Por lo tanto $ \displaystyle\sum_{i=1}^{n} (2i-1) = 1 +3 +5 + ... + (2n-1) = n^2$\\\\
\item $\displaystyle\sum_{i=1}^{n} (2i-1)^2 = 1^2 + 3^2 + 5^2 + ... + (2n-1)^2$
\begin{center}
\begin{tabular}{r c l l}
$1^2 + 3^2 + 5^2 + ... + (2n-1)^2$&$=$&$\left[ 1^2 +2^2 +...+(2n)^2 \right] - \left[ 2^2 + 4^2 +6^2 +...+ (2n)^2\right]$& verifique sustrayendo\\\\
&$=$&$\left[ 1^2 + 2^2 + ...+ (2n)^2 \right] - 4\left[ 1^2 + 2^2 +3^2 + ... + n^2 \right]$&\\\\
&$=$&$\dfrac{2n(2n+1)(4n+1)}{6} -\dfrac{4n(n+1)(2n+1)}{6}$&por ejercicio 5.13\\\\
&$=$&$\dfrac{2n(2n+1)\left[ 4n+1 -2 (n+1) \right]}{6}$&\\\\
&$=$&$\dfrac{n(2n+1)(2n-1)}{6}$&\\\\
\end{tabular}
\end{center}
\end{enumerate}
\end{ej}

\subsection[Teoremas]{Teoremas\footnote{Cálculo infinitesimal, Michael Spivak,Pag. 35 al 45}}

%problema 3 pag 35
\begin{teo}
Si $0 \leq k \leq n,$ se define el coeficiente binomial $ {n \choose k} $ por $${n \choose k} = \dfrac{n!}{k!(n-k)!}=\dfrac{n(n-1)...(n - k + 1)}{k!}, \; si \; k \neq 0, \; n$$ $${n \choose 0} = {n \choose n} = 1.$$ Esto se convierte en un caso particular de la priera fórula si se define $0! = 1.$
\begin{enumerate}[\bfseries a)]
\item Demostrar que $${n +1 \choose k} = {n \choose k - 1} + {n \choose k}$$ Esta relación de lugar a la siguiente configuración, conocida por triángulo de Pascal: Todo número que no esté sobre uno de los lados es la suma de los dos números que tiene encima: El coeficiente binomial ${n \choose k}$ es el número k-ésimo de la fila $(n+1)$.
\begin{center}
\begin{tabular}{ccccccccccc}
  &    &    &    &    &  1 &    &    &    &    &   \\
  &    &    &    &  1 &    &  1 &    &    &    &   \\
  &    &    &  1 &    &  2 &    &  1 &    &    &   \\
  &    &  1 &    &  3 &    &  3 &    &  1 &    &   \\
  &  1 &    &  4 &    &  6 &    &  4 &    &  1 &   \\
1 &    &  5 &    & 10 &    & 10 &    &  5 &    & 1 \\\\
\end{tabular}
\end{center}
Demostración.- \; \\
\begin{center}
\begin{tabular}{r c l}
$ {n \choose k-1}  +  {n \choose k} $&$=$&$\dfrac{n!}{(k-1)(n-k+1)!}+ \dfrac{n!}{k!(n-k)!}$\\\\
&$=$&$\dfrac{kn!}{k!(n+1-k)!} + \dfrac{(n+1-k)n!}{k!(n+1-k)!}$\\\\
&$=$&$\dfrac{(n+1)n!}{k!(n+1-k)!}$\\\\
&$=$&$ {n+1 \choose k} $\\\\
\end{tabular}
\end{center}
\item Obsérvese que todos los números del triángulo de Pascal son números naturales. Utilícese la parte $(a)$ para demostrar por inducción que $ {n \choose k}$ es siempre un número natural.\\\\
Demostración.- \; Se ve claramente que ${1 \choose 1}$ es un número natural. Supóngase que ${n \choose p}$ es un número natural para todo $p \leq n$. Al ser: $${ n+1 \choose p } = {n \choose p-1} + {n \choose p} \; para \; p \leq n,$$ se sigue que ${n+1 \choose p}$ es un número natural para todo $p \leq n,$ mientras que ${n+1 \choose n+1}$ es también un número natural. Así pues, ${n+1 \choose p}$ es un número natural para todo $p \leq n+1$\\\\

\item Dése otra demostración de que ${n \choose k}$ es un número natural, demostrando que ${n \choose k}$ es el número de conjuntos de exactamente $k$ enteros elegidos cada uno entre $1,...,n$.\\\\
Demostración.- \; Existen $n(n -1) \cdot ... \cdot (n - k + 1)$ k-tuplas de enteros distintos elegidos entre $1, ..., n$, ya que el primero puede ser elegido de n maneras, el segundo de $n - 1$ maneras, etc. Ahora bien, cada conjunto formado exactamente por $k$ enteros distintos, da lugar a $k!$ k-tuplas, de
modo que el número de conjuntos será $n(n — 1) \cdot ... \cdot (n - k + l)/k! = {n\choose k}$\\\\ 

\item Demostrar el \textbf{TEOREMA DEL BINOMIO}: Si $a$ \; y \; $b$ son números cualesquiera, entonces
$(a+b)^n = a^n + {n \choose 1} a^{n-1} b + {n \choose 2} a^{n-2} b^2 + ... + {n \choose n-1} a b^{n-1} + b^n = \displaystyle \sum_{j=0}^n {n \choose j} a^{n-j} b^j.$\\\\
El teorema del binomio resulta claro para $n=1.$ Supónganse que $$(a+b)^n = \displaystyle \sum_{j=0}^n {n \choose j} a^{n-j} b^j.$$
Entonces 
\begin{center}
\begin{tabular}{r c l l}
$(a+b)^{n+1}$&=&$(a+b)(a+b)^n$&\\\\
&=&$(a+b) \displaystyle \sum_{j=0}^n {n \choose j} a^{n-j} b^j$&\\\\
&=&$\displaystyle \sum_{j=0}^n {n \choose j} a^{n+1-j} b^j + \sum_{j=0}^{n} {n \choose j} a^{n-j} b^{j+1}$&\\\\
&=&$\displaystyle \sum_{j=0}^n {n \choose j} a^{n+1-j} b^j + \sum_{j=0}^{n+1} {n \choose j-1} a^{n+1-j} b^{j}$&sustituimos $j$ por $j-1$ en la $2^{da}$ suma\\\\
&=&$\displaystyle \sum_{j=0}^{n+1} {n+1 \choose j} a^{n+1-j} b^j$& por la parte $b)$\\\\
\end{tabular}
\end{center}
Según la parte $a)$, con lo que el teorema del binomio es válido para $n+1.$\\\\

\item Demostrar que 
\begin{enumerate}[\bfseries i)]
\item $\displaystyle\sum_{j=0}^{n} {n \choose j} = {n \choose 0} + ... + {n \choose n} = 2^n$\\\\
Demostración.- \; Por el teorema del binomio $2^n=(1+1)^n = \displaystyle\sum_{j=0}^n {n \choose j}(1^j)(1^{n-j})=\sum_{j=0}^n {n \choose j}$\\\\

\item $\displaystyle\sum_{j=0}^n (-1)^j {n \choose j} = {n \choose 0}- {n \choose 1}+...\pm {n \choose n} =0$\\\\
Demostración.- \;  De igual manera por el teorema del binomio $0=(1+(-1))^n = \displaystyle\sum_{j=0}^n(-1)^j {n \choose j}$\\\\ 

\item $\displaystyle\sum_{l \; impar} {n \choose l} = {n \choose 1} + {n \choose 3}+ ... = 2^{n-1}$\\\\
Demostración.- \;  {\color{green}Terminar demostración xxxxxxxxxxxxxxxxxxxxxxxxxxxxxxxxxxxxxxx}

\item $\displaystyle\sum_{l \; par} {n \choose l} = {n \choose 0} + {n \choose 2} + ...  = 2^{n-1}$\\\\
Demostración.- \; {\color{green}Terminar demostración xxxxxxxxxxxxxxxxxxxxxxxxxxxxxxxxxxxxxxx}
\end{enumerate}
\end{enumerate}
\end{teo}


%problema 3.4 pag 37
\begin{teo}Demostrar:
\begin{enumerate}[\bfseries a)]
\item Demostrar que $$\displaystyle\sum_{k=0}^{l} {n \choose k} {m \choose l-k} = {n+m \choose l}$$\\\\
Demostración.- \; Aclaremos la proposición primeramente con un ejemplo.\\
Tenemos $n$ hombres y $m$ mujeres, y queremos formar un conjunto de combinaciones $l$ de personas de la forma $n+m$. Claramente hay ${n+m \choose l}$ caminos para formar dichas combinaciones.\\
Contemos el número de combinaciones de otra manera. Tenemos $0$ hombres y $l$ mujeres. Tal combinación se pueden formar en ${n \choose 0 } {m \choose l}$ caminos, o también:
\begin{itemize}
\item Podemos tener $1$ hombre y $l-1$ mujeres. Tal combinación se puede formar como ${n \choose l}{m \choose l-1}$ maneras.
\item Ó podemos tener $2$ hombre y $l-2$ mujeres. Tal combinación se puede formar como ${n \choose 2}{m \choose l-2}$ maneras.
\end{itemize}
Así las combinaciones totales serian $$\displaystyle\sum_{k=0}^l {n \choose k}{m \choose l -k}$$
Pero ya vimos que el número de combinaciones es ${n+m \choose l}$ siempre y cuando se cumpla la condición ${a \choose b} = 0$ si $b>a$\\
Ahora si pasemos a demostrar. Sea por producto de Cauchy de dos series infinitas:
$$\left( \displaystyle\sum_{k=0}^{\infty} a_k x^k \right) \left( \displaystyle\sum_{k=0}^{\infty} b_k x^k \right) = \sum_{k=0}^{\infty} \left( \displaystyle\sum_{j=0}^{k} a_j x^j b_{k-j} x^{k-j}\right) = \sum_{k=0}^{\infty} \left( \displaystyle\sum_{j=0}^{k} a_j b_{k-j} \right) x^k$$ entonces $(1+x)^m = \displaystyle\sum_{k=0}^{\infty} {m \choose k}x^k$ y $(1+x)^n \displaystyle\sum_{k=0}^{\infty} {m \choose k}x^k$ entonces queda:
$$(1+x)^{m+n} = \sum_{k=0}^{\infty} {m+n \choose k} x^k$$
Se extiende los índices en las sumas a $\infty$  ya que $k>n$, ${n \choose k} = 0 $. Luego 
$$(1+x)^m (1+x)^n = \displaystyle\sum_{k=0}^{\infty} \left( \sum_{j=0}^k {m \choose j} {n \choose k-j} \right) x^k$$ Así 
$${m+n \choose k} = \displaystyle\sum_{j=0}^k {j \choose m}{n \choose k-j}$$\\\\


\item demostrar que $$\displaystyle\sum_{k=0}^{n} {n \choose k}^2 = {2n \choose n}$$\\\\
Demostración.- \; 
\end{enumerate}

\end{teo}