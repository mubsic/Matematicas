\chapter{Funciones}
\begin{tcolorbox}
\begin{def.}
Si $f$ \; y \; $g$ son dos funciones cualesquiera, podemos definir una nueva función $f+g$ denominada \textbf{suma} de $f+g$ mediante la ecuación:
$$(f+g)(x)=f(x)+g(x)$$
Para el conjunto de todos los $x$ que están a la vez en el dominio de $f$ y en el dominio de $g$, es decir: $$dominio \; (f+g)=dominio \; f \; \cap \; dominio \; g$$\\
\end{def.}
\begin{def.}
El dominio de $f \cdot g$ es $dominio \; f \cap \; dominio \; g$ $$(f \cdot g)(x)=f(x)\cdot g(x)$$\\
\end{def.}
\begin{def.} Se expresa por dominio $f$ $\cap$ dominio $g$ $\cap$ ${x:g(x)\neq 0}$
$$\left( \dfrac{f}{g}\right) (x)=\dfrac{f(x)}{g(x)}$$ \\
\end{def.}
\begin{def.}[Función constante]
$$(c \cdot g)(x)=c \cdot g(x)$$\\
\end{def.}
\end{tcolorbox}

%teorema 1
\begin{teo}
$(f+g)+h=f+(g+h)$\\\\
Demostración.- \; \textbf{La demostración es característica de casi todas las demostraciones que prueban que dos funciones son iguales: se debe hacer ver que las dos funciones tienen el mismo dominio y el mismo valor para cualquier número del dominio.} Observese que al interpretar la definición de cada lado se obtiene:
\begin{center}
\begin{tabular}{r c l}
$\left[ (f+g) + h \right](x)$&=&$(f+g)(x)+h(x)$\\
&=&$\left[ f(x) +g(x) \right] +h(x)$\\\\
&y&\\\\
$\left[ f+(g+h) \right](x)$&=&$f(x)+(g+h)(x)$\\
&=&$f(x)+\left[ g(x)+h(x) \right]$\\\\
\end{tabular}
\end{center}
Es esta demostración no se ha mencionado la igualdad de los dos dominios porque esta igualdad parece obvia desde el momento en que empezamos a escribir estas ecuaciones: el dominio de $(f+g)+h$ y el de $f+(g+h)$ es evidentemente dominio $f$ $\cap$ dominio $g$ $\cap$ dominio $h$. nosotros escribimos, naturalmente $f+g+h$ por $(f+g)+h=f+(g+h)$\\\\
\end{teo}

%teorema 2
\begin{teo}
Es igual fácil demostrar que $(f\cdot g)\cdot g=f\cdot (g \cdot h)$ y ésta función se designa por $f \cdot g \cdot h$. Las ecuaciones $f+g=g+f$ \; y \; $f\cdot g=g \cdot f$ no deben presentar ninguna dificultad.\\\\
\end{teo}

%teorema 3
\begin{teo}\footnote{Calculus Vol 1, Tom Apostol, pag. 66}
Dos funciones $f$ y $g$ so iguales si y sólo si
\begin{enumerate}[\bfseries a)]
\item $f$ y $g$ tienen el mismo dominio, y 
\item $f(x)=g(x)$ para todo $x$ del dominio de $f$.
\end{enumerate}
Demostración.- \; Sea $f$ una función tal que $\forall x \in D_f, \; \exists y \; / \; y=f(x)$ es decir $\left( x, f(x) \right)$ y $g$ una función tal que $\forall z \in D_g, \; \exists y \; / \; y=g(z)$ es decir $\left( z, g(z) \right)$, por definición 1.6 dos pares ordenados $\left( x, f(x) \right) = \left( z, g(z) \right)$ si y sólo si $x = z$ \; y \; $f(x)=g(z)$ \\\\    
\end{teo}


\begin{tcolorbox}
\begin{def.}[Composición de función]
$$(f \circ g)(x)=f(g(x))$$
El dominio de $f\circ g$ es $\lbrace $ $x$ : $x$ está en el dominio de $g$ \: y \; $g(x)$ está en el dominio de $f$ $\rbrace$
$$D_{f \circ g}= \lbrace x \; / \; x \in D_g \; \land \; g(x)\in D_f \rbrace$$
\end{def.}
\begin{prop}
$(f \circ g) \circ h = f \circ (g \circ h)$   La demostración es una trivalidad.
\end{prop}
\end{tcolorbox}

\begin{tcolorbox}
\begin{def.}\footnote{Definición de Tom Apostol}
Decimos que dos pares ordenados $(a,b)$ \; y \; $(c,d)$ son iguales si sólo si sus primeros elementos son iguales y sus segundos elementos son iguales.\\  $$(a,b)=(c,d) \; \; \mbox{si sólo si} \; \; a=c \; \; y  \; \; b=d$$
\end{def.}

\begin{def.}[Definición de función]\footnote{Definición de Tom Apostol}
Una función $f$ es un conjunto de pares ordenados $(x,y)$ ninguno de los cuales tiene el mismo primer elemento.\\
Por lo tanto, $$\forall x \in D_f, \exists y \; /\; (x,y)\in f$$ 
Esto es, para todo $x$ en el dominio de la $f$ existe exactamente un $y$ tal que $(x,y) \in f$
Es costumbre escribir $y=f(x)$ en lugar de $(x,y)\in f$, por lo tanto, $$\forall x \in D_f, \exists y \; / \; y=f(x)$$
\end{def.}
\end{tcolorbox}
\begin{tcolorbox}
\begin{def.} \footnote{Definición de Michael Spivak}
Una \textbf{función} es una colección de pares de números con la siguiente propiedad: Si $(a,b)$ \; y \; $(a,c)$ pertenecen ambos a la colección, entonces $b=c$; en otras palabras, la colección no debe contener dos pares distintos con el mismo primer elemento.\\
\end{def.}

\begin{def.} \footnote{Definición de Michael Spivak}
Si $f$ es una función, el \textbf{dominio} de $f$ es el conjunto de todos los $a$ para los que existe algún $b$ tal que $(a,b)$ está en $f$. Si $a$ está en el dominio de $f$, se sigue de la definición de función que existe, en efecto, un número $b$ único tal que $(a,b)$ está en $f$. Este $b$ único se designa por $f(a)$.\\  
\end{def.}
\end{tcolorbox}

\section{Teoremas y ejercicios}
\subsection[Ejercicios]{Ejercicios \footnote{Calculo infinitesimal, Micheal Spivak, Pag. 61 al 68}}
%ejercicio 1
\begin{ej}Sea $f(x)=\dfrac{1}{1+x},$ Interpretar lo siguiente:
\begin{enumerate}[\bfseries i)]
\item $f\left( f(x) \right)$\\\\
Vemos que el dominio de    $\dfrac{1}{1+x}$ son todos los reales excepto $-1$ ya que cualquier número dividido entre $0$ es indeterminado.\\
Por otro lado el dominio de $f\left( f(x) \right)$ es $f \left( \dfrac{1}{1 - x} \right) = \dfrac{1}{1 + \dfrac{1}{1+x}} = \dfrac{1+x}{2+x}, \; \; \; x \neq -2$\\\\
Así por definición de dominio $$D_{f \circ g} = \lbrace x \;  / \; x \neq -1 \; \land \; x \neq -2  \rbrace.$$\\

\item $f \left( \dfrac{1}{x} \right)$\\\\
El dominio de $f$ esta dada por $ (x\neq 0, \; -1)$ ya que $\dfrac{1}{0}$ es indeterminado. Como también $\dfrac{1}{1 + \dfrac{1}{x}} = \dfrac{x}{x+1}$\\\\ 

\item $f(cx)$\\\\
$f(cx)=\dfrac{1}{1+cx}$ por lo tanto $\dfrac{1}{c\left( \dfrac{1}{c}+x \right)}$ \; si \; $x\neq \dfrac{1}{c}, \; 0$\\\\

\item $f(x+y)$\\\\
$f(x+y)=\dfrac{1}{1+(x+y)}$ para $x+y \neq -1$\\\\

\item $f(x)+ f(y)$\\\\
$f(x)+ f(y) = \dfrac{1}{1+x} + \dfrac{1}{1+y} = \dfrac{x+y+2}{(1+x)(1+y)}$ para $x\neq -1$, $y\neq -1$\\\\

\item ¿Para que números $c$ existe un número $x$ tal que  $f(cx)=f(x)$?\\\\
Existe para todo $c$ ya que $f(c \cdot 0) = f(0)$ entonces $f(0)=1$\\\\

\item ¿Para que números $c$ se cumple que $f(cx)=f(x)$ para dos números distintos $x$?\\\\
Solamente para $c=1.$ Ya que $f(cx)=f(x)$ implica $x=cx$ y esto debe cumplirse por lo menos para un $x\neq 0$\\\\ 
\end{enumerate}
\end{ej}

%ejercicio 2
\begin{ej}
Sea $g(x)=x^2$ y sea 
\begin{equation}
h(x) = \left\lbrace
\begin{array}{rr}
0, & x \; racional\\
1, & x \; irracional
\end{array}
\right.
\end{equation}
\begin{enumerate}[\bfseries i)]
\item ¿Para cuáles $y$ es $h(y)\leq y$?\\\\
Vemos que $h(y)$ sólo puede ser $1$ ó $0$. Para que se cumpla la condición $h(y)\leq y$, \; debe ser $y\geq 0$ é $y$ racional ya que si $y$ es irracional, no se cumple la condición, debido a que si $y$ es irracional  entonces es $1$. También se cumpliría si $y\geq 1$ sea para $y$ racional o irracional.\\\\

\item ¿Para cuáles $y$ es $h(y) \leq g(y)$?\\\\
Se cumple para $y$ racional entre $-1,1$ inclusive y para todo $y$ tal que $|y| > 1$\\\\

\item ¿Qué es $g(h(z)) - h(z)$?\\\\
Sabemos que 
\begin{equation}
h(z) = \left\lbrace
\begin{array}{rl}
0, & z \; racional\\
1, & z \; irracional\\
\end{array}
\right.
\end{equation}
Ahora tenemos que $g(0) = 0^2 = 0$ \; y \; $g(1)=1^2=1$ por lo tanto 
\begin{equation}
g(h(z)) = \left\lbrace
\begin{array}{rl}
0^2, & z \; racional\\
1^2, & z \; irracional\\
\end{array}
\right.
\end{equation}
Y restando a $h(z)$ nos queda $0$.\\\\
\item ¿ Para cuáles $w$ es $g(w) \leq w$?\\\\
Sabemos que un número cualquiera elevado al cuadrado siempre dará un número positivo entonces el rango del dominio para que se cumple la condición $g(w) \leq w$ es $-1\geq w \geq 1$\\\\ 

\item ¿Para cuales cuáles $\epsilon$ es $g(g(\epsilon)) = g(\epsilon)$?\\\\
Solamente se cumple para $-1, \; 0 \; 1$\\\\
\end{enumerate}
\end{ej}

%ejercicio 3
\begin{ej}
Encontrar el dominio de las funciones definidas por las siguientes fórmulas:
\begin{enumerate}[\bfseries i)]
\item $f(x)=\sqrt{1-x^2}$\\\\
Por la propiedad de raíz cuadrada, se tiene  $1-x^2 \geq 0$ entonces $x^2 \leq 1$ por lo tanto el dominio son todos los $x$ tal que $|x| \leq 1$\\\\

\item $f(x)=\sqrt{1-\sqrt{1-x^2}}$\\\\
Se observa claramente que el dominio es $-1\leq x \leq 1$\\\\

\item $f(x)=\dfrac{1}{x-1} + \dfrac{1}{x-2}$\\\\
Operando un poco tenemos $$f(x)=\dfrac{2x-3}{(x-1)(x-2)},$$ sabemos que el denominador no puede ser $0$ por lo tanto el $D_{f} = \lbrace x\; / \; x \neq 1, \; x\neq  2 \rbrace$\\\\ 

\item $f(x)=\sqrt{1-x^2}+\sqrt{x^2-1}$\\\\
Claramente notamos que el dominio de $f$ es $[-1,1]$ ya que si se tomara un número mayor a este daría un número imaginario.\\\\

\item $f(x) = \sqrt{1-x} + \sqrt{x-2}$\\\\
Notamos que no cumple para ningún $x$ ya que si $0\leq x \leq 1$ entonces no se cumple para $\sqrt{x-2} $ y si $x\geq 2$ no se cumple para $\sqrt{1-x}$\\\\ 
\end{enumerate}
\end{ej}

%ejercicio 4
\begin{ej}
Sea $S(x)=x^2,$ $P(x)=2^x ,$ $s(x)=sen x$. Determinar los siguientes valores. En cada caso la solución debe ser un número.
\begin{enumerate}[\bfseries 1)]
\item $(S \circ P)(y)$\\\\
Demostración.- \; Por definición $S(P(y))$ por lo tanto $S(2^y) = (2^y)^2 = 2^{2y}$ para todo $x$ existe en los números reales\\\\

\item $(S \circ s)(y)$ \\\\
Demostración.- \; $(S \circ s)(y) = S(s(y)) = S(\sin y) = \sin^2 y$\\\\

\item $(S \circ P \circ s)(t) + (s \circ P)(t)$\\\\
Demostración.- \; $(S \circ P \circ s)(t) + (s \circ P)(t) = S(P(s(t))) + s(P(t)) = S(P(\sin t)) + s(2^x) = S(2^{\sin t}) + \sin 2^{t} = \left( 2^{sin t} \right)^2 + \sin 2^{t} = 2^{2 \sin t}+ \sin 2^t$\\\\

\item $s(t^3)$\\\\
Demostración.- \; $s(t^3) = \sin t^3$\\\\ 
\end{enumerate}
\end{ej}

%ejercicio 5
\begin{ej}
Expresar cada una de las siguientes funciones en términos de $S, \; P, \; s$ usando solamente $+, \; \cdot \; y \; \circ$ en cada caso la solución debe ser una función.\\\\
\begin{enumerate}[\bfseries i)]
\item $f(x) = 2^{\sin x} = (P \circ s)(x)$\\\\
\item $f(x)=\sin 2^x = (s \circ S)(x)$\\\\
\item $f(x) = \sin x^2 = (s \circ S)(x)$\\\\
\item $f(x) = \sin^2 x = (S \circ s)(x)$\\\\
\item $f(t) = 2^{2t} = (S \circ P)(t)$\\\\
\item $f(u) = \sin(2^u + 2^{u^{2}}) = s \circ ( P + P \circ S )$\\\\
\item $f(a) = 2^{\sin^2 a} + \sin(a^2) + 2^{\sin(a^2 + \sin a)} = P \circ S \circ s + P \circ s \circ S + P \circ s \circ (S + s)$\\\\
\end{enumerate}
\end{ej}

%ejercicio 6
\begin{ej} Las funciones polinómicas, por ser sencillas y al mismo tiempo flexibles, ocipan un lugar destacado en el estudio de las funciones. Los dos problemas siguientes ponen de manifiesto su flexibilidad y dan una orientación para deducir sus propiedades elementales más importantes.
\begin{enumerate}[\bfseries a)]
\item Si $x_1, ..., x_n$ son números distintos, encontrar una función polinómica, $f_i$ de grado $n-1$ que tome el valor $1$ en $x_i$ \, y \; 0 en $x_j$ para $j \neq i$.Indicación: El produto de todo los $(x-x_j)$ para $j\neq i$ es $0$ en $x_j$ si $j\neq i$. (Este producto es designado generalmente por)
\begin{center}
\[
\prod_{\overset{j=1}{j \neq i}}^{n} (x-x_j) 
\]
\end{center}
donde el símbolo $\Pi$ (pi mayúscula) desempeña para productos el mismo papel que $\sum$ para sumas).\\\\
Solución.- \; Lo que Spivak afirma es que entre los números $x_1, ..., x_n$ hay un sólo número $x_i$ en el que la función $f_i$ tome el valor $1$ y que todas los demás números $(x_j \; con \; j\neq i)$ son ceros en $f_i$.\\
Una forma de pensar sobre esta pregunta es considerar una solución fija $n$ y elegir un conjunto de distintas $x_1, x_2, ..., x_n$. Por ejemplo supongamos que elegimos $n=3$ $x_1=1$, $x_2=2$, $x_3 = 3.$ Entonces supongamos que queremos encontrar un polinomio $f_i(x_1)=f_1(1)=1,$ pero $f_1(x_2)=f_1(2)=f_1(3)=0.$ Es decir, $F_1$ es un cuadrático que tiene ceros en $x=2$ \; y\; $x=3$, pero es igual a $1$ en $x=1.$ Naturalmente, esto sugiere mirar un polinomio de la forma $$a(x-2)(x-3),$$ para que la igualdad sea igual a $1$ por alguna constante $a.$ Pero, ¿Qué es esta constante? Bueno, si nos conectamos con $x=1$, debemos tener $$f_1(1)=1=a(x-2)(x-3)=2a,$$ por lo tanto $a=1/2$ y la solución deseada es $$f_1(x)=\dfrac{1}{2}(x-2)(x-3).$$ Del mismo modo, si tratamos de encontrar un polinomio $f_2(x)$ tal que $f_2(2)=1$ con raíces en $x=1,3$ tendríamos que resolver la ecuación $1=a(2-1)(2-3),$ lo que da $a=-1$ por lo tanto $f_2(x)=-(x-1)(x-3)$\\
Ahora veamos el caso general. El polinomio $f_i(x)$ satisface $f_i(x_i)$ \; y \; $f_i(x_j)=0$ para todo $j \neq i$, entonces debe tomar la forma 
\[
f_i(x)=a \prod_{j \neq i} (x-x_j) 
\]
Para alguna constante $a$. Para encontrar esta constante, aplicamos $x=x_1$: 
\[
f_i(x_i)=1=a\prod_{j \neq i} (x_i-x_j), 
\]
por lo tanto:
\[
a= \dfrac{1}{\displaystyle\prod_{j \neq i} (x_i-x_j)} 
\]  
Así queda
\[
f_i(x)= \prod_{j \neq i} \dfrac{(x-x_j)}{(x_i-x_j)}  
\]\\\\
\item Encontrar ahora una función polinómica de grado $n-1$ tal que $f(x_i)=a_i$, donde $a_i,...,a_n$ son números dados. (Utilícese las funciones $f_i$ de la parte $(a)$. La fórmula que se obtenga es la llamada formula de interpolación de Lagrange).\\\\
Entonces \[ f(x) = \sum_{j=1} a_i f_i(x) \]
por lo tanto  \[ f(x) = \sum_{j=1} a_i \prod_{j \neq i} \dfrac{(x-x_j)}{(x_i-x_j)} \]\\\\ 
\end{enumerate}
\end{ej}

%ejercicio 8 pag 64
¿Para qué números a, b, c y d la función $f(x) = \dfrac{ax+b}{cx+b}$ satisface $f(f(x))=x$ para todo $x$? (¿Para qué números dicha ecuación tiene sentido?)\\\\
Si para todo $x$, $$x = f(f(x)) = \dfrac{a \left( \dfrac{ax+b}{cx+d} \right) +b}{c \left( \dfrac{ax+b}{cx+d} \right)+d}$$
entonces $\dfrac{a^2 x +ab +bcx +bd}{acx +cb +cdx +d^2} - x = 0$, para luego $(ac+cd)x^2 + (d^2 - a^2)x -ab -db = 0$ para todo $x$, de modo que:
\begin{center}
\begin{tabular}{r c l}
$ac+cd$&$=$&$0$\\
$ab+bd$&$=$&$0$\\
$d^2 -a^2$&$=$&$0$\\
\end{tabular}
\end{center}
Así por propiedades de raíz $a=d$ ó $a =-d$. Una posibilidad es que $a=d=0$ el cual se tiene $f(x) = \dfrac{b}{cx}$, donde nos queda que $f(f(x)) = \dfrac{b}{\dfrac{cb}{cx}} = x$, para todo $x\neq 0$. Si $a=d=\neq 0$, entonces $b=c=0$ con lo que también se tiene $f(x)=x$. Y la tercera posibilidad es que $a+d = 0$, de modo que $f(x)= \dfrac{ax+b}{cx-a},$ la cual satisface $f(f(x))=x$ para todo $x\neq \dfrac{a}{c}$. Estrictamente hablando, podemos añadir la condición $f(x)\neq \dfrac{a}{c}$ para $x\neq \dfrac{a}{c}$, lo que significa que $$\dfrac{ax+b}{cx-a} \neq \frac{a}{c}$$ ó $a^2+bc\neq 0$\\\\

%ejercicio 10 pag 64
\begin{ej} Resolver: 
\begin{enumerate}[\bfseries a)]
\item ¿Para qué funciones $f$ existe una función $g$ tal que $f=g^2$?\\\\
Debido a que algún número elevado al cuadrado siempre será no negativo podemos afirmar que las funciones $f$ satisfacen a todo $x$ tal que $f(x)\geq 0$\\\\

\item ¿Para qué función $f$ existe una función $g$ tal que $f=\dfrac{1}{g}$?
Dado a que un número divido entre cero es indeterminado se ve claramente que satisfacen a todo $x$ tal que $f(x)\neq 0$\\\\

\item ¿Para qué funciones $b$ \; y \; $c$ podemos encontrar una función $x$ tal que $$(x(t))^2 + b(t)x(t) + c(t) =0$$ para todos los números $t$?\\\\
Por teorema se observa que para las funciones $b$ \; y \; $c$ que satisfacen $(b(t))^2 - 4c(t) \geq 0$ para todo $t$\\\\ 

\item ¿Qué condiciones deben satisfacer las funciones $a$ \, y \; $b$ si ha de existir ua función $x$ tal que $$a(t)x(t)+ b(t) = 0$$ para todos los números $t$? ¿Cuántas funciones $x$ de éstas existirán?\\\\
Es facil notar que $b(t)$ tiene que ser igual a $0$ siempre que $a(t) = 0$. Si $a(t) \neq 0$ para todo $t$, entonces existe una función única con esta condición, que es $x(t) = a(t)/b(t)$. Si $a(t)=0$ para algún $t$, entonces puede elegirse arbitrariamente $x(t)$, de modo que existen infinitas funciones que satisfacen la condición.\\\\

%ejercicio 11 pag 64 y 65
\begin{ej} Supóngase:
\begin{enumerate}[\bfseries a)]
\item Supóngase que $H$ es una función é \; $y$ \; un número tal que $H(H(y)) = y$ ¿Cuál es el valor de $H(H(H(...(H(y)...)$ $80$ veces?\\\\
Si aplicamos la hipótesis, tendremos que aplicar $78$ veces la función, luego $76$ y así, hasta llegar a $2$, donde la función sera $H(H(y))$, y una vez más por hipótesis tenemos como resultado $y$.\\\\
\item  La misma pregunta sustituyendo 80 por 81\\\\
Sea $H(H(y))$ la $78ava$ vez de la función, entonces la $81ava$ vez será $H(H(H(y))$, por lo tanto queda como resultado $H(y)$.\\\\ 

\item La misma pregunta si $H(H(y))=H(y)$\\\\
Análogamente a la parte $a)$ si la $80ava$ vez es $y$ entonces por hipótesis nos queda $H(y)$.\\\\ 

\item Encuéntrese una función $H$ tal que $H(H(x))=H(x)$ para todos los números $x$ \; y tal que $H(1)=36,$ $H(2)=\dfrac{\pi}{3}$, $H(13)=47,$ $H(36)=36$, $H(\pi / 3)=\pi / 3$, $H(47)=47$\\\\
Dar a $H(l)$, $H(2)$, $H(13)$, $H(36)$, $H(\pi /3)$, y $H(47)$ los valores especificados y hágase $H(x) = 0$ para $x \neq 1, 2, 13, 36, \pi /3, 47.$ Al ser, en particular, $H(0) = 0$, la condición $H(H(x)) = H(x)$ se cumple para todo $x$.\\\\

\item Encontrar una función $H$ tal que $H(H(x))=H(x)$ para todo $x$ \; y tal que $H(1)=7$, $H(17)=18$\\\\
Hágase $H(1) = 7$, $H(7) = 7$, $H(17) = 18$, $H(18) = 18$, y $H(x) = 0$ para $x \neq l , 7, 17, 18$.\\\\
\end{enumerate}
\end{ej}
\end{enumerate}
\end{ej}

%ejercicio 12 
\begin{ej}
Una función $f$ es par si $f(x)=f(-x)$ e impar si $f(x)=-f(-x)$. Por ejemplo, $f$ es par si $f(x)=x^2$ ó $f(x)=|x|$ ó $f(x)=cos\; x$, mientas que $f$ es impar si $f(x)=x$ ó $f(x)= sen \; x.$.
\begin{multicols}{2}
\begin{center}
\begin{tikzpicture}[scale=0.5, draw opacity = .6]
% abscisa y ordenada
\tkzInit[xmax= 7,xmin=-6,ymax=10,ymin=-2]
\tiny\tkzLabelXY[opacity=0.6,step=1, orig=false]
% label x, f(x)
\tkzDrawX[opacity= .6,label=x,right=0.3]
\tkzDrawY[opacity= .6,label=f(x),below = -0.6]
%dominio y función
\tkzFct[opacity=1,domain = -7:8]{\x**2} 
\tkzText[above,opacity=0.6](3,10){\tiny $f(x)=x^2$}
\tkzFct[opacity=1,domain = -6:6]{abs(\x)} 
\tkzText[above,opacity=0.6](6,6){\tiny $f(x)=|x|$}
\tkzFct[opacity=1,domain = -6:6]{cos(\x)} 
\tkzText[above,opacity=0.6](6,1){\tiny $f(x)=cos \; x$}
\end{tikzpicture}\\
$f$ es par
\end{center}

\begin{center}
\begin{tikzpicture}[scale=0.5, draw opacity = .6]
% abscisa y ordenada
\tkzInit[xmax= 7,xmin=-6,ymax=7,ymin=-5]
\tiny\tkzLabelXY[opacity=0.6,step=1, orig=false]
% label x, f(x)
\tkzDrawX[opacity= .6,label=x,right=0.3]
\tkzDrawY[opacity= .6,label=f(x),below = -0.6]
%dominio y función
\tkzFct[opacity=1,domain = -6:6]{\x} 
\tkzText[above,opacity=0.6](6,6){\tiny $f(x)=x$}
\tkzFct[opacity=1,domain = -6:6]{sin(\x)} 
\tkzText[above,opacity=0.6](6,1){\tiny $f(x)=sen \; x$}
\end{tikzpicture}\\
$f$ es impar
\end{center}
\end{multicols}

\begin{enumerate}[\bfseries a)]
\item Determine si $f+g$ es par, impar o ninguna de las dos, en los cuatro casos obtenidos eligiendo $f$ par o impar, y $g$ par o impar. (Las respuestas se pueden disponer de manera conveniente en una tabla $2x2$.)\\\\
Sea $f(x)=x^2$ y $g(x)=|x|$ entonces $f(-x)+g(-x)=(-x)^2 + |-x| = x^2 + |x| = f(x) + g(x)$ por lo tanto par y par es par.\\
Sea $f(x) = x$ y $g(x)=x$ entonces $-f(-x) + (-g(-x)) = -(-x) + [-x(-x)] = x + x = f(x) + g(x)$, por lo tanto impar e impar es impar.\\
Los otros dos últimos se prueba fácilmente y se llega a la conclusión de que ni uno ni lo otro.
\begin{center}
\begin{tabular}{c|cc}
&\textbf{Par}&\textbf{Par}\\
\hline\\
\textbf{Par}&Par&Ninguno\\\\
\textbf{Par}&Ninguno&Par\\
\end{tabular}
\end{center}
\item La misma cuestión para $f \cdot g.$\\\\
Sea $f(x)=x^2$ \; y \; $g(x)=|x|$, entonces $f(-x) \cdot g(-x) = x^2 \cdot |x| = f(x) \cdot g(x)$, por lo tanto se cumple para par y par.\\
Sea $f(x)=x$ \; y \; $g(x)=x$, entonces $-f(-x) \cdot -g(-x) = -(-x) \cdot -(-x) = x \cdot x = f(x) \cdot g(x)$, por lo tanto impar impar da impar\\
Sea $f(x)=x^2$ \; y \; $g(x)=x$, podemos crear otra función llamada $h$ que contiene a $x^2 \cdot x$ por lo tanto $h(x)=x^3 = -(-x)^2$ y así demostramos que par e impar es impar.\\
De igual forma al anterior se puede probar que impar y par es impar.
\begin{center}
\begin{tabular}{c|cc}
&\textbf{Par}&\textbf{Par}\\
\hline\\
\textbf{Par}&Par&Impar\\\\
\textbf{Par}&Impar&Par\\\\
\end{tabular}
\end{center}
\item La misma cuestión para $f \circ g.$
Sea $f(x)=x$ \; y \; $g(x)=x$, luego $h(x)=(f \circ g)(x)$ entonces $h(x) = x$ luego $-f(-x) = x$, por lo tanto impar e impar da impar.\\
De similar manera se puede encontrar para los demás problemas y queda:
\begin{center}
\begin{tabular}{c|cc}
&\textbf{Par}&\textbf{Par}\\
\hline\\
\textbf{Par}&Par&Par\\\\
\textbf{Par}&Par&Impar\\\\
\end{tabular}
\end{center}
\item Demuestre que cada función par $f$ puede escribirse como $f(x)=g(|x|),$ para infinitas funciones $g$.\\\\ 
Demostración.- \; Sea $f(x)=f(x)$ sabemos que $f$ es par si $f(x)=f(-x)$, y por definición de valor absoluto se tiene $f$ es igual a $f|x|$, por lo tanto $f(x)=g(|x|)$\\\\
\end{enumerate}
\end{ej}

%problema 14 pag 65
\begin{ej}
Si $f$ es una función cualquiera, definir una nueva función $|f|$ mediante $|f|(x)=|f(x)|$. Si $f$ \; y \; $g$ son funciones, definir dos nuevas funciones, $max(f,g)$ y $min(f,g)$, mediante $$max(f,g)(x)=max(f(x),g(x)),$$ $$min(f,g)(x)=min(f(x),g(x))$$
Encontrar una expresión para $max(f,g)$ y $min(f,g)$ en términos de $|\; |$.\\\\
Por definición de $max$ y $min$ visto anteriormente en el primer capítulo se tiene: $$max(f,g) = \dfrac{f+g + |f-g|}{2} \; \; \; y \; \; \; min(f,g) \dfrac{f+g - |f-g|}{2}$$ 
\end{ej} 

\subsection[Demostraciones]{Demostraciones \footnote{Calculo infinitesimal, Michael Spivak, pag. 63-68}}
%teorema (7)a
\begin{teo}
Demostrar que para cualquier función polinómica $f$ y cualquier número $a$ existe una función polinómica $g$ \; y un número $b$ tales que $f(x)=(x-a) \; g(x)+b$ para todo $x$. \\\\
Si el grado de $f$ es 1, entones $f$ es de la forma: $$f(x)=cx+d=(x-a)c + (d+ac)$$
de modo que podemos poner $g(x)=c$ \; y \;$b=d+ac$. Supóngase que el resultado es válido para polinomios de grado $\leq k.$ Si $f$ tiene grado $k+1$, entonces $f$ tiene la forma: $$f(x) = a_{k+1} (x-a) = (x-a) \; g(x) + b,$$ ó  $$f(x) = (x-a)[g(x) + a_{k+1}] + b,$$ con lo que tenemos la forma requerida.
{\color{green} Completar demostración xxxxxxxxxxxxxxxxxxxxxxxxxxxxxxxxxxxxxxxxxxxxxxxxxxxxxxxx}
\end{teo}

%teorema (7)b
\begin{teo}
Demostrar que si $f(a)=0$, entonces $f(x)=(x-a)\, g(x)$ para alguna función polinómica $g$. (La recíproca es evidente)\\\\
Demostración.- \; Por el teorema anterior, $f(x)=(x-a)\; g(x) + b.$ Luego $$0=f(a)=(a-a) \; g(a) + b = b,$$
de modo que $f(x) = (x-a) \; g(x)$\\\\
\end{teo}

%teorema (7)c
\begin{teo}
Demostrar que si $f$ es una función polinómica de grado $n$ entonces $f$ tiene a lo sumo $n$ raíces, es decir, existen a lo sumo $n$ números $a$ tales que $f(a)=0$\\\\
Demostración.- \; Supongámos que $f$ tiene $n$ raíces $a_1,...,a_n$. Entonces según el anterior teorema $f(x) = (x-a)\; g_1(x)$ donde el grado de $g_1(x)$ es $n-1.$ Pero $$0=f(a_2) = (a_2 - a_1) \; g_1(a_2)$$
de modo que $g_1(a_2) = 0$, ya que $a_2 \neq a_1.$ Luego podemos escribir $$f(x)=(x-a_1)(x-a_2)g_2(x)$$ donde el grado de $g_2$ es $n-2.$ Prosigue de esta manera, obtenemos que $$f(x)=(x-a_1)(x-a_2) \cdot ... \cdot (x-a_n)c$$ para algún número $c\neq 0.$ Está claro que $f(a) \neq 0$ si $a\neq a_1,..., a_n.$ Así pues, $f$ puede tener a lo sumo $n$ raíces.\\\\
\end{teo}

%teorema (7)d
\begin{teo}
Demostrar que para todo $n$ existe una función polinómica de grado $n$ con raíces. Si $n$ es par, encontrar una función polinómica de grado $n$ sin raíces y sin $n$ es impar, encontrar una con una sola raíz.\\\\
Demostración.- \; Si $f(x) = (x-1)(x-2) \cdot ... \cdot (x-n),$ entonces $f$ tiene $n$ raíces. Si $n$ es par, entonces $f(x) = x^n + 1$ no tiene raíces. Si $n$ es impar, entonces $f(x)=x^n$ tiene una raíz única, que es $0.$\\\\ 
\end{teo}

%teorema 9 pag 64
\begin{teo}
\begin{enumerate}[\bfseries a)]
\item Si $A$ es un conjunto cualquiera de números reales, defina una función $C_A$ de la manera siguiente:
\begin{equation}
C_A(x) = \left\lbrace
\begin{array}{rr}
1, & x \; si \; x \; está \; en \; A\\
0, & x \; si \; x \; no \; está \; en \; A
\end{array}
\right.
\end{equation}
Encuentre expresiones para $C_{A\ cap B},$   $C_{A\cup B}$ \; y \; $C_{R-A}$, en términos de $C_A$ \; y \; $C_B$ (El símbolo $A \cap B$ se ha definido en este capítulo, pero los otros dos pueden ser nuevos para el lector. Pueden definirse de la siguiente manera:
$$A \cap B = \lbrace x: \; x \; \mbox{pertenece a} \; A \, o \; \mbox{pertenece a} \; B \rbrace$$
$$R - B = \lbrace x: \: x \; \mbox{pertenece a} \; R \; pero \; x \; \mbox{no pertenece a} \; A \rbrace$$
\begin{center}
\begin{tabular}{r c l}
$C_{A \cap B}$&$=$&$C_A \cdot C_B$\\
$C_{A \cup B}$&$=$&$1 - C_A$\\
$C_{A \cup B}$&$=$&$C_A + C_B - C_A \cdot C_B$\\\\
\end{tabular}
\end{center}

\item Suponga que $f$ es una función tal que $f(x) = 0$ ó $1$ para todo $x$. Demuestre que existe un
conjunto $A$ tal que $f = C_A$\\\\
Demostración.- \;  Sea $x \in A$, por definición $C_A = 1$ entonces $f(x)=C_A$, por lo tanto existe un elemento en el conjunto $A$ tal que $f=C_A$\\\\

\item  Demuestre que $f = f^2$ si y sólo si f$ = C_A$ para algún conjunto $A$\\\\
Demostración.- \; Sea $x\in A$ por definición de la parte $a)$ se tiene $C_A(x)=1$ el cual satisface $f(x)=C_A(x)$, esto por hipótesis.\\\\

\item Demostrar que $f=f^2$ si y sólo si $f=C_A$ para algún conjunto $A$\\\\
Demostración.- \; Sea $x \in A$ por $b)$ $f=C_A$ si solo si $f=f^2$ ya que $1=1^2$\\\\
\end{enumerate}
\end{teo}

%teorema problema 13 pag 65
\begin{teo} Demostrar:
\begin{enumerate}[\bfseries a)]
\item Demostrar que cualquier función $f$ con dominio $\mathbb{R}$ puede ser puesta en la forma $f=E+O$, con $E$ par y $O$ impar.\\\\
Demostración.- \;  Despejamos $E$ y $O$ de la parte $b)$ así nos que da, $$E(x)=\dfrac{f(x) + f(-x)}{2}, \; \; \; \dfrac{f(x)-f(-x)}{2}.$$\\\\
\item Demuéstrese que esta manera de expresar $f$ es única. (Si se intenta resolver primero la parte $b)$ despejando $E$ \; y \; $O$, se encontrará probablemente la solución de la parte $a)$).\\\\
Demostración.- \; Si $f=E+O$, siendo $E$ par y $O$ impar, entonces $f(x)=E(x)+O(x),$ $f(-x)=E(x) - O(x)$. Por lo tanto es única para $f$\\\\
\end{enumerate}
\end{teo} 

%problema 3.15 pag 65
\begin{teo} Demostrar.
\begin{enumerate}[\bfseries a)]
\item Demostrar que $f=max(f,0)+min(f,0)$. Esta manera particular de escribir $f$ es bastante usada; las funciones $max(f,0)$ y $min(f,0)$ se llaman respectivamente \textbf{parte positiva} y \textbf{parte negativa} de $f$.\\\\
Demostración .- \; Sea $f=max(f,0)+min(f,0)$, ya que $f(x) = max(f(x),0)+ min(f(x),0)$ para todo $x$, al satisfacer la ecuación $c = max(c,0)+ min(c,0)$ para cualquier valor de $c$.\\\\
\item Una función $f$ se dice que es \textbf{no negativa} si $f(x)\geq 0$ para todo $x$. Demostrar que para cualquier función $f$ puede ponerse $f=g-h$ de infinitas maneras con $g$ \; y \; $h$ no negativas. (La manera corrientes es $g=max(f,0)$4 \; y \; $h=-min(f,0)$). Indicación: Cualquier número puede ciertamente expresarse de infinitas maneras como diferencia de dos números no negativos.\\\\
Demostración.- \; Para cada $x$, elíjase números $g(x)$, $h(x) \geq 0$ con $f(x) = g(x) - h(x).$ Puesto que cada uno de los pares formados por $g(x)$ \; y \; $h(x)$, puede ser elegido de infinitas maneras, existirán infinitos pares de funciones $g$ \; y \; $h$ que realizarán la descomposición pedida de $f.$\\\\
\end{enumerate}
\end{teo}

% Capitulo 3 problema 16 pag 66
\begin{teo}
Supongase que $f$ satisface $f(x+y) = f(x) + f(y)$ para todo $x$ \; e \; $y$.
\begin{enumerate}[\bfseries a)]
\item Demostrar que $f(x_1 +...+ x_n) = f(x_1) + ... + f(x_n).$\\\\
Demostración.- \; Vemos fácilmente que el resultado se cumple para $n=1$. Si $f(x_1 + ... + x_n) = f(x_1)+...+f(x_n)$ para todo $x_1,...,-x_n.$ Luego,
\begin{center}
\begin{tabular}{r c l}
$f(x_1 + ... + x_{n+1})$&=&$f\left( [x_1 + ... + x_n] + x_{x+1} \right)$\\
&=&$f(x_1 + ... + x_n) + f(x_{n+1})$\\
&=&$f(x_1) +...+ f(x_n) + f(x_{x+1})$\\\\
\end{tabular}
\end{center}

\item Demostrar que existe algún número $c$ tal que $f(x) = cx$ para todos los números racionales $x$ (en este punto no intentamos decir nada acerca de $f(x)$ cuando $x$ es irracional). Indicación: Piensese primero en cómo debe ser $c.$ Demostrar luego que $f(x) = cx,$ primero cuando $x$ es un entero, después cuando $x$ es el recíproco de un entero, y finalmente para todo racional $x$.\\\\
Demostración.- \; Sea $c = f(1).$ Ahora para cualquier numero natural $n.$ $$f(n) = f(1+...+1) = f(1) + ... + f(1) = cn$$
Ya que $f(x) + f(0) = f(x+0) = f(x),$ se sigue que $f(0) = 0$. Entonces desde $f(x) + f(-x) = f(x + (-x)) = f(0) = 0,$ se sigue que $f(-x) = f(x).$ En particular, para cualquier número natural $n.$ Además $$f \left( \dfrac{1}{n} \right) + ... + f(\dfrac{1}{n}) = \left( \dfrac{1}{n} + ... + \dfrac{1}{n} \right) = f(1) = c,$$
Así $$f\left( \dfrac{1}{n} \right) = c \cdot \dfrac{1}{n},$$
 y en consecuencia $$f\left( \dfrac{1}{-n} \right) = f \left( - \dfrac{1}{n} \right) = -f \left( \dfrac{1}{n} \right) = - c \dfrac{1}{n} = c \left(\dfrac{1}{-n} \right)$$
 Finalmente, algún número racional puede ser escrito como $m/n$,  $m$
 un número natural, y $n$ un entero, por lo tanto $$f\left( \dfrac{m}{n} \right) = f\left( \dfrac{1}{n} + ... + \dfrac{1}{n} \right) = f\left( \dfrac{1}{n} \right) + ... + f\left( \dfrac{1}{n} \right) = mc \dfrac{1}{n} = c \cdot \dfrac{m}{n}$$\\\\
\end{enumerate}
\end{teo}

%capítulo 3 problema 17
\begin{teo}
Si $f(x)=0$ para todo $x$, entonces $f$ satisface $f(x+y) = f(x) +f(y)$ para todo $x$ é $y$, también $f(x \cdot y) = f(x) \cdot f(y)$ para todo $x$ é $y$. Supóngase ahora que $f$ satisface estas dos propiedades, pero que $f(x)$ no es siempre $0$. Demostrar que $f(x)=x$ para todo $x$ como sigue:
\begin{enumerate}[\bfseries a)]
\item Demostrar que $f(1) = 1$\\\\
Demostración.- \; Como $f(a)=f(a \cdot 1)=f(a) \cdot f(1)$, tal que $f(a) \neq 0$ para algún $a$, resulta ser $f(1)=1$\\\\

\item Demostrar que $f(x) = x$ si $x$ es racional. \\\\
Demostración.- \; Según el teorema anterior, $f(x) = f(1)x = x$ para todo número racional $x$\\\\

\item Demostrar que $f(x)> 0$ si $x>0.$ (Esta parte es artificioso, pero habiendo puesto atención a las observaciones filosóficas que van con los problemas de los dos últimos capítulos, se sabrá lo que hacer.)\\\\
Demostración.- \; 

\item Demostrar que $f(x)>f(y)$ si $x>y$\\\\
Demostración.- \; 

\item Demostrar que $f(x)=x$ para todo $x$. Indicación: Hágase uso del hecho de que entre dos números cualesquiera existe un número racional.\\\\
Demostración.- \; 
\end{enumerate}
\end{teo}


\subsection[Ejercicios]{Ejercicios \footnote{Calculus Vol 1, Tom Apostol, pag 69-70}}
%ejercicio 1
\begin{ej}
Sea $f(x)=x+1$ para todo real $x$. Calcular:
\begin{itemize}
\item $f(2) = 2+1 = 3$\\\\
\item $f(-2) = -2 +1 = -1$\\\\
\item $-f(2) = -(2+1)=-3$\\\\
\item $f \left( \dfrac{1}{2} \right) = \dfrac{1}{2} + 1 = \dfrac{3}{2}$\\\\
\item $\dfrac{1}{f(2)}= \dfrac{1}{3}$\\\\
\item $f(a+b) = a+b+1$\\\\
\item $f(a)+f(b)= (a+1) + (b+1) = a+b+2$\\\\
\item $f(a) \cdot f(b) = (a+1)(b+1) = ab + a + b + 1$\\\\
\end{itemize}
\end{ej}

%ejercico 2
\begin{ej}
Sean $f(x)= 1+x$ \; y \; $g(x)=1-x$ para todo real $x$. calcular:
\begin{itemize}
\item $f(2)+g(2) = (1+2) + (1-2) = 2$\\\\
\item $f(2)-g(2) = (1+2) - (1-2) = 3$\\\\
\item $f(2)\cdot g(2) = (1+2) \cdot (1-2) = 3 \cdot (-1) = -3$\\\\
\item $\dfrac{f(2)}{g(2)}= \dfrac{1+2}{1-2} = \dfrac{3}{-1} = -3$\\\\
\item $f\left[ g(2)\right] = f(1-2) = f(-1) = 1+(-1)= 0$\\\\
\item $g\left[ f(2)\right] = f(1+2) = g(3) = 1 - 3 = -2$\\\\
\item $f(a) + g(-a) = (1+a) + (1 - a) = 2$\\\\
\item $f(t)\cdot g(-t) = (1+t) \cdot (1+t) = 1 + t + t + t^2 = t^2 +2t + 1 = (t+1)^2$\\\\
\end{itemize}
\end{ej}

%ejercicio 3
\begin{ej}
Sea $f(x)=|x-3|+|x-1|$ para todo real $x$. Calcular:\\
\begin{itemize}
\item $f(0) = |0-3|+|0-1| = 3 + 1 = 4$
\item $f(1) = |1-3|+|1-1| = 2$
\item $f(2) = |2-3|+|2-1| = -1 + 1 = 0$
\item $f(3) = |3-3|+|3-1| = 2$
\item $f(-1) = |-1-3|+|-1-1| = 4 + 2 = 6$
\item $f(-2) = |-2-3|+|-2-1| = 5 + 3 = 8$\\
\end{itemize}
Determinar todos los valores de $t$ para los que $f(t+2)=f(t)$\\
\begin{center}
\begin{tabular}{r c l}
$|t+2-3| + |t+2-1|$&=&$|t-3| + |t-1|$\\
$|t-1|+|t+1|$&=&$|t-3|+|t-1|$\\
$|t+1|$&=&$t-3$\\
\end{tabular}
\end{center}
Pro lo tanto  $t=1$\\\\
\end{ej}

%ejercicio 4
\begin{ej}
Sea $f(x)=x^2$  para todo real $x$. Calcular cada una de las fórmulas siguientes. En cada caso precisar los conjuntos de números erales $x, \; y \; t,$ etc., para los que la fórmula dada es válida.
\begin{enumerate}[\bfseries a)]
\item $f(-x)=f(x)$ \\\\
Demostración.- \; Se tiene $f(-x) = (-x)^2 = x^2 = f(x) \; \forall x \in \mathbb{R}$\\\\

\item $f(y)-f(x)=(y-x)(y+x)$\\\\
Demostración.- \; $f(y)-f(x)= y^2 - x^2 = (x-y)(x+y), \; \forall x, \; y \in \mathbb{R}$\\\\

\item $f(x+h) + f(x) = 2xh + h^2$\\\\
Demostración.- \; $f(x+h) + f(x) = (x+h)^2 -x^2 = x^2 + 2xh +h^2 - x^2 = 2xh + h^2, \; \forall x \in \mathbb{R}$\\\\

\item $f(2y) = 4f(y)$\\\\
Demostración.- \; $f(2y) = (2y)^2 = 4y^2 = 4 f(y), \; \forall y \in \mathbb{R}$\\\\

\item $f(t^2)=f(t)^2$\\\\
Demostración.- \; $f(t^2) = (t^2)^2 = f(t)^2$\\\\

\item $\sqrt{f(a)} = |a|$\\\\
Demostración.- \; $\sqrt{f(a)} = \sqrt{a^2} = |a|$\\\\
\end{enumerate}
\end{ej}

%ejercicio 5
\begin{ej}
Sea $g(x) = \sqrt{4-x^2}$ para $|x| \leq 2$. Comprobar cada una de las fórmulas siguientes e indicar para qué valores de $x, \; y, s$ y $t$ son válidas.
\begin{enumerate}[\bfseries a)]
\item $g(-x) = g(x)$\\\\
Se tiene $g(-x)=\sqrt{2-(-x)^2} = \sqrt{2-(x)^2} = g(x), \; \; para \; |x| \leq 2$\\\\

\item $g(2y) = 2\sqrt{1-y^2}$\\\\
$g(2y)=\sqrt{4-(2y)^2}= \sqrt{4(1-y^2)} = 2 \sqrt{1-y^2}, \; \; para \; |y|\leq 1$ Se obtiene $|y| \leq 1$  de $\sqrt{1-y^2}$ es decir $1-y^2 \geq 0$ entonces $\sqrt{y^2} \leq \sqrt{1}$ \; y \; $|y|\leq 1$\\\\

\item $g\left( \dfrac{1}{t} \right) = \dfrac{\sqrt{4t^2}-1}{|t|}$\\\\
$g\left( \dfrac{1}{t} \right) = \sqrt{4 - \left( \dfrac{1}{t} \right)^2} = \sqrt{\dfrac{4t^2 - 1}{t^2}} =\dfrac{\sqrt{4t^2 - 1}}{|t|}, \; \; para \; |t| \geq \dfrac{1}{2}$\\\\
Para hallar los valores correspondientes debemos analizar $\sqrt{4t^2 - 1}$\\\\
\item $g(a-2) = \sqrt{4a-a^2}$\\\\
$g(a-2) = \sqrt{4 - x^2} = \sqrt{4 - (a-2)^2} = \sqrt{4a - a^2}, \; \; para \; 0\leq a \leq 4$\\\\

\item $g \left( \dfrac{s}{2} \right) = \dfrac{1}{2} \sqrt{16 - s^2}$\\\\
$s\left( \dfrac{s}{2} \right) = \sqrt{4 - \left( \dfrac{s}{2} \right)^2} = \dfrac{\sqrt{16 - s^2}}{2}, \; \; para \; |s| \leq 4$\\\\

\item $\dfrac{1}{2 +g(x)} = \dfrac{2-g(x)}{x^2}$\\\\ 
$\dfrac{1}{2 +g(x)} = \dfrac{1}{2+ \sqrt{4-x^2}} \cdot \dfrac{2 - \sqrt{4-x^2}}{2 - \sqrt{4-x^2}} = \dfrac{2 - g(x)}{x^2}\; para \; \; 0 < |x| \leq 2 $\\\\ 
Evaluemos $\sqrt{4-x^2}$. Sea $4-x^2 \geq 0$ entonce $\sqrt{x^2} \leq 2$. Por otro lado tenemos que la función no puede ser $0$ por $\dfrac{1}{x^2}$, por lo tanto debe ser $x^2\leq 0$.\\\\ 
\end{enumerate}
\end{ej}

%ejercicio 6
\begin{ej}
Sea $f$ la función definida como sigue: $f(x)=1$ para $0 \leq x \leq 1;$ $f(x)=2$ para $1 < x \leq 2$. La función no está definida si $x<0$ ó si $x >2.$
\begin{enumerate}[\bfseries a)]
\item Trazar la gráfica de $f$ 

\begin{center}
\begin{tikzpicture}[scale=1,draw opacity = 0.6]
% abscisa y ordenada
\tkzInit[xmax= 3,xmin=0,ymax=3,ymin=0]
\tiny\tkzLabelXY[opacity=0.6,step=1, orig=false]
% label x, f(x)
\tkzDrawX[opacity=0.6,label=x,right=0.3]
\tkzDrawY[opacity=0.6,label=f(x),below = -0.6]
%dominio y función
\tkzFct[opacity=1,domain = 0:1]{1} 
\tkzText[opacity=0.6,above](0.5,1){\tiny $f(x)=1$}
\tkzFct[opacity=1,domain = 1:2]{2} 
\tkzText[opacity=0.6,above](1.5,2){\tiny $f(x)=2$}
% intervalos
\tkzSetUpPoint[shape=circle, size = 3, color=black, fill=black]
\tkzDefPointByFct[draw,with = a](0)
\tkzDefPointByFct[draw,with = a](1)
\tkzDefPointByFct[draw,with = b](2)
\tkzSetUpPoint[shape=circle, size = 3, color=black, fill=white]
\tkzDefPointByFct[draw,with = b](1)
\end{tikzpicture}
\end{center}

\item Poner $g(x) = f(2x).$ Describir el dominio de $g$ y dibujar su gráfica.
\begin{center}
\begin{tikzpicture}[scale=1,draw opacity = 0.6]
% abscisa y ordenada
\tkzInit[xmax= 3,xmin=0,ymax=3,ymin=0]
\tiny\tkzLabelXY[opacity=0.6,step=1, orig=false]
% label x, f(x)
\tkzDrawX[opacity=0.6,label=x,right=0.3]
\tkzDrawY[opacity=0.6,label=f(x),below = -0.6]
%dominio y función
\tkzFct[opacity=1,domain = 0:0.5]{1} 
\tkzFct[opacity=1,domain = 0.5:1]{2} 
% intervalos
\tkzSetUpPoint[shape=circle, size = 3, color=black, fill=black]
\tkzDefPointByFct[draw,with = a](0)
\tkzDefPointByFct[draw,with = a](0.5)
\tkzDefPointByFct[draw,with = b](1)
\tkzSetUpPoint[shape=circle, size = 3, color=black, fill=white]
\tkzDefPointByFct[draw,with = b](0.5)
\end{tikzpicture}
\end{center}
Debido a que $1\leq 2x \leq 1$ \; y \; $1 < 2x \leq 2$ el dominio de $g(x)$ es $0\leq x \leq 1$\\\\ 

\item Poner $h(x) = f(x-2).$ Describir el dominio de $k$ \; y dibujar su gráfica.
\begin{center}
\begin{tikzpicture}[scale=1,draw opacity = 0.6]
% abscisa y ordenada
\tkzInit[xmax= 4,xmin=0,ymax=3,ymin=0]
\tiny\tkzLabelXY[opacity=0.6,step=1, orig=false]
% label x, f(x)
\tkzDrawX[opacity=0.6,label=x,right=0.3]
\tkzDrawY[opacity=0.6,label=f(x),below = -0.6]
%dominio y función
\tkzFct[opacity=1,domain = 2:3]{1} 
\tkzFct[opacity=1,domain = 3:4]{2} 
% intervalos
\tkzSetUpPoint[shape=circle, size = 3, color=black, fill=black]
\tkzDefPointByFct[draw,with = a](2)
\tkzDefPointByFct[draw,with = a](3)
\tkzDefPointByFct[draw,with = b](4)
\tkzSetUpPoint[shape=circle, size = 3, color=black, fill=white]
\tkzDefPointByFct[draw,with = b](3)
\end{tikzpicture}
\end{center}
Debido a que $1\leq x-2 \leq 1$ \; y \; $1 < x-2 \leq 2$ el dominio de $h(x)$ es $2\leq x \leq 4$\\\\ 

\item Poner $k(x) = f(2x) + f(x-2).$ Describir el dominio de $k$ \; y dibujar su gráfica.\\\\
El dominio está vacío ya $f(2x)$ que solo está definido para $0 \leq x \leq 1$ \; y \;  $f(x-2)$ solo está definido para $2 \leq x \leq 4$. Por lo tanto no hay ninguno $x$ que satisfaga ambas condiciones. \\\\
\end{enumerate}
\end{ej}

%ejercicio 7
\begin{ej}
Las gráficas de los dos polinomios $g(x)=x$ \; y \; $f(x)=x^3$ se cortan en tres puntos. Dibujar una parte suficiente de sus gráficas para ver cómo se cortan.

\begin{center}
\begin{tikzpicture}[scale=0.9, draw opacity = .6]
% abscisa y ordenada
\tkzInit[xmax= 3,xmin=-2,ymax=3,ymin=-2]
\tiny\tkzLabelXY[opacity=0.6,step=1, orig=false]
% label x, f(x)
\tkzDrawX[opacity= .6,label=x,right=0.3]
\tkzDrawY[opacity= .6,label=f(x),below = -0.6]
%dominio y función
\tkzFct[opacity=1,domain = -2:3]{\x} 
\tkzText[above,opacity=0.6](3,3){\tiny $f(x)=x$}
\tkzFct[opacity=1,domain = -2:3]{\x**3} 
\tkzText[above,opacity=0.6](1,3){\tiny $f(x)=2x^3$}
\end{tikzpicture}
\end{center}
\end{ej}

%ejercicio 8
\begin{ej}
Las gráficas de los dos polinomios cuadráticos $f(x) = x^2-2$ \; y \; $g(x)=2x^2+4x+1$ se cortan en dos puntos. Dibujar las porciones de sus gráficas comprendidas entre sus intersecciones.
\begin{center}
\begin{tikzpicture}[scale=0.6, draw opacity = .6]
% abscisa y ordenada
\tkzInit[xmax= 4,xmin=-4,ymax=10,ymin=-3]
\tiny\tkzLabelXY[opacity=0.6,step=1, orig=false]
% label x, f(x)
\tkzDrawX[opacity= .6,label=x,right=0.3]
\tkzDrawY[opacity= .6,label=f(x),below = -0.6]
%dominio y función
\tkzFct[opacity=1,domain = -4:4]{\x**2 - 2} 
\tkzText[above,opacity=0.6](1.5,10){\tiny $f(x)=x^2 - 2$}
\tkzFct[opacity=1,domain = -4:4]{2*\x**2 + 4*\x +1} 
\tkzText[above,opacity=0.6](5.5,10){\tiny $f(x)=2x^2 +4x + 1$}
\end{tikzpicture}
\end{center}
\end{ej}

% 1.5 problema  10 pag 70
\begin{ej}
En cada caso, hallar todos los polinomios $p$ de grado $\leq 2$ que satisfacen las condiciones dadas.\\\\
Sabemos que para un polinomio de grado $\leq 2$ es:
$$p(x) = ax^2 + bx + c$$ 
para todo $a,b,c \in \mathbb{R}$.\\\\
\begin{enumerate}[\bfseries a)]
\item $p(x) = p(1-x)$\\\\
Sea $f(x) = p(x) -1$, entonces $f$ es de grado como máximo $2$ por la parte $d)$ del problema $9$ tenemos que  todos los coeficientes de $f$ son $0$ \; y \; $f(x)=0$ para todo $x \in \mathbb{R}$, así,
$$p(x)-1 = 0 \; \; \Rightarrow \; \; p(x)=1 \; \; \forall x \in \mathbb{R}$$\\\\ 
\item $p(x) = p(1+x)$\\\\
Tenemos $p(0)=1 \Rightarrow c=1$ luego, $p(1)=1 \Rightarrow a+b=0 \Rightarrow b=-a$ y finalmente, con $c=1$ \; y \; $b=-a$, tenemos: $p(2)=2 \Rightarrow 4a-2a=1 \Rightarrow a=\dfrac{1}{2}, \; b=-\dfrac{1}{2}$. por lo tanto $$p(x)=\dfrac{1}{2}x^2 - \dfrac{1}{2}x + 1 = \dfrac{1}{2}x(x-1) + 1$$\\\\
\item $p(x) = p(0) = p(1) =1$\\\\
Una vez mas, desde $p(0)=1$ tenemos: $a+b=0 \Rightarrow b=-a$ así, $p(x) = ax^2 - ax + 1 = ax(x-1) + 1$\\\\
\item $p(0) = p(1)$\\\\
Simplemente sustituyendo estos valores que tenemos, $p(0)=p(1) \Rightarrow c=a+b+c \Rightarrow b=-a$ entonces, $$p(x) = ax^2 -ax + c = ax(x-1) + c$$\\\\
\end{enumerate}
\end{ej}

%1.5 problema 11 pag 70
\begin{ej}
En cada caso, hallar todos los polinomios $p$ de grado $\leq 2$ que para todo real $x$ satisfacen las condiciones que se dan.
Como $p$ es un polinomio de grado por lo mucho $2$, podemos escribir
$$p(x) = ax^2 + bx + c, \; \; \; para \; a,b,c \in \mathbb{R}$$
\begin{enumerate}[\bfseries a)]
\item $p(x) = p(1-x)$\\\\
Sustituyendo se tiene $p(x) = p(1-x) = ax^2 + bx + c = a(1-x)^2 + b(1-x) + c \Rightarrow a - 2ax + ax^2 + b - bx + c$ por lo tanto $$ax^2 + (-2a-b)x + (a+b+c)$$
Así para $a=a$, $b = -2a - b \Rightarrow a=-b$, $c=a+b+c$ entonces $$p(x) = -bx^2 + bx + c = bx(1-x)+c$$\\\\
\item $p(x) = p(x) = p(1+x)$\\\\
Una vez más sustituyendo, $p(x) = p(1+x) \Rightarrow ax^2 + bx + c = a(1+x)^2 + b(1+x) + c = ax^2 + (2a+b)x + (a+b+c)$. Luego, igualando como potencias de $x$, $a=a$, $b=2a+b \Rightarrow a = 0$, $c=a+b+c \Rightarrow b=0$. Por lo tanto $p(x) = c$ donde $c$ es una constante arbitraria.\\\\
\item $p(2x) = 2p(x)$\\\\
Sustituyendo, $p(2x) = 2p(x) \Rightarrow 4ax^2 + 2bx + c = 2ax^2 + 2bx +2c.$. Igualando a las potencias de $x$, $4a=2a \Rightarrow a=0,$ $2b=2b \Rightarrow  b \; arbitrario,$ $c=2c \Rightarrow c=0$.\\
Así $$p(x)bx, \; b \; arbitrario$$\\\\
\item $p(2x) = p(x+3)$\\\\
Sustituyendo $p(3x) = p(x+3) \Rightarrow 9ax^2 + 3bx + c = ax^2 + (6a+b)x + (9a+3b+c)$. Igualando como potencias de $x,$ $9a=a \Rightarrow a=0,$ $3b = 6a+b \Rightarrow b=0,$ $c=9a + 3b + c = c \Rightarrow c \; arbitrario$. Por lo tanto $$p(x)=c \mbox{para c constante arbitrario.}$$\\\\
\end{enumerate}
\end{ej}



\subsection[Teoremas]{Teoremas \footnote{Calculus Vol 1, Tom Apostol, pag 69-70}}

%teorema problema 9 pag 70
\begin{teo}
Este ejercicio desarrolla ciertas propiedades fundamentales de los polinomios. Sea $f(x)=\displaystyle\sum_{k=0}^{n} c_k x^k$ un polinomio de grado $n$. Demostrar cada uno de los siguientes apartados:
\begin{enumerate}[\bfseries a)]
\item Si $n \geq 1$ \; y \; $f(0)=0,$ $f(x)=xg(x)$, siendo $g$ un polinomio de grado $n-1.$\\\\
Para entender lo que nos quiere decir Apostol pongamos un ejemplo. Supongamos que tenemos un polinomio donde $f(x)=2x^2+3x-x$ entonces notamos que $f(x)=x(2x+3-1)$ donde $g(x)=2x+3-1$, esto quiere decir que si $0 = f(0)=c_0 \Rightarrow c_1 x + c_2 x^2 + ... + c_n x^n = x(c_1 + c_2 x + ... + c_n x^{n-1})$ 
Así que debemos demostrar que $f(x)$ es un polinomio arbitrario de grado $n \geq 1$ tal que $f(0)=0$, entonces debe haber un polinomio de grado $n-1, \; g(x)$, tal que $f(x)=xg(x)$\\\\
Demostración.- \; Sabemos que $$f(0) = c_n \cdot 0^n + c_{n-1} \cdot 0^{n-1} + ... + c_1 \cdot 0 + c_0 =c_0,$$ como $f(0)=0$ se concluye que $c_0=0$. Así tenemos $$f(x)=\displaystyle\sum_{k=1}^{n} c_k x^k.$$ Ahora crearemos una función $g(x).$ Dada la función $f(x)$ como la anterior, definamos, $$f(x)=\displaystyle\sum_{k=0}^{n} c_k x^{k-1} = \sum_{k=1}^{n} c_k x^{k-1}$$ 
Ahora crearé una función $g(x)$. Dada una función $f(x)$ como la anterior, definamos  $$g(x) = \displaystyle\sum_{k=1}^{n} c_k x^{k-1}$$
donde $c_k$ son los mismos que los dados por la función $f(x)$. Primero notemos que el grado de $g(x)$ es $n-1$. Finalmente, tenemos que $$ xg(x) = x \displaystyle\sum_{k=1}^{n} c_k x^{k-1} = \sum_{k=1}^{n} c_k x^k = f(x).$$ \\\\

\item Para cada real $a$, la función $p$ dada por $p(x)=f(x+a)$ es un polinomio de grado $n.$\\\\
Demostración.- \; Usando el teorema del binomio,
\begin{center}
\begin{tabular}{r c l}
$f(x+a)$&=&$ \displaystyle\sum_{k=0}^{n} (x+a)^k c_k$\\\\
&=&$c_o + (x+a)c_1 + (x+a)^2 c_2 + ...+(x+a)^n c_n$\\\\
&=&$ c_o + c_1 \left( \displaystyle\sum_{j=0}^{1} {1 \choose j} a^j x^{1-j} \right) + c_2 \left( \displaystyle\sum_{j=0}^2 {2 \choose j} a^j x^{2-j} \right) 
+ ... + c_n \left( \displaystyle\sum_{j=0}^{n} {n \choose j} a^j x^{n-j} \right)$\\\\
&=&$(c_o + ac_1 + a^2 c_2 + ... + a^n c_n) + x(c_1 + 2ac_2 + ... + na^{n-1} c_n)$\\\\
&=&$\displaystyle\sum_{k=0}^n \left( x^k \left( \displaystyle\sum_{j=k} {j \choose j-k} c_j a{j-k}\right) \right)$\\
\end{tabular}
\end{center}
En la linea final reescribimos los coeficientes como sumas para verlos de manera más concisa. De cualquier manera, dado que todos los $c_i$ son constantes, tenemos $\displaystyle\sum_{j=k}^n {j \choose j-k} c_j a^{j-k}$ es alguna constante para cada $k,$ de $d_k$ y tenemos, $$p(x) = \displaystyle\sum_{k=0}^n d_k x^k$$\\\\

\item Si $n \geq 1$ \; y \; $f(a)=0$ para un cierto valor real $a$, entonces $f(x) = (x-a) \, h(x),$ siendo $h$ un polinomio de grado $n-1$. (considérese $p(x)=f(x+a).$)\\\\
Demostración.- \; Por la parte $b)$ se sabe que $f(x)$ es un polinomio de grado $n$, entonces $p(x)=f(x+a)$ también es un polinomio del mismo grado. Ahora si $f(a)=0$ entonces por hipótesis $p(0)=f(a)=0$. Luego por la parte $a)$, tenemos $$p(x)=x \cdot g(x)$$ donde $g(x)$ es un polinomio de grado $n-1$. Así,
$$p(x-a)=f(x)= f(x) = (x-a) \cdot g(x-a)$$ ya que $p(x)=f(x+a)$. Pero, si $g(x)$ es un polinomio de grado $n-1$, entonces por la parte $b)$ nuevamente, también lo es $h(x)=g(x+(-a)) = g(x-a)$. Por lo tanto, $$f(x)=(x-a) \cdot h(x)$$ para $h$ un grado $n-1$ polinomial, según lo solicitado.\\\\\

\item Si $f(x)=0$ para $n+1$ valores reales de $x$ distintos, todos los coeficientes $c_k$ son cero y $f(x)=0$ para todo real de $x$\\\\
Demostración.- \;  La prueba se realizara por inducción. Sea $n=1$, entonces $f(x)=c_o + c_1 x$. Dado que la hipótesis es que existen $n+1$ distintos  $x$ de tal manera que $f(x)=0$, sabemos que existen $a_1$, $a_2$ $\in \mathbb{R}$ tal que $$f(a_1)=f(a_2)=0, \; \; \; a_1 \neq a_2,$$ Así, 
\begin{center}
\begin{tabular}{r c l c r c l l}
$c_0 + c_1 a_1$&$=$&$0$&$\Rightarrow$&$c_1 a_1 - c_1 a_2$&$=$&$0$&\\
&&&$\Rightarrow$&$c_1(a_1 - a_2)$&$=$&$0$&\\
&&&$\Rightarrow$&$c_1$&$=$&$0$& ya que $a_1 \neq a_2$\\
&&&$\Rightarrow$&$c_0$&$=$&$0$&ya que $c_0 + c_1 a_1 = 0$\\
\end{tabular}
\end{center}
Por lo tanto, la afirmación es verdadera. Suponga que es cierto para algunos $n=k \in \mathbb{Z}^{+}$. Luego Sea $f(x)$ un polinomio de grado $k+1$ con $k+2$ distintos de $0$, $a_1,...,a_{k+2}.$ ya que $f(a_{k+2}) = 0$, usando la parte $c)$, tenemos, $$f(x)=(x- a_{k+2})h(x)$$
donde $h(x)$ es un polinomio de grado $k$. Sabemos que hay $k+1$ valores distintos $a_1,... a_{k+1}$ tal que  $h(a_i) = 0.$ Dado que $f(a_i)=0$ para $1< i < k+2y$ y $(x- a_{k+2}) \neq 0$ para $x=a_i$ con $1 < i < k+1$ ya que todos los $a_1$ son distintos), por lo tanto, según la hipótesis de inducción, cada coeficiente de $h$ es $0$ y $h(x)=0$ para todo $x \in \mathbb{R}.$ Así,
\begin{center}
\begin{tabular}{r c r c l}
$f(x)$&$=$&$(x - a_{k+2})h(x)$&$=$&$(x- a_{k+2}) \cdot \displaystyle\sum_{j=0}^k c_j x^j$\\\\
&&&$=$&$\displaystyle\sum_{j=0}^k (x - a_{k+2})c_j x^j$\\\\
&&&$=$&$c_k x^{k+1} + (c_{k-1} - a_k + 2c_k)x^k + ... + (c_1 -a_{k+2} c_0)x + a_{k+2} c_0$\\
\end{tabular}
\end{center}
Pero dado que todos los coeficientes de $h(x)$ son cero y $f(x) = 0$ para todo $x \in \mathbb{R}$. Por lo tanto, la afirmación es verdadera para el caso $k+1$ y para todo $n \in \mathbb{Z}^+$\\\\

\item Sea $g(x) = \displaystyle\sum_{k=0}^m b_k x^k$ un polinomio de grado $m$, siendo $m \geq n.$ Si $g(x)=f(x),$ para $m+1$ valores reales de $x$ distintos, entonces $m=n$, $b_k =c_k$ para cada valor de $k$, y $g(x)=f(x)$ para todo real $x$\\\\
Demostración.- \; Sea $$p(x)=g(x)-f(x) = \displaystyle\sum_{k=0}^m b_k x^k - \sum_{k=0}^n c_k x^k = \sum_{k=0}^m (b_k - c_k)x^k$$ donde $c_k=0$ para $n< k \leq m$, cabe recordar que tenemos $m \geq n$.\\
Entonces, hay $m+1$ distintos reales $x$ para los cuales $p(x)=0$. Dado que hay $m+1$ valores reales distintos para lo cuál $g(x)=f(x),$ así en cada uno de estos valores $p(x)=g(x)-f(x)=0$. Por lo tanto, por la parte $d)$, $b_k-c_k =0$ para $k=0,...,m$ y $p(x)=0$ para todo $x \in \mathbb{R}$. Es decir $$b_k - c_k =0 \; \; \; \Rightarrow \; \; \; b_k=c_k \; \; \; para \; k=0,...,m$$ y $$p(x)=0 \; \; \; \Rightarrow \; \; \; g(x)-f(x)=0 \; \; \; \Rightarrow \; \; \; f(x)=g(x),$$ para todo $x \in \mathbb{R}$. Ademas desde $b_k -c_k =0$ para $k=0,..., m$ y por supuesto $c_k=0$ para $k=n+1, ... , m,$ tenemos $b_k=0$ para $k= n+1, ... , m.$ Pero entonces, $$g(x)=\displaystyle\sum_{k=0}^n b_k x^k + \sum_{k=n+1}^m 0 \cdot x^k = \sum_{k=0}^n b^k x^k$$ significa que $g(x)$ es un polinomio de grado $n$ también.\\\\
\end{enumerate}
\end{teo}

\begin{col.}Probar que:
$$\displaystyle\sum_{k=0}^n x^k = \dfrac{1 - x^{n+1}}{1-x} \; \; para x\neq 1$$\\\\
Demostración.- \; Usando propiedades de suma, $$(1-x)\displaystyle\sum_{k=0}^n x^k = \sum_{k=0}^n (x^k - x^{k+1}) = - \sum_{k=0}^n (x^{k+1} - x^k) = -(x^{n+1} -1 = 1 - x{n+1})$$
En la penultima igualdad se deriva de la propiedad telescópica, por lo tanto nos queda,
$$\displaystyle\sum_{k=0}^n x^k = \dfrac{1 - x^{n+1}}{1-x}$$ \\\\
\end{col.}

\begin{col.}Probar la identidad
$$\displaystyle\prod_{k=1}^n \left( 1 + x^{2^{k-1}} \right) = \dfrac{1 - x^{2^n}}{1-x}, \; \; para \; x\neq 1$$\\\\
Demostración.- \; Para $n=1$ a la izquierda tenemos,
$$\displaystyle\prod_{k=1}^n \left( 1 + x^{2^{k-1}} \right) = \prod_{k=0}^1 \left( 1 + x^{2^{k-1}} = 1 + x^{2^0} = 1 + x  \right)$$
Por otro lado a la derecha se tiene, $$\dfrac{1- x^{2^n}}{1-x} = \dfrac{1 - x^2}{1-x} = \dfrac{(1-x)(1+x)}{1-x} = 1+x$$
Concluimos que la identidad se mantiene para $n=1$. Ahora supongamos que es válido para algunos $n=m \in \mathbb{Z}^+,$
\begin{center}
\begin{tabular}{r c l}
$\displaystyle\prod_{k=1}^{m+1}$&$=$&$\left( +x^{2^m} \right) \cdot \displaystyle\prod_{k=1}^m \left( 1 + x^{2^{k+1}} \right)$\\\\
$=$&$\left( 1 + x^{2^m} \right) \cdot  \left( \dfrac{1-x^{2^m}}{1-x} \right)$\\\\
$=$&$\dfrac{(1+x^{2^m}) (1- x^{2^m})}{1-x}$\\\\
$=$&$\dfrac{1 - x^{2^{m+1}}}{1-x}$\\\\
\end{tabular}
\end{center} 
Por lo tanto, la afirmación es verdadera para $m+1,$ y así para todo $n \in \mathbb{Z}^+$\\\\
\end{col.}

% 1.5 problema 12 pag 70
\begin{teo}
Demostrar que las expresiones siguientes son polinomios poniéndolas en la forma $\displaystyle\sum_{k=0}^m a_k x^k$ para un valor de $m$ conveniente. En cada caso $n$ es entero positivo.
\begin{enumerate}[\bfseries a)]
\item $(1+x)^{2n}$\\\\
Demostración.- \; Usando el teorema binomial $(1+x)^{2n} = \displaystyle\sum_{k=0}^2n {2n \choose k} x^k$, sea $m=2n$ entonces $\displaystyle\sum_{k=0}^m {m \choose n} x^k$, por lo tanto $\displaystyle\sum_{k=0}^m c_k x^k$ si $c_k = {m \choose k} $ para cada $k$.\\\\

\item $\dfrac{1- x^{n+1}}{1-x}, \; x \neq 1$\\\\
Demostración.- \; Por el corolario anterior 
\begin{center}
\begin{tabular}{r c l}
$\dfrac{1 - x^{n+1}}{1-x}$&=&$\dfrac{(1-x)(1+x+...+x^n)}{1-x}$\\\\
&=&$1 + x + ... + x^n$\\\\
&=&$\displaystyle\sum_{k=0}^n 1 \cdot x^k$\\\\
\end{tabular}
\end{center}

\item $\displaystyle\prod_{k=0}^n (1+x^{2^k})$\\\\
Demostración.- \; Por le corolario anterior,
\begin{center}
\begin{tabular}{r c l}
$\displaystyle\prod_{k=0}^n \left( 1+x^{2^k} \right)$&=&$\dfrac{(1 - x^{2^{n+1}})}{1-x}$\\
&=&$\dfrac{(1-x^{2^n})(1+x^{2^n})}{1-x}$\\\\
&=&$\left( \dfrac{1 - x^{2^n}}{1-x} \right) (1+x^{2^n})$\\\\
&=&$(1+x+...+x^{2^n + 1})(1 + x^{2^n})$\\\\
&=&$(1+x+...+x^{2^n + 1})(x^{2^n} + x^{2^n +1} + ... + x^{2^{n+1} - 1})$\\\\
&=&$\displaystyle\sum_{k=0}^{2^{n+1} - 1} 1 \cdot x^k$\\\\
&=&$\displaystyle\sum_{k=0}^{m} 1 \cdot x^k$ si $m = 2^{n+1} - 1$\\\\
\end{tabular}
\end{center}
\end{enumerate}
\end{teo}