\chapter{Valor absoluto}
\begin{tcolorbox}
\begin{def.}
\begin{equation}
|x| = \left\lbrace
\begin{array}{crr}
\textup{si } & x, & x\geq 0\\
\textup{si }& -x, & x \leq 0
\end{array}
\right.
\end{equation}
\end{def.}
\end{tcolorbox}
%teorema 4.1
\begin{teo}
Probemos que $|x|=\sqrt{x^2}$\\\\
Sea $x<0$, entonces por definición $|x|=-x>0$. Después vemos que $(-x)^2=x^2$.\\
Decimos que hay dos raíces cuadradas de cualquier número positivo, $y$ e $-y$, por lo tanto por el teorema 3.21 \: $y$ tiene una raíz cuadrada no negativa única $\sqrt{y}$. Así $\sqrt{x^2}=-x=|x|$ cuando $x<0$ y $\sqrt{x^2}=x=|x|$ cuando $x\geq 0 $ \\\\
\end{teo}

%teorema 4.2
\begin{teo}
Si $a\geq 0$, es \; $|x|\leq a$ \; si y sólo si \; $-a\leq x \leq a$\\\\
Demostración.- \; Debemos probar dos cuestiones: primero, que la desiguadad $|x|\leq a$ implica las dos desigualdades $-a \leq x \leq a$ y recíprocamente, que $-a \leq x \leq a$ implica $|x|\leq a$.\\ 
Ya supuesto $|x|\leq a$ se tiene también $-a \leq - |x|$. Pero ó \; $x=|x|$ \; ó \; $x = -|x|$ y, por lo tanto, $x\leq a$ \; y \; $-a \leq x$, lo cual prueba la primera parte del teorema.\\
Para probar el recíproco, supóngase $-a\leq x \leq a$. Si $x\leq 0$ se tiene $|x|=x\leq a$; si por el contrario es $x\leq 0$ , entonces $|x|=-x \leq a$. En ambos casos se tiene $|x|\leq a$, lo que demuestra el teorema.\\\\   
\end{teo}

%teorema 4.3
\begin{teo}[Desigualdad triangular]
Para todos los números $a$ \; y \; $b$ se tiene $$|a+b|\leq |a|+|b|$$
Demostración.- \; Vamos a considerar cuatro casos:
\begin{center}
\begin{tabular}{c r r}
$(1)$&$a\geq 0$&$b\geq 0$\\
$(2)$&$a\geq $&$b\leq 0$\\
$(4)$&$a\leq $&$b \geq 0$\\
$(5)$&$a\leq 0$&$b \leq 0$\\
\end{tabular}
\end{center}
En el caso ($1)$ tenemos también $a+b\geq 0$, esto es evidente; en efecto por definición $$|a+b|=a+b=|a|+|b|$$ de modo que en este caso se cumple la igualdad.\\
En el caso $(4)$ se tiene $a+b\leq 0$ y de nuevo se cumple la igualdad: $$|a+b|=-(a+b)=(-a)+(-b)=|a|+|b|$$
En el caso $(2)$, cuando $a\geq 0$ y $b\leq 0$, debemos demostrar que $$|a+b|\leq a - b$$ Este caso puede dividirse en dos subcasos. Si $a+b\geq 0$, entonces tenemos que demostrar que $$a+b \leq a-b$$ es decir, $$b\leq -b,$$  lo cual se cumple ciertamente puesto que $b$ es negativo y $-b$ positivo. Por otra parte, si $a+b\leq 0$ debemos demostrar que $$-a-b\leq a-b$$ es decir $$-a\leq a,$$ lo cual es verdad puesto que $a$ es positivo y \; $-a$ negativo.\\
Nótese finalmente que el caso $(3)$ puede despacharse sin ningún trabajo adicional aplicando el caso $(2)$ con $a$ \; y \; $b$ intercambiados.\\\\ 
Se puede dar una demostración mas corta dado que $$|a|=\sqrt{a^2} \; ó \; |a|^2=a^2$$. Sea $(|a+b|)^2=(a+b)$ Entonces 
\begin{center}
\begin{tabular}{r c r c l}
$(|a+b|)^2$&$=$&$(a+b)^2$&$=$&$a^2+2ab+b^2$\\\\
&&&$\leq$&$a^2+2|a|\dot |b|+b^2$\\\\
&&&$=$&$|a|^22|a|\dot |b|+|b|^2$\\\\
&&&$=$&$(|a|+|b|)^2$\\\\
\end{tabular}
\end{center}
De esto podemos concluir que $|a+b|\leq |a|+|b|$ porque $x^2<y^2$ implica $x<y$\\\\
Hay una tercera forma de probar que es utilizando el teorema anterior.\\
Puesto que $x=|x|$ \; ó \; $x=-|x|$, se tiene $-|x|\leq x \leq |x|$. Análogamente $-|y| \leq y \leq |y|$. Sumando ambas desigualdades se tiene: $$-(|x|+|y|)\leq x+y \leq |x|+|y|$$ y por tanto en virtud del teorema 4.2 se concluye que: $|x+y|\leq |x|+|y|$\\\\  
\end{teo}

%teorema 4.4
\begin{teo}
Demostrar que $|a|\geq b$ entonces $a\leq -b$ ó $a\geq b$\\\\
Demostración.- \; Por definición se tiene que
\begin{equation}
|a| = \left\lbrace
\begin{array}{crr}
\textup{si } & a, & a\geq b\\
\textup{si }& -a, & a \leq b
\end{array}
\right.
\end{equation}
Entonces $a\geq b$ \; $\lor$ \; $-a \geq b = a\leq -b$\\\\
\end{teo}

\section{Ejercicios y Demostraciones}
\subsection[Ejercicios]{Ejercicios\footnote{Calculo infinitesimal, Michael Spivak, pag 19-26}}
%ejercicio 4.1
\begin{ej}
Dese una expresión equivalente de cada una de las siguientes utilizando como mínimo una vez menos el signo de valor absoluto.\\
\begin{center}
\begin{tabular}{r r c l}
$(i)$&$|\sqrt{2}+\sqrt{3}-\sqrt{5}+\sqrt{7}|$&$\Rightarrow$&$\sqrt{2}+\sqrt{3}-\sqrt{5}+\sqrt{7}$\\\\
$(ii)$&$||a+b|-|a|-|b||$&$\Rightarrow$&$|a+b|-|a|-|b|$\\\\
$(iii)$&$|\left( |a+b|+|c|-|a+b+c| \right)|$&$\Rightarrow$&$|a+b|+|c|-|a+b+c|$\\\\
$(iv)$&$|x^2-2xy+y2|$&$\Rightarrow$&$x^2-2xy+y2$\\\\
$(v)$&$|\left(  |\sqrt{2}+ \sqrt{3}|-|\sqrt{5}-\sqrt{7}|  \right)|$&$\Rightarrow$&$\sqrt{2}+ \sqrt{3}|-|\sqrt{5}-\sqrt{7}$\\\\
\end{tabular}
\end{center}
\end{ej}

%ejercicio 4.2
\begin{ej}
Expresar lo siguiente prescindiendo de signos de valor absoluto, tratando por separado distintos casos cuando sea necesario.
\begin{enumerate}[\bfseries (i)]
\item $|a+b|-|b|$
\begin{center}
\begin{tabular}{rcrclcrcl}
$a$&si&$a$&$\geq$&$-b$&y&$b$&$\geq$&$0$\\
$-a$&si&$a$&$\leq$&$-b$&y&$b$&$\leq$&$0$\\
$a+2b$&si&$a$&$\geq$&$-b$&y&$b$&$\leq$&$0$\\
$-a-2b$&si&$a$&$\leq$&$-b$&y&$b$&$\geq$&$0$\\
\end{tabular}
\end{center}
\item $|x|-|x^2|$
\begin{center}
\begin{tabular}{r c r c l}
$x-x^2$&si&$x$&$\geq$&$0$\\
$-x-x^2$&si&$x$&$\leq$&$0$\\\\
\end{tabular}
\end{center}
\end{enumerate}
\end{ej}

%ejercicio 4.3
\begin{ej}
Encontrar todos los números $x$ para los que se cumple.
\begin{enumerate}[\bfseries (i)]
\item $|x-3|=8$
\begin{center}
\begin{tabular}{rcccll}
$-8$&$=$&$x-3$&$=$&$8$&teorema 4.1\\
$-5$&$=$&$x$&$=$&$11$&\\\\
\end{tabular}
\end{center}
\item $|x-3|<8$
\begin{center}
\begin{tabular}{rcccll}
$-8$&$<$&$x-3$&$<$&$8$&teorema 4.1\\
$-5$&$<$&$x$&$<$&$11$&\\\\
\end{tabular}
\end{center}
\item $|x+4|<2$
\begin{center}
\begin{tabular}{rcccll}
$-2$&$<$&$x+4$&$<$&$-2$&teorema 4.1\\
$-6$&$<$&$x$&$<$&$-2$&\\\\
\end{tabular}
\end{center}
\item $|x-1|+|x-2|>1$\\\\
Por definición:
\begin{equation}
|x-1| = \left\lbrace
\begin{array}{rcr}
  x-1& si & x\geq 1\\
 1-x& si & x \leq 1\\\\
\end{array}
\right.
\end{equation}
\begin{equation}
|x-2| = \left\lbrace
\begin{array}{rcr}
  x-2& si & x\geq 2\\
 2-x& si & x \leq 2\\
\end{array}
\right.
\end{equation}
Por lo tanto queda comprobar:\\
\begin{center}
\begin{tabular}{c c c r c l}
Si&$x\leq 1$&$\Rightarrow$&$(1-x)+(2-x)>1$&$\Rightarrow$&$x<1$\\\\
Si&$1\leq x\leq 2$&$\Rightarrow$&$(x-1)+(2-x)>1$&$\Rightarrow$&$1>1$\\\\
Si&$x\geq 2$&$\Rightarrow$&$(x-1)+(x-2)>1$&$\Rightarrow$&$x>2$\\\\
\end{tabular}
\end{center}
Así: $x<1 \; \lor \; x>2$\\\\

\item $|x-1|+|x+1|<2$\\\\
Por definición:
\begin{equation}
|x-1| = \left\lbrace
\begin{array}{rcr}
  x-1& si & x\geq 1\\
 1-x& si & x \leq 1\\\\
\end{array}
\right.
\end{equation}
\begin{equation}
|x+1| = \left\lbrace
\begin{array}{rcr}
  x+1& si & x\geq -1\\
 -1-x& si & x \leq -1\\
\end{array}
\right.
\end{equation}
Por lo tanto queda comprobar:\\
\begin{center}
\begin{tabular}{c c c r c l}
Si&$x\leq -1$&$\Rightarrow$&$(1-x)+(1-x)<2$&$\Rightarrow$&$x>-1$\\\\
Si&$-1\leq x \leq 1$&$\Rightarrow$&$(1-x)+(x+1)<2$&$\Rightarrow$&$2<2$\\\\
Si&$x\geq 1$&$\Rightarrow$&$(x-1)+(x+1)<2$&$\Rightarrow$&$x<1$\\\\
\end{tabular}
\end{center}
Pero es falso que $x$ satisface a $-1\leq x \leq 1$, y contradice a que $x$ satisface a todos los reales, por lo tanto no existe solución\\\\ 
\item $|x-1|+|x+1|<1$ \\\\ De la misma manera que el anterior ejercicio no tiene solución para ningún $x$.\\\\

\item $|x-1|\cdot |x+1|=0$\\\\
Por definición:
\begin{equation}
|x-1| = \left\lbrace
\begin{array}{rcr}
  x-1& si & x\geq 1\\
 1-x& si & x \leq 1\\\\
\end{array}
\right.
\end{equation}
\begin{equation}
|x+1| = \left\lbrace
\begin{array}{rcr}
  x+1& si & x\geq -1\\
 -1-x& si & x \leq -1\\
\end{array}
\right.
\end{equation}
queda comprobar:\\
\begin{center}
\begin{tabular}{c c c r c l}
Si&$x\leq -1$&$\Rightarrow$&$(1-x)+(-1-x)=0$&$\Rightarrow$&$x\leq-1 \, \cup \, x = 1 \, \cup \, x=-1 $\\\\
Si&$-1\leq x \leq 1$&$\Rightarrow$&$(1-x)\dot (x+1)=0$&$\Rightarrow$&$-1\leq x \leq 1 \; \cup \; x=1 \; \cup \; x=-1$\\\\
Si&$x\geq 1$&$\Rightarrow$&$(x-1)\dot (x+1)=0$&$\Rightarrow $&$x \notin \mathbb{R}$\\\\
\end{tabular}
\end{center}
Por lo tanto $x=1$ \; ó \; $x=-1$\\\\

\item $|x-1|\cdot |x+2|= 3$
\\\\
\end{enumerate}
\end{ej}

\section[Demostraciones]{Demostraciones \footnote{Calculo infinitesimal, Michael Spivak, pag 20}}
%teorema 4.4
\begin{teo}
$|xy|=|x|\cdot |y|$\\\\
Demostración.- \; Si $|xy|$ Por teorema 4.1 \; $\sqrt{(xy)^2}$ luego por propiedad 3.1 \; $\sqrt{x^2 \cdot y^2}$, así $\sqrt{x^2}\cdot \sqrt{y^2}$ \; y \;  $|x|\cdot |y|$\\\\
\end{teo}

%teorema 4.5
\begin{teo}
$\left| \dfrac{1}{x} \right|=\dfrac{1}{|x|}$\\\\

Demostración.- \; Si $\left| \dfrac{1}{x} \right|$ por definición $\sqrt{(x^{-1})^2}$, después $\left( \dfrac{1}{x}\right) ^{2/2}$, por propiedad 3.1 \; $\dfrac{\sqrt{1^2}}{\sqrt{x^2}}$, luego $\dfrac{1}{|x|}$  \\\\
\end{teo}

%teorema 4.6
\begin{teo}
$\dfrac{|x|}{|y|}=\left| \dfrac{x}{y} \right|$ si $y\neq 0$\\\\
Demostración.- $$ \dfrac{|x|}{|y|} = \dfrac{\sqrt{x^2}}{\sqrt{y^2}}=\dfrac{x^{2/2}}{y^{2/2}}=\left( \dfrac{x}{y} \right)^{2/2} = \sqrt{\left( \dfrac{x}{y} \right)^2}=\left| \dfrac{x}{y} \right| $$ \\\\
\end{teo}

%teorema 4.7
\begin{teo}
$|x-y|\leq |x|+|y|$\\\\
Demostración.- \; Sea $(|x-y|)^2\leq( |x|+|y| )^2$, entonces:
\begin{center}
\begin{tabular}{r c l l}
$(|x-y|)^2$&$=$&$(x-y)^2$&\\\\
&$=$&$x^2-2xy+y^2$&\\\\
&$\leq$&$|x|^2+|-2xy|+|y|^2$&Ya que $-2xy\leq |-2xy|$\\\\
&$=$&$|x|^2+|-2||x||y|+|y|^2$&Por teorema 4.4.\\\\
&$=$&$|x|^2+2|x||y|+|y|^2$&\\\\
&$=$&$(|x|+|y|)^2$&\\\\
\end{tabular}
\end{center}
luego por teorema 2.18 \; $|x-y|\leq |x|+|y|$\\\\
\end{teo}

%teorema 4.8
\begin{teo}
$|x|-|y|\leq |x-y|$\\\\
Demostración.- \; Su demostración es parecida al anterior teorema, 
\begin{center}
\begin{tabular}{r c l l}
$(|x|-|y|)^2$&$=$&$|x|^2-2|x||y|+|y|^2$&\\\\
&$\leq$&$x^2-2xy+y^2$& por el contrareciproco de $|2xy|\geq 2xy$\\\\
&$=$&$(x-y)^2$&\\\\
&$=$&$|x-y|^2$&\\\\
\end{tabular}
\end{center}
Por lo tanto $|x|-|y|\leq |x-y|$\\\\
\end{teo}

%teorema 4.9
\begin{teo}
$\left| |x|-|y| \right| \leq |x-y|$ (¿ Por qué se sigue esto inmediatamente del anterior teorema ?)\\\\
Demostración.- \;  Sea $\sqrt{(|x|-|y|)^2}$ entonces, $$\sqrt{(|x|-|y|)^2}=\sqrt{(x^2-2|x||y|+y^2}\leq \sqrt{x^2-2xy+y^2}$$ y por definición se tiene $|x-y|$\\\\
\end{teo} 

%teorema 4.10
\begin{teo}
$|x+y+z| \leq |x|+|y|+|z|$\\\\
Demostración.- \: Sea $\sqrt{(x+y+9)^2}$ entonces,  $$\sqrt{x^2+z^2++y^2+2xy+2xz+2yz} \leq \sqrt{|x|^2+|z|^2++|y|^2+2|x||y|+2|x||z|+2|y||z|}$$ por lo tanto $\sqrt{(|x|+|y|+|z|)^2}$. La igualdad se prueba si $\forall x,y,z \geq 0$ ó $\forall x,y,z \leq 0$\\\\
\end{teo}

%teorema 4.11
\begin{teo}
El máximo de dos números $x$ e $y$ se denota por $max(x,y)$. Así $max(-1,3)=max(3,3)$ y $max(-1,-4)=max(-4,-1=-1)$. El mínimo de $x$ e $y$ se denota por $min(x,y)$. Demostrar que:
\begin{enumerate}[\bfseries 1.]
\item $max(x,y)=\dfrac{x+y+|y-x|}{2}$
\item $min(x,y)=\dfrac{x+y-|y-x|}{2}$
\end{enumerate}
Derivar una fórmula para $max(x,y,z)$ y $min(x,y,z),$ utilizando. por ejemplo, $max(x,y,z)=max(x,max(x,y))$ \\\\
Demostración.- \; Por definición de valor absoluto se tiene:
\begin{equation}
|x-y| = \left\lbrace
\begin{array}{crr}
x-y& si, & x\geq y\\
y-x& si, & x \leq y
\end{array}
\right.
\end{equation}
Por lo tanto 
\begin{itemize}
\item $max(x,y) = \dfrac{x+y+|x-y|}{2}= \dfrac{x+y+x-y}{2} =\dfrac{2x}{2}=x$\\
\item $max(x,y) = \dfrac{x+y+|x-y|}{2}= \dfrac{x+y+y-x}{2} =\dfrac{2y}{2}=y$
\end{itemize} 
La demostración es parecido para para $min(x,y)$\\
Se deriva una formula para $mas(x,y,z)=max(x,max(y,z))$ de la siguiente manera\\
$$max(x,max(y,z))=\dfrac{x+\dfrac{y+z+|y-z|}{2} + \left| x - \dfrac{y+z+|y-z|}{2} \right|}{2}$$\\\\ 
\end{teo}

%teorema 4.12
\begin{teo}
Demostrar que $|a|=|-a|$\\\\
Demostración.- \; Si $a\geq 0$, para $|a|^2$ entonces $a^2$, en virtud del teorema 2.5 $(-a)^2=|-a|^2$, así se demuestra que $|a|=|-a|$. Luego es evidente para $a \leq 0$\\\\
\end{teo} 

%teorema 4.13
\begin{teo} \footnote{Calculo infinitesimal, Michael Spivak, pag 23-24}
Demostrar que si $$|x - x_0|<\dfrac{\epsilon}{2} \hspace{0.5cm} y \hspace{0.5cm} |y - y_0|<\dfrac{e}{2}$$ entonces $$|(x+y)-(x_o + y_0)|<\epsilon,$$ $$|(x-y)-(x_o - y_0)|<\epsilon.$$
Demostración.- \; primeramente si $|(x+y)-(x_o + y_0)|= |(x-x_0)+(y-y_0)|$, por desigualdad triangular e hipótesis, $$ |(x-x_0)+(y-y_0)|\leq |x - x_0| + |y - y_0| <\frac{\epsilon}{2} + \frac{\epsilon}{2} =\epsilon$$
Demostramos de similar manera y por teorema 1.7, $$|(x-y)-(x_o - y_0)| = |(x - x_0)-(y - y_0)| \leq |x - x_0| + |y - y_0|< \frac{\epsilon}{2} + \frac{\epsilon}{2}=\epsilon$$\\
\end{teo}

%teomrea 4.14 
\begin{teo}\footnote{Calculo infinitesimal, Michael Spivak, pag 23-24}
Demostrar que si $$|x-x_0|< min \left( 1,\dfrac{\epsilon}{2(|y_0|+1)} \right) \hspace{0.5cm} y \hspace{0.5cm} |y-y_0|< \dfrac{\epsilon}{2(x_0+1)}$$ entonces $|xy - x_0 y_0 |< \epsilon.$\\\\
La primera igualdad de la hipótesis significa precisamente que: $$|x-x_0|<1 \hspace{0.5cm} y \hspace{0.5cm} |x - x_0| \dfrac{\epsilon}{2(|y_0|+1)}$$\\\\
Demostración.- \; puesto que $|x-x_0| < 1$ se tiene $$|x|-|x_0| \leq |x-x_0| < 1$$ de modo que $$|x| < 1 + |x_0|$$ Así pues 
\begin{center}
\begin{tabular}{r c l l}
$|xy - x_0 y_0|$&=&$| xy -xy_0 + xy_0 -x_0 y_0 |$&\\\\
&=&$|x(y - y_0) + y_0(x - x_0)|$&\\\\
&$\leq$&$|x| \cdot |y - y_0| + |y_0| \cdot |x - x_0|$&\\\\
&$<$&$(1+|x_0|) \cdot \dfrac{\epsilon}{2(|x_0|+1|)} +  |y_0| \cdot \dfrac{\epsilon}{2(|y_0|+1|)}$&ya que $ |y_0|\dfrac{\epsilon}{2(|y_0|+ 1|)} < |y_0| \dfrac{\epsilon}{2|y_0|}$ para $|y_0|\neq 0$\\\\
&$\leq$&$\dfrac{\epsilon}{2} + \dfrac{\epsilon}{2}$& ó $|y_0|\dfrac{\epsilon}{2(|y_0|+ 1|)} = \dfrac{\epsilon}{2}$ si $|y_0|=0$ \\\\
&$=$&$\epsilon$&\\\\
\end{tabular}
\end{center}
\end{teo}

%teorema 4.15
\begin{teo}
Demostrar que si $y_0 \neq 0$ y 
$$|y - y_0|< min\left( \dfrac{|y_0|}{2}, \dfrac{\epsilon |y_0|^2}{2} \right),$$ 
entonces $y \neq 0$ y $$\left| \dfrac{1}{y} - \dfrac{1}{|y_0|} \right|$$\\\\
Demostración.- \; Se tiene $$|y_0| - |y| < |y - y_0| < \dfrac{y_0}{2},$$ de modo que $|y|<\dfrac{|y_0|}{2}$. En particular, $y \neq 0,$ y $$\dfrac{1}{|y|} < \dfrac{2}{y_0}.$$ Así pues $$\left| \dfrac{1}{y} - \dfrac{1}{y_0} \right| = \dfrac{|y_0 - y|}{|y| \cdot |y_0|} < \dfrac{2}{y_0} \cdot \dfrac{1}{|y_0|} \cdot \dfrac{\epsilon |y_o|^2}{2} = \epsilon$$\\\\
\end{teo}

%teorema 4.16
\begin{teo}
Sustituir los interrogantes del siguiente enunciado por expresiones que encierren $\epsilon, \; x_0 \; e \; y_0$ de tal manera que la conclusión sea válida:\\
Si $y_0$ y $$|y - y_0|< ? \hspace{0.5cm} y \hspace{0.5cm} |x - x_0|< ?$$ entonces $y \neq 0$ y $$\left| \dfrac{x}{y} - \dfrac{x_0}{y_0} \right|< \epsilon$$ \\
Sea $|x \dfrac{1}{y} - x_0 \dfrac{1}{y_0}|< \epsilon$ entonce $|x \cdot y^{-1} - x_o \cdot y_0^{-1}| < \epsilon$ por teorema 4.14 $$|x- x_0| < min \left( 1, \dfrac{\epsilon}{2(|y_0^{-1}|+1)} \right) \hspace{0.5cm} y \hspace{0.5cm} |y^{-1} - y_0^{-1}| < \dfrac{\epsilon}{2(|x_0|+1)}  $$ luego por teorema 4.15 $$|y - y_0| < min \left( \dfrac{\dfrac{\epsilon}{2 (|x_o|)+1} \cdot |y_0|^2}{2} \right) = \min \left( \dfrac{\epsilon \cdot |y_0|^2}{4(|x_o|+1)} \right)$$ \\
\end{teo}



\section[Demostraciones]{Demostraciones \footnote{Calculus Vol 1, Tom Apostol, pag 53-54}}
\begin{teo}
$|x|=0$ si sólo si $x=0$\\\\
Demostración.- \; Si $|x|=0$, por definición $x=0$. Luego, si $x=0$, entonces por teorema $\sqrt{x^2}=\sqrt{0^2}=0$\\\\
\end{teo}

\begin{teo}
$|x-y|=|y-x|$\\\\
Demostración.- \; Por teorema $|x-y|=\sqrt{(x-y)^2}=\sqrt{x^2-2xy+y^2}=\sqrt{y^2-2yx+x^2}=\sqrt{(y-x)^2}=|y-x|$\\\\ 
\end{teo}

\begin{teo}
$|x|^2=x^2$\\\\
Demostración.- \; Si $|x|^2$, por teorema $\left( \sqrt{x^2} \right)^2$, por propiedad de potencia $x^2$\\\\
\end{teo}


\section[Ejercicios]{Ejercicios \footnote{Calculus Vol 1, Tom Apostol, pag 53-54}}
% ejercicio 4.4 
\begin{ej}
Cada desigualdad $(a_i)$, de las escritas a continuación, equivale exactamente a una desigualdad $(b_j)$. Por ejemplo, $|x|<3$ si y sólo si $-3<x<3$ y por tanto $(a_1)$ es equivalente a $(b_2)$. Determinar todos los pares equivalentes.\\
\begin{multicols}{2}
\begin{tabular}{r c l}
$|x|<3$&$\longrightarrow$&$-3<x<3$\\\\
$|x-1|<3$&$\longrightarrow$&$-2<x<4$\\\\
$|3-2x|<1$&$\longrightarrow$&$1<x<2$\\\\
$|1+2x|\leq 1$&$\longrightarrow$&$-1\leq x \leq 0$\\\\
$|x-1|>2$&$\longrightarrow$&$x>4 \; \lor \; x<-1$\\\\
\end{tabular}
\begin{tabular}{l c l}
$|x+2| \geq 5$&$\longrightarrow$&$x\geq 3 \; \lor \; x\leq -7$\\\\
$|5-x^{-1}|<1 $&$\longrightarrow$&$4<x<6$\\\\
$|x-5|<|x+1|$&$\longrightarrow$&$x>2$\\\\
$|x^2-2|\leq 1$&$\longrightarrow$&$-\sqrt{3} \leq x \leq -1 \; ó \; 1\leq x \leq \sqrt{3}$\\\\
$x<x^2-12<4x$&$\longrightarrow$&$\dfrac{1}{6}<x<\dfrac{1}{4}$\\\\
\end{tabular}
\end{multicols}
\end{ej}

\begin{ej}
Decidir si cada una de las siguientes afirmaciones es cierta o falsa. En cada caso razonar la decisión.
\begin{enumerate}[\bfseries a)]
\item $x<5$ implica $|x|<5$\\\\
Es falso ya que $|-6|<5$ entonces $6>5$.\\\\
\item $|x-5|<2$ implica $3<x<7$\\\\
Es verdad ya que por teorema $-2 < x-5 < 2$ entonces $3<x<7$.\\\\
\item $|1 + 3x| \leq 1$ implica $x > - \dfrac{2}{3}$ \\\\
es verdad ya que $-1\leq 1 + 3x \leq 1$ entonces $-\dfrac{2}{3} \leq x \leq 0$.\\\\
\item No existe número real $x$ para el que $|x-1|= |x-2|$\\\\
Es falso ya que se cumple para $\dfrac{3}{2}$.\\\\
\item Para todo $x>0$ existe un $y>0$ tal que $|2x + y|=5$\\\\
Es falso ya que si tomas $x=3$ será $y<0$\\\\
\end{enumerate}
\end{ej}