\chapter{Rectas}
\begin{tcolorbox}
\begin{post}[Postulado de la distancia]
$$\mbox{A cada par de puntos diferentes corresponde un número positivo único.}$$
\end{post}
\begin{def.}
La distancia entre dos puntos es el número obtenido mediante el postulado de la distancia Si los puntos son $P$ y $Q$, entonces indicamos la distancia por $PQ.$ \\\\
\end{def.}
\begin{post}[Postulado de la regla]
Podemos establecer una correspondencia entre los puntos de una recta y los números de manera que:
\begin{enumerate}[\bfseries 1)]
\item a cada de la recta correspondiente exactamente un número real;
\item a cada número real corresponde exactamente un punto de la recta; y
\item la distancia entre dos puntos cualesquiera es el valor absoluto de la diferencia de los números correspondientes.
\end{enumerate}
\end{post}
\begin{def.}
Una correspondencia como la descrita en el postulado de la regla se llama un sistema de coordenadas. El número correspondiente a un punto dado se llama la coordenada del punto
\end{def.}
\end{tcolorbox}
\section{Conjunto de problemas 2-5}
%problema 3
\begin{prob}
Utilizar el postulado de la regla para hallar la distancia entre los pares de puntos con las coordenadas siguientes:
\begin{enumerate}[\bfseries a)]
\item 0 y 8\\
\item 8 y 0\\
\item 0 y -8\\
\item -5 y -7\\
\item $- \dfrac{2}{3}$ y $\dfrac{1}{3}$\\
\item $\sqrt{2}$ y $\sqrt{5}$\\
\item $\sqrt{3}$ y $- \sqrt{5}$\\
\item $x$ e $y$\\
\item 2a y -a\\
\item 0 y x\\
\end{enumerate}
\end{prob}