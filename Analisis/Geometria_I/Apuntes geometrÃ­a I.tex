\documentclass[10pt]{book}
\usepackage[text=17cm,left=2.5cm,right=2.5cm, headsep=20pt, top=2.5cm, bottom = 2cm,letterpaper,showframe = false]{geometry} %configuración página
\usepackage{latexsym,amsmath,amssymb,amsfonts} %(símbolos de la AMS).7
\parindent = 0cm  %sangria
\usepackage[T1]{fontenc} %acentos en español
\usepackage[spanish]{babel} %español capitulos y secciones
\usepackage{graphicx} %gráficos y figuras.

%--------------------GRÀFICOS--------------------------

\usepackage{tkz-fct}
\usepackage{tkz-euclide}


%---------------FORMATO de letra--------------------%

\usepackage{lmodern} % tipos de letras
\usepackage{titlesec} %formato de títulos
\usepackage[backref=page]{hyperref} %hipervinculos
\usepackage{multicol} %columnas
\usepackage{tcolorbox, empheq} %cajas
\usepackage{enumerate} %indice enumerado
\usepackage{marginnote}%notas en el margen
\tcbuselibrary{skins,breakable,listings,theorems}
\usepackage[Bjornstrup]{fncychap}%diseño de portada de capitulos
\usepackage[all]{xy}%flechas
\counterwithout{footnote}{chapter}
\usepackage{xcolor}


%------------------------------------------

\newtheorem{axioma}{\large\textbf{AXIOMA}}
\newtheorem{teo}{\textbf{TEOREMA}}[chapter]%entorno para teoremas
\newtheorem{ejem}{{\textbf{EJEMPLO}}}[chapter]%entorno para ejemplos
\newtheorem{def.}{\textbf{Definición}}[chapter]%entorno para definiciones
\newtheorem{post}{\textbf{POSTULADO}}[part]%entorno de postulados
\newtheorem{col.}{\textbf{COROLARIO}}[chapter]
\newtheorem{ej}{\textbf{EJERCICIO}}[chapter]
\newtheorem{prop}{\textbf{PROPIEDADES}}[chapter]
\newtheorem{lema}{\textbf{LEMA}}[chapter]


%---------------------------------

\titleformat*{\section}{\LARGE\bfseries\rmfamily}
\titleformat*{\subsection}{\Large\bfseries\rmfamily}
\titleformat*{\subsubsection}{\large\bfseries\rmfamily}
\titleformat*{\paragraph}{\normalsize\bfseries\rmfamily}
\titleformat*{\subparagraph}{\small\bfseries\rmfamily}

%------------------------------------------

\renewcommand{\labelenumi}{\Roman{enumi}.}%primer piso II) enumerate
\renewcommand{\labelenumii}{\arabic{enumii}$)$}%segundo piso 2)
\renewcommand{\labelenumiii}{\alph{enumiii}$)$}%tercer piso a)
\renewcommand{\labelenumiv}{$\bullet$}%cuarto piso (punto)

%----------Formato título de capítulos-------------

\usepackage{titlesec}
\renewcommand{\thechapter}{\arabic{chapter}}
\titleformat{\chapter}[display]
{\titlerule[2pt]
\vspace{4ex}\bfseries\sffamily\huge}
{\filleft\Huge\thechapter}
{2ex}
{\filleft}

\usepackage[htt]{hyphenat}

\begin{document}
\normalfont
\input xy
\xyoption{all}
\author{\Large por FODE}
\title{GEOMETRÍA EUCLIDIANA \\ \small Apuntes}
\date{}
\pagestyle{empty}
\maketitle
\thispagestyle{empty}
\let\cleardoublepage\clearpage
\tableofcontents 								%indice


%------------------------------------------
 
\let\cleardoublepage\clearpage

%------------------------------------------
\chapter{Rectas y Planos}
\begin{center}
\textbf{¿Como podría definirse un punto, una recta, un plano y el espacio?}\\
\end{center}
Estos cuatro conceptos son muy importantes en el estudio de la geometría; para estos términos no pueden definirse en términos simples y los llamamos términos no definidos ó primitivos.
\paragraph{Punto:} Ubicación, sin longitud, anchura, ni altura.
\paragraph{Recta:} Longitud ilimitada, derecha, sin grosor, ni extremos.
\paragraph{Plano:} Ilimitado, continuo en todas direcciones, llano, sin grosor.
\paragraph{Espacio:} Ilimitado, sin logitud, anchura, ni altura.\\\\
\begin{tcolorbox}
\begin{def.}[El espacio]
es la colección de todos los puntos.
\end{def.}
\end{tcolorbox}

\subsection*{Notación}
\begin{itemize}
\item Los puntos se representan con letras mayúsculas A, B, C, etc.
\begin{center}
\begin{tikzpicture}
\draw[black] (0,0)node[]{$\bullet$} node[below]{$P$};
\end{tikzpicture}
\end{center}
\item Las rectas los denotamos por: \textit{L} ó $\overline{AB}$.
\begin{center}
\begin{tikzpicture}
\draw[<->](0,0)--(5,0)node[right]{$\textit{L} \; ó \; \overline{AB}$}; 
\draw(1,0)node[]{$\bullet$}node[below]{$A$} (4,0)node[]{$\bullet$}node[below]{$B$};
\end{tikzpicture}
\end{center}
\item Los planos se representará con letras cursivas: \textit{P}.
\end{itemize}
\begin{center}
\begin{tikzpicture}[scale=0.4]
\draw(0,0)--(10,0)--(12,5)node[right]{$\textit{P}$}--(2,5)--(0,0);
\draw(2,2)node[]{$\bullet$}node[right]{$P$} ((5,3)node[]{$\bullet$}node[right]{$R$} (8,4)node[]{$\bullet$}node[right]{$Q$};
\end{tikzpicture}
\end{center}
\begin{tcolorbox}
\begin{def.}Puntos y rectas.
\begin{itemize}
\item \textbf{Los puntos colineales} son puntos que están en la misma recta.
\item \textbf{Los puntos coplanares} son puntos que se encuentram en el mismo plano
\item \textbf{Las rectas intersecantes} son dos rectas con un punto en común.
\item \textbf{Las rectas concurrentes} son tres ó más rectas coplanares que tienen un punto en común.
\end{itemize}
\end{def.}
\end{tcolorbox}

\begin{tcolorbox}
\begin{post}[Postulado de la distancia] A cada par de puntos diferentes le corresponde un número real positivo único.\\
\end{post}

\begin{def.}
el número dado por el postulado 1 se llama distancia entre los puntos $P$ y $Q$; lo que se denota por $PQ$.\\
\begin{center}
Si $P=Q$ entonces $PQ=0$
\end{center}
\end{def.}
\end{tcolorbox}

\begin{tcolorbox}
\begin{post}[Postulado de la regla] Podemos establecer una correspondencia entre los puntos de una recta y los números reales de manera que:
\begin{enumerate}[\bfseries a)]
\item A cada punto de la recta corresponde exactamente un número real,
\item a cada número real corresponde exactamente un punto de la recta y
\item la distancia entre los dos puntos cualesquiera es el valor absoluto de su diferencia.\\
\end{enumerate}
\end{post}
\begin{def.}
La correspondencia descrita por el postulado 2 se llama sistema de coordenadas.
\begin{center}
\begin{tikzpicture}[scale=2]
\tkzDefPoints{-1/0/P,0/0/Q,1/0/R,2/0/S}
\tkzDefPoints{-1/0/-1,0/0/0,1/0/1,2/0/2}
\tkzDrawLine[<->](P,S)
\tkzLabelPoints(-1,0,1,2)
\tkzDrawPoints(P,Q,R,S)
\tkzLabelPoints[above](P,Q,R,S)
\end{tikzpicture}
\end{center}
\end{def.}
\end{tcolorbox}

\begin{tcolorbox}
\begin{post}[Posulado de la colocación de la regla]
Dados dos puntos $P$ y $Q$ de una recta, se puede escoger el sistema de coordenadas de manera que la coordenada de $P$ sea cero y la coordenada de $Q$ sea positiva.\\
\end{post}
\begin{def.} $B$ esta $"$entre$"$ $A$ y $B$ si,
\begin{enumerate}[\bfseries i)]
\item $A,$ $B$ y $C$ son colineales.
\item $AB + BC = AC$\\
\end{enumerate}
\end{def.}
\begin{def.}
Un segmento , $\overline{AB}$ es le conjunto de los puntos $A$ y $B$ de todos los puntos que están entre $A$ y $B$.\\
\end{def.}
\begin{def.}
Un rayo $\overrightarrow{AB}$ es la unión del segmento $\overline{AB}$ y de todos los puntos $C$ para los cuales es verdad que $B$ está entre $A$ y $C,$
\begin{center}
\begin{tikzpicture}[scale=1]
\tkzDefPoints{0/0/A,2/0/B,4/0/C}
\tkzDrawLine[->](A,C)
\tkzDrawPoints(A,B,C)
\tkzLabelPoints(A,B,C)
\end{tikzpicture}
\end{center}
\end{def.}
\end{tcolorbox}

\begin{tcolorbox}
\begin{post}[Postulado de la recta]
Por dos puntos puntos distintos cualesquiera pasa exactamente una recta.
\begin{center}
$AB$ se llama longitud del segmento $\overline{AB}$\\
\end{center}
\end{post}
Correspondencia biunivoca .- correspondencia uno a uno
\end{tcolorbox}

\begin{teo}[Teorema de la localización de puntos]
Sea $\overrightarrow{AB}$
un rayo y $x\in \mathbb{R}^+$. Entonces existe exactamente un punto $P\in \overrightarrow{AB} / AP = x$\\\\
Demostración.- \; Dada la recta $\overleftrightarrow{AB}$; pro el postulado de la colocación de la regla podemos elegir un sistema de coordenadas de $A$ sea cero y la coordenada de $B$ sea un número positivo $r$
\begin{center}
\begin{tikzpicture}[scale=1]
\tkzDefPoints{0/0/A,5/0/B,8/0/P}
\tkzDefPoints{0/0/0,5/0/r,8/0/x}
\tkzDrawLine[<->](0,x)
\tkzLabelPoints(0,r,x)
\tkzDrawPoints(A,B,P)
\tkzLabelPoints[above](A,B,P)
\end{tikzpicture}
\end{center}
Sea $P$ el punto cuyo coordenada es $x$; como $x\in \mathbb{R}^+$ entonces $x\in \overrightarrow{AB}$ y $AP=|x-0|=x$; pero $x>0$. La unicidad de $P$ se da por el postulado de la regla.\\\\
\end{teo}

\begin{tcolorbox}
\begin{def.}
Un punto $B$ se llama punto medio de un segmento $\overline{AC}$, si $B$ está entre $A$ y $B$ tal que $AB=BC$\\
\begin{center}
Decimos que el punto medio de un segmento \textbf{biseca} al segmento. \\
\end{center}
\end{def.}
\end{tcolorbox}

\begin{teo}
Todo segmento tiene exactamente un punto medio.\\\\
Demostración.- \; Si $B$ es el punto medio de $\overline{AC}$ entonces debe cumplirse: 
\begin{center}
$\left.
\begin{array}{rcl}
AB + BC & = & AC\\ 
AB & = & BC
\end{array}
\right\}
\Rightarrow AB = \dfrac{AC}{2}$
\end{center}
Luego por el Teorema 1, el rayo $\overrightarrow{AC}$ con $x=\dfrac{AC}{2} \in \mathbb{R}^+$ hay exactamente un punto $B$ tal que $AB=\dfrac{AC}{2}$. Así $\overline{AC}$ tiene exactamente un punto medio.\\\\ 
\end{teo}



\end{document}